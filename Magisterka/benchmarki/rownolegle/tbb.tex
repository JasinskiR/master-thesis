\section{Implementacje w C++ (TBB)}
\subsection{Struktura i organizacja kodu}
Implementacje TBB zachowują proceduralną strukturę z minimalnymi modyfikacjami względem wersji OpenMP:
\begin{lstlisting}[language=C++, style=VS2017,  caption={Implementacja TBB - struktura kodu}, label={lst:TBB-structure}]
// TBB: funkcyjny interfejs + globalne zmienne
#include <TBB/parallel_reduce.h>
#include <TBB/task_scheduler_init.h>

static double (*data) = (double*)malloc(sizeof(double) * SIZE);

int main() {
    // TBB: explicit scheduler initialization  
    TBB::task_scheduler_init init(TBB::task_scheduler_init::automatic);
    
    parallel_algorithm();
    
    free(data); // Ręczne zarządzanie pamięcią
}
}
\end{lstlisting}
\begin{itemize}
    \item Minimalne zmiany strukturalne - zachowanie proceduralnego stylu programowania bez konieczności znaczącej przebudowy istniejącego kodu.
    
    \item Jawna inicjalizacja planisty zadań - konieczność ręcznego utworzenia i konfiguracji obiektu planisty (\texttt{task\_scheduler\_init}) w celu zarządzania wątkami.
    
    \item Równoległość oparta na funkcjach - zastąpienie dyrektyw kompilatora wywołaniami funkcji bibliotecznych, takich jak \texttt{parallel\_for} czy \texttt{parallel\_reduce}.
    
    \item Te same problemy z zarządzaniem pamięcią - biblioteka nie wprowadza dodatkowych zabezpieczeń; nadal istnieje ryzyko błędów takich jak wycieki pamięci, przepełnienie bufora czy dostęp do już zwolnionej pamięci.
\end{itemize}
  
\subsection{Zarządzanie pamięcią}
TBB nie wprowadza ulepszeń w zarządzaniu pamięcią - wykorzystuje identyczny model jak OpenMP z tymi samymi problemami.

\subsection{Mechanizmy równoległości}
TBB zastępuje dyrektywy OpenMP funkcyjnym API z work-stealing scheduler:
\begin{lstlisting}[language=C++, style=VS2017,  caption={Implementacja TBB - równoległość}, label={lst:TBB-parallelism}]
/// TBB: funkcyjny interfejs + work-stealing scheduler
void parallel_algorithm() {
    // TBB: parallel reduction with lambda
    double sum = TBB::parallel_reduce(
        TBB::blocked_range<size_t>(0, SIZE),
        0.0,
        [&](const TBB::blocked_range<size_t>& r, double local_sum) {
            for (size_t i = r.begin(); i != r.end(); ++i) {
                local_sum += compute_element(i);
            }
            return local_sum;
        },
        std::plus<double>()
    );
    
    // TBB: parallel for with automatic load balancing
    TBB::parallel_for(
        TBB::blocked_range<size_t>(0, SIZE),
        [&](const TBB::blocked_range<size_t>& r) {
            for (size_t i = r.begin(); i != r.end(); ++i) {
                process_element(i);
            }
        }
    );
}
\end{lstlisting}
\begin{itemize}
    \item Funkcyjny interfejs programowania - zastosowanie wyrażeń lambda i funkcji wyższego rzędu zamiast dyrektyw kompilatora, co sprzyja większej elastyczności kompozycyjnej.
    
    \item Harmonogram z kradzieżą pracy \eng{work-stealing scheduler} - automatyczne równoważenie obciążenia pomiędzy wątkami poprzez dynamiczne przydzielanie zadań zależnie od dostępnych zasobów.
    
    \item Podziały blokowe - automatyczny podział przestrzeni danych na zakresy, co ułatwia równoległe przetwarzanie dużych kolekcji bez konieczności ręcznego zarządzania iteracjami.
    
    \item Bezpieczeństwo typów - silniejsze mechanizmy typizacji w porównaniu do OpenMP, co zmniejsza liczbę błędów w czasie kompilacji.
    
    \item Większy narzut składniowy - bardziej rozwlekła i złożona składnia niż w przypadku OpenMP, co może wpływać na czytelność i prostotę kodu.
\end{itemize}

\subsection{Specyfika poszczególnych benchmarków}
TBB implementacje różnią się głównie sposobem wyrażenia równoległości, zachowując identyczną logikę algorytmów:
\begin{itemize}
    \item EP: Zastąpienie \texttt{\#pragma omp} dla redukcji przez \texttt{parallel\_reduce} z lambda
    \item CG: Konwersja operacji \texttt{sparse\_matrix} na \texttt{parallel\_for} z \texttt{blocked\_range}
    \item IS: Implementacja sortowania kubełkowego przez \texttt{parallel\_for} z synchronizacją przez mutexy
\end{itemize}
