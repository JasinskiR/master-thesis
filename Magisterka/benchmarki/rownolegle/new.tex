\section{Implementacje w C++ (nowoczesne podejście)}
\subsection{Struktura i organizacja kodu}
Nowoczesne implementacje C++ wykorzystują obiektowy design z RAII i enkapsulacją:
\begin{lstlisting}[language=C++, style=VS2017,  caption={Implementacja nowoczesnego C++ - struktura kodu}, label={lst:modern-cpp-structure}]
// New C++: RAII + enkapsulacja
class NPBBenchmark {
    std::vector<double> data_;        // Automatic memory management
    std::vector<int> indices_;        // vs malloc/free
    BenchmarkResults results_;
    
public:
    virtual void run() = 0;
    virtual bool verify() const = 0;
    
    // RAII ensures automatic cleanup
    ~NPBBenchmark() = default;
};

}
\end{lstlisting}
\begin{itemize}
    \item RAII (Resource Acquisition Is Initialization) - automatyczne zarządzanie zasobami przy użyciu konstruktorów i destruktorów, co eliminuje potrzebę jawnego zwalniania pamięci oraz innych zasobów systemowych.
    
    \item Enkapsulacja - hermetyzacja danych oraz logiki w obrębie klas i struktur, co sprzyja modularności, izolacji błędów i możliwości wielokrotnego wykorzystania kodu.
    
    \item Bezpieczeństwo typów - wykorzystanie nowoczesnych konstrukcji językowych C++, takich jak typy wyliczeniowe, silna typizacja i szablony, pozwalające na wczesne wykrywanie błędów.
    
    \item Zarządzanie zasobami - użycie inteligentnych wskaźników (\texttt{std::unique\_ptr}, \texttt{std::shared\_ptr}) oraz kontraktów programistycznych (np. \texttt{noexcept}, \texttt{[[nodiscard]]}), co zwiększa niezawodność i bezpieczeństwo kodu.
\end{itemize}

  
\subsection{Zarządzanie pamięcią}
\begin{lstlisting}[language=C++, style=VS2017,  caption={Implementacja nowoczesnego C++ - zarządzanie pamięcią}, label={lst:modern-cpp-memory}]
// Modern C++: RAII + smart pointers vs manual memory
class NPBBenchmark {
    std::vector<double> data_;                    
    std::unique_ptr<WorkingMemory> working_mem_;  
    
public:
    NPBBenchmark(size_t size) : data_(size) {}   
    
    void worker_task() {
        std::vector<double> local_data(size_);
        /* computation... */
    }  // Automatic cleanup on scope exit
};
\end{lstlisting}

\subsection{Mechanizmy równoległości}
\begin{lstlisting}[language=C++, style=VS2017,  caption={Implementacja nowoczesnego C++ - równoległość}, label={lst:modern-cpp-parallelism}]
// Nowy C++: std::thread + RAII
void parallel_computation() {
    std::vector<std::thread> threads;
    
    // Exception-safe thread creation
    for (int i = 0; i < num_threads_; ++i) {
        threads.emplace_back([this, i]() { worker_task(i); });
    }
    
    // RAII ensures proper cleanup
    for (auto& t : threads) { t.join(); }
}
\end{lstlisting}

