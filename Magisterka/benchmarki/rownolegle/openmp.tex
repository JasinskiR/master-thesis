\section{Implementacje w C++ (OpenMP)}
\subsection{Struktura i organizacja kodu}
Implementacje C++ z OpenMP następują klasyczny wzorzec proceduralny wywodzący się z~oryginalnych implementacji w Fortranie:

% \begin{lstlisting}[language=Rust, caption={Implementacja benchmarku CG w języku Rust}, label={lst:cg_rust}]
% \begin{lstlisting}[language=Rust, caption={Struktura kodu benchmarków w języku Rust}, label={lst:rust_structure}]
\begin{lstlisting}[language=C++, caption={Struktura kodu benchmarków w języku C++ z OpenMP}, label={lst:openmp_structure}]
// Globalne zmienne statyczne dla wszystkich benchmarków
#if defined(DO_NOT_ALLOCATE_ARRAYS_WITH_DYNAMIC_MEMORY_AND_AS_SINGLE_DIMENSION)
static DataType primary_array[MAX_SIZE];
static DataType secondary_array[MAX_SIZE];
#else
static DataType (*primary_array)=(DataType*)malloc(sizeof(DataType)*(MAX_SIZE));
static DataType (*secondary_array)=(DataType*)malloc(sizeof(DataType)*(MAX_SIZE));
#endif

// Funkcje proceduralne operujące na globalnych danych
static void algorithm_core(/* parametry */);
static void initialize_data(/* parametry */);
static void verify_results(/* parametry */);

int main(int argc, char **argv){
    // Główna logika bez enkapsulacji
    initialize_data(/* argumenty */);
    algorithm_core(/* argumenty */);
    verify_results(/* argumenty */);
}
\end{lstlisting}
\begin{itemize}
    \item Globalny stan - wszystkie dane przechowywane są jako zmienne globalne, co upraszcza współdzielenie zasobów pomiędzy wątkami.
    
    \item Styl proceduralny - funkcje operują bezpośrednio na globalnych strukturach danych, zgodnie z tradycyjnym podejściem programowania proceduralnego.
    
    \item Minimalna enkapsulacja - brak hermetyzacji danych i logiki prowadzi do luźnej struktury kodu oraz ograniczonej kontroli nad jego modyfikacjami.
    
    \item Kompatybilność wsteczna - zachowanie zgodności z kodem Fortran ułatwia migrację i integrację z istniejącymi systemami HPC.
    
    \item Warunkowa alokacja - możliwość wyboru pomiędzy alokacją statyczną a dynamiczną umożliwia dostosowanie strategii zarządzania pamięcią do charakterystyki platformy.
\end{itemize}
  
\subsection{Zarządzanie pamięcią}
\begin{lstlisting}[language=C++, caption={Zarządzanie pamięcią w benchmarkach C++ z OpenMP}, label={lst:openmp_memory}]
// Warunkowa strategia alokacji pamięci
#if defined(DO_NOT_ALLOCATE_ARRAYS_WITH_DYNAMIC_MEMORY_AND_AS_SINGLE_DIMENSION)
static int colidx[NZ];
static int rowstr[NA+1];
static double a[NZ];
static double x[NA+2];
#else
static int (*colidx)=(int*)malloc(sizeof(int)*(NZ));
static int (*rowstr)=(int*)malloc(sizeof(int)*(NA+1));
static double (*a)=(double*)malloc(sizeof(double)*(NZ));
static double (*x)=(double*)malloc(sizeof(double)*(NA+2));
#endif

int main(int argc, char **argv) {
    // Brak sprawdzania błędów alokacji
    // Brak jawnego zwalniania pamięci
    // Potencjalne wycieki pamięci przy wcześniejszym wyjściu
    
    // Warunkowe zwalnianie (jeśli w ogóle)
    #if !defined(DO_NOT_ALLOCATE_ARRAYS_WITH_DYNAMIC_MEMORY_AND_AS_SINGLE_DIMENSION)
    // Często brakuje free() calls
    #endif
}
\end{lstlisting}
\begin{itemize}
    \item Ręczne zarządzanie pamięcią - eksplicytne wywołania \texttt{malloc} i \texttt{free} bez wsparcia automatyzacji, co zwiększa ryzyko błędów programistycznych.
    
    \item Brak sprawdzania błędów alokacji - operacje przydziału pamięci mogą zakończyć się niepowodzeniem, a ich wyniki często nie są weryfikowane.
    
    \item Wycieki pamięci - brak gwarancji, że zaalokowane zasoby zostaną zwolnione, co prowadzi do stopniowego wzrostu zużycia pamięci.
    
    \item Niezdefiniowane zachowanie - możliwość wystąpienia błędów, takich jak użycie pamięci po jej zwolnieniu czy podwójne zwolnienie pamięci, które mogą prowadzić do niestabilności programu, naruszenia integralności danych lub poważnych luk w bezpieczeństwie.
    
    \item Przepełnienie bufora (buffer overflows) - brak automatycznego sprawdzania granic tablic sprzyja nadpisywaniu pamięci poza przydzielonym obszarem.
\end{itemize}
  
\subsection{Mechanizmy równoległości}
\begin{lstlisting}[language=C++, caption={Mechanizmy równoległości w benchmarkach C++ z OpenMP}, label={lst:openmp_parallelism}]
    static void parallel_computation(/* parametry */) {
        double local_sum = 0.0;
        
        // Klasyczne dyrektywy OpenMP
        #pragma omp parallel for reduction(+:local_sum) schedule(static)
        for (int64_t i = 0; i < problem_size; i++) {
            // Równoległe przetwarzanie z ręczną kontrolą synchronizacji
            local_sum += compute_element(i);
        }
        
        // Sekcje krytyczne dla complex operations
        #pragma omp parallel
        {
            double thread_local_data[THREAD_ARRAY_SIZE];
            
            #pragma omp for schedule(dynamic, chunk_size)
            for (int64_t j = 0; j < iterations; j++) {
                // Thread-local processing
            }
            
            #pragma omp critical
            {
                // Synchronization of shared state
                update_global_results(thread_local_data);
            }
        }
    }
\end{lstlisting}
\begin{itemize}
    \item Oparte na dyrektywach - wykorzystanie konstrukcji \texttt{\#pragma} do deklaratywnego sterowania równoległością w kodzie źródłowym.
    
    \item Ręczna synchronizacja - konieczność samodzielnego zarządzania sekcjami krytycznymi, barierami oraz mechanizmami ochrony danych współdzielonych.
    
    \item Jawne planowanie - programista ma bezpośrednią kontrolę nad strategiami podziału pracy, takimi jak \texttt{static}, \texttt{dynamic} czy \texttt{guided}.
    
    \item Model oparty na wątkach - równoległość realizowana przez bezpośrednie zarządzanie wątkami systemowymi, co wpływa na wydajność i kontrolę niskopoziomową.
    
    \item Dojrzały ekosystem - bogaty zbiór dyrektyw, opcji konfiguracyjnych i narzędzi wspierających optymalizację oraz profilowanie programów równoległych.
\end{itemize}
  
\subsection{Specyfika benchmarku EP}
\begin{lstlisting}[language=C++, caption={Implementacja benchmarku EP w języku C++ z OpenMP}, label={lst:ep_openmp}]
    #pragma omp parallel
    {
        double t1, t2, t3, t4, x1, x2;
        int kk, i, ik, l;
        double qq[NQ];        // Prywatna kopia q[0:NQ-1]
        double x[NK_PLUS];    // Lokalna tablica na stosie
    
        for (i = 0; i < NQ; i++) qq[i] = 0.0;
    
        #pragma omp for reduction(+:sx,sy)
        for(k=1; k<=np; k++){
            kk = k_offset + k;
            t1 = S;
            t2 = an;
            int thread_id = omp_get_thread_num();
    
            // Znajdowanie ziarna dla tego kk
            for(i=1; i<=100; i++){
                ik = kk / 2;
                if((2*ik)!=kk){t3=randlc(&t1,t2);}
                if(ik==0){break;}
                t3=randlc(&t2,t2);
                kk=ik;
            }
    
            // Generowanie liczb pseudolosowych
            if(timers_enabled && thread_id==0){timer_start(2);}
            vranlc(2*NK, &t1, A, x);
            if(timers_enabled && thread_id==0){timer_stop(2);}
    
            // Box-Muller transform
            if(timers_enabled && thread_id==0){timer_start(1);}
            for(i=0; i<NK; i++){
                x1 = 2.0 * x[2*i] - 1.0;
                x2 = 2.0 * x[2*i+1] - 1.0;
                t1 = pow2(x1) + pow2(x2);
                if(t1 <= 1.0){
                    t2 = sqrt(-2.0 * log(t1) / t1);
                    t3 = (x1 * t2);
                    t4 = (x2 * t2);
                    l = max(fabs(t3), fabs(t4));
                    qq[l] += 1.0;
                    sx = sx + t3;
                    sy = sy + t4;
                }
            }
            if(timers_enabled && thread_id==0){timer_stop(1);}
        }
    
        #pragma omp critical
        {
            for (i = 0; i <= NQ - 1; i++) q[i] += qq[i];
        }
    }
\end{lstlisting}

\subsection{Specyfika benchmarku CG}
\begin{lstlisting}[language=C++, caption={Implementacja benchmarku CG w języku C++ z OpenMP}, label={lst:cg_openmp}]
// Globalne zmienne statyczne - identyczne w OpenMP i TBB
#if defined(DO_NOT_ALLOCATE_ARRAYS_WITH_DYNAMIC_MEMORY_AND_AS_SINGLE_DIMENSION)
static int colidx[NZ];
static int rowstr[NA+1];
static int iv[NA];
static int arow[NA];
static int acol[NAZ];
static double aelt[NAZ];
static double a[NZ];
static double x[NA+2];
static double z[NA+2];
static double p[NA+2];
static double q[NA+2];
static double r[NA+2];
#else
static int (*colidx)=(int*)malloc(sizeof(int)*(NZ));
static int (*rowstr)=(int*)malloc(sizeof(int)*(NA+1));
static int (*iv)=(int*)malloc(sizeof(int)*(NA));
static int (*arow)=(int*)malloc(sizeof(int)*(NA));
static int (*acol)=(int*)malloc(sizeof(int)*(NAZ));
static double (*aelt)=(double*)malloc(sizeof(double)*(NAZ));
static double (*a)=(double*)malloc(sizeof(double)*(NZ));
static double (*x)=(double*)malloc(sizeof(double)*(NA+2));
static double (*z)=(double*)malloc(sizeof(double)*(NA+2));
static double (*p)=(double*)malloc(sizeof(double)*(NA+2));
static double (*q)=(double*)malloc(sizeof(double)*(NA+2));
static double (*r)=(double*)malloc(sizeof(double)*(NA+2));
#endif

int main(int argc, char **argv){
    // Inicjalizacja bez sprawdzania błędów
    makea(naa, nzz, a, colidx, rowstr, firstrow, lastrow, firstcol, lastcol, 
            arow, (int(*)[NONZER+1])(void*)acol, 
            (double(*)[NONZER+1])(void*)aelt, iv);

    // Główna pętla iteracyjna
    for(it = 1; it <= NITER; it++){
        if(timeron){timer_start(T_CONJ_GRAD);}
        conj_grad(colidx, rowstr, x, z, a, p, q, r, &rnorm);
        if(timeron){timer_stop(T_CONJ_GRAD);}

        // Sekwencyjne obliczenia norm
        norm_temp1 = 0.0;
        norm_temp2 = 0.0;
        for(j = 0; j < lastcol - firstcol + 1; j++){
            norm_temp1 = norm_temp1 + x[j]*z[j];
            norm_temp2 = norm_temp2 + z[j]*z[j];
        }
        norm_temp2 = 1.0 / sqrt(norm_temp2);
        zeta = SHIFT + 1.0 / norm_temp1;

        // Sekwencyjna aktualizacja x
        for(j = 0; j < lastcol - firstcol + 1; j++){
            x[j] = norm_temp2 * z[j];
        }
    }
}

static void conj_grad(int colidx[], int rowstr[], double x[], double z[], 
                        double a[], double p[], double q[], double r[], 
                        double* rnorm){
    int j, k;
    int cgit, cgitmax;
    double d, sum, rho, rho0, alpha, beta;

    cgitmax = 25;
    rho = 0.0;

    // Sekwencyjna inicjalizacja
    for(j = 0; j < naa+1; j++){
        q[j] = 0.0;
        z[j] = 0.0;
        r[j] = x[j];
        p[j] = r[j];
    }

    // OpenMP: Równoległe obliczenie rho
    #pragma omp parallel for reduction(+:rho)
    for(j = 0; j < lastcol-firstcol+1; j++){
        rho += r[j]*r[j];
    }

    for(cgit = 1; cgit <= cgitmax; cgit++){
        // OpenMP: Równoległe mnożenie macierz-wektor
        #pragma omp parallel for
        for(j = 0; j < lastrow-firstrow+1; j++){
            double sum = 0.0;
            for(k = rowstr[j]; k < rowstr[j+1]; k++){
                sum = sum + a[k]*p[colidx[k]];
            }
            q[j] = sum;
        }

        // OpenMP: Równoległe obliczenie dot product
        d = 0.0;
        #pragma omp parallel for reduction(+:d)
        for (j = 0; j < lastcol-firstcol+1; j++) {
            d += p[j]*q[j];
        }

        alpha = rho / d;
        rho0 = rho;
        rho = 0.0;

        // OpenMP: Równoległe aktualizacje wektorów z redukcją
        #pragma omp parallel for reduction(+:rho)
        for(j = 0; j < lastcol-firstcol+1; j++){
            z[j] += alpha*p[j];
            r[j] -= alpha*q[j];
            rho += r[j]*r[j];
        }

        beta = rho / rho0;

        // OpenMP: Równoległa aktualizacja p
        #pragma omp parallel for
        for(j = 0; j < lastrow-firstrow+1; j++){
            p[j] = r[j] + beta*p[j];
        }
    }

    // Obliczenie residuum
    sum = 0.0;
    #pragma omp parallel for
    for(j = 0; j < lastrow-firstrow+1; j++){
        double d = 0.0;
        for(k = rowstr[j]; k < rowstr[j+1]; k++){
            d = d + a[k]*z[colidx[k]];
        }
        r[j] = d;
    }

    #pragma omp parallel for reduction(+:sum)
    for(j = 0; j < lastcol-firstcol+1; j++){
        double d = x[j] - r[j];
        sum += d*d;
    }

    *rnorm = sqrt(sum);
}
\end{lstlisting}

\subsection{Specyfika benchmarku IS}
\begin{lstlisting}[language=C++, caption={Implementacja benchmarku IS w języku C++ z OpenMP}, label={lst:is_openmp}]
static void rank(/* parametry */) {
    // Parallel histogram construction
    #pragma omp parallel
    {
        int key_buff_ptr[MAX_KEY];
        
        #pragma omp for
        for (int i = 0; i < num_keys; i++) {
            key_buff_ptr[key_buff[i]]++;
        }
        
        // Complex synchronization for bucket sort
        #pragma omp barrier
        
        #pragma omp for
        for (int i = 0; i < num_buckets; i++) {
            // Redistribute keys
        }
    }
}
\end{lstlisting}