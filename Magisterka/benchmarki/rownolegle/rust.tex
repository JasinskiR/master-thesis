\section{Implementacje w języku Rust}
\subsection{Struktura i organizacja kodu}
Implementacje w języku Rust charakteryzują się proceduralną strukturą podobną do implementacji C++, wykorzystującą globalne stałe i funkcje standalone. Wszystkie benchmarki (EP, CG, IS) następują jednolity wzorzec architektoniczny oparty na funkcji main():

\begin{lstlisting}[language=Rust, caption={Struktura kodu benchmarków w języku Rust}, label={lst:rust_structure}]
pub struct NPBBenchmark {
// Globalne parametry problemu - stałe w czasie kompilacji
const CLASS: &str = "S";
const M: u32 = 24;
const MM: u32 = M - MK;
const NN: u32 = 1 << MM;
const NK: usize = 1 << MK;
const NPBVERSION: &str = "4.1";
const COMPILETIME: &str = "2024-01-15";

// Główna funkcja proceduralna
fn main() {
    let args: Vec<String> = env::args().collect();
    let num_threads = args.get(1)
        .map(|s| s.parse::<usize>().unwrap_or(1))
        .unwrap_or(1);

    // Konfiguracja Rayon thread pool
    rayon::ThreadPoolBuilder::new()
        .num_threads(num_threads)
        .build_global()
        .unwrap();

    // Proceduralna logika benchmarku
    algorithm_implementation();
    verification_and_results();
}

// Standalone functions operujące na globalnych danych
fn algorithm_implementation() { /* ... */ }
fn verification_and_results() { /* ... */ }
\end{lstlisting}
Kluczowe cechy organizacji kodu Rust:
\begin{itemize}
    \item Proceduralna struktura: główna logika w funkcji main() podobnie jak w C++
    \item Globalne stałe: parametry problemu definiowane jako const w czasie kompilacji
    \item Standalone functions: logika algorytmu w oddzielnych funkcjach, nie metodach
    \item Automatyczne zarządzanie pamięcią: Vec<T> zamiast malloc/free
    \item Rayon integration: jednolite użycie biblioteki Rayon we wszystkich benchmarkach
\end{itemize}


\subsection{Zarządzanie pamięcią}
Rust wykorzystuje system własności z automatycznym zarządzaniem pamięcią przez Vec<T>, eliminując ręczną alokację:
\begin{lstlisting}[language=Rust, caption={Zarządzanie pamięcią w benchmarkach NPB w języku Rust}, label={lst:rust_memory_management}]
impl NPBBenchmark {
fn main() {
    // Automatyczna alokacja przez Vec<T> - odpowiednik malloc w C++
    let mut key_array: Vec<i32> = vec![0; TOTAL_KEYS];
    let mut key_buff1: Vec<i32> = vec![0; MAX_KEY];
    let mut working_data: Vec<f64> = vec![0.0; PROBLEM_SIZE];
    
    // Thread-local storage dla bezpieczeństwa współbieżności
    thread_local! {
        static THREAD_X: RefCell<Vec<f64>> = RefCell::new(vec![0.0; NK_PLUS]);
    }
    
    // Główna logika - Vec<T> automatycznie zarządza pamięcią
    algorithm_core(&mut key_array, &mut working_data);
    
    // Automatyczna dealokacja po wyjściu z zakresu - brak potrzeby free()
}

fn algorithm_core(data: &mut Vec<i32>, working: &mut Vec<f64>) {
    // Borrowing pozwala na bezpieczny dostęp bez przenoszenia ownership
    parallel_processing(data, working);
    // Kompilator gwarantuje brak data races i use-after-free
}
\end{lstlisting}
Zalety modelu własności:
\begin{itemize}
    \item Automatyczna dealokacja: Vec<T> zwalnia pamięć automatycznie po wyjściu z zakresu
    \item Brak wycieków pamięci: gwarancja na poziomie kompilatora
    \item Abstrakcje bez narzutu kosztów: brak narzutu wydajnościowego
    \item Bezpieczeństwo wątków: mechanizm sprawdzania pożyczania eliminuje wyścigi danych
\end{itemize}

\subsection{Mechanizmy równoległości}
Wszystkie implementacje Rust wykorzystują bibliotekę Rayon dla spójnego podejścia do równoległości:
\begin{lstlisting}[language=Rust, caption={Równoległość w benchmarkach NPB w języku Rust}, label={lst:rust_parallelism}]
use rayon::prelude::*;

fn parallel_computation(data: &mut Vec<f64>, num_threads: usize) {
    // Konfiguracja thread pool - odpowiednik task_scheduler_init w TBB
    rayon::ThreadPoolBuilder::new()
        .num_threads(num_threads)
        .build_global()
        .unwrap();

    // Równoległe przetwarzanie chunks - odpowiednik #pragma omp parallel for
    let results: Vec<_> = data
        .par_chunks_mut(optimal_chunk_size())
        .map(|chunk| process_chunk(chunk))
        .collect();
        
    // Równoległa redukcja - odpowiednik reduction(+:sum) w OpenMP
    let final_result = data.par_iter()
        .fold(|| 0.0, |acc, &x| acc + compute_element(x))
        .reduce(|| 0.0, |acc1, acc2| acc1 + acc2);
}

fn process_chunk(chunk: &mut [f64]) -> f64 {
    // Bezpieczne przetwarzanie bez mutexów dzięki par_chunks_mut
    // Rayon gwarantuje thread safety przez podział danych
    chunk.iter_mut().map(|x| *x * 2.0).sum()
}
\end{lstlisting}
\begin{itemize}
    \item Harmonogram z kradzieżą pracy \eng{work-stealing scheduler} – automatyczne równoważenie obciążenia pomiędzy wątkami poprzez dynamiczne przydzielanie zadań.
    
    \item Równoległość danych – naturalne ukierunkowanie na przetwarzanie kolekcji danych w sposób równoległy.
    
    \item Bezpieczeństwo w czasie kompilacji – brak warunków wyścigu  gwarantowany przez mechanizm pożyczania.
    
    \item Ergonomiczne API – intuicyjne przekształcenie kodu sekwencyjnego na równoległy bez znacznego zwiększania złożoności.
    
    \item Jednolita abstrakcja: ten sam wzorzec we wszystkich benchmarkach NPB
  \end{itemize}
  

  \subsection{Specyfika benchmarku EP}
  \begin{lstlisting}[language=Rust, caption={Implementacja benchmarku EP w języku Rust}, label={lst:ep_rust}]
const CLASS: &str = "S";
const M: u32 = 24;
const MK: u32 = 16;
const MM: u32 = M - MK;
const NN: u32 = 1 << MM;
const NK: usize = 1 << MK;
const NQ: u32 = 10;
const A: f64 = 1220703125.0;
const S: f64 = 271828183.0;

fn main() {
    let args: Vec<String> = env::args().collect();
    let num_threads: usize = args.get(1)
        .map(|s| s.parse::<usize>().unwrap_or(1))
        .unwrap_or(1);

    rayon::ThreadPoolBuilder::new()
        .num_threads(num_threads)
        .build_global()
        .unwrap();

    // Globalne dane - odpowiednik static arrays w C++
    let mut x: Vec<f64> = Vec::with_capacity(NK_PLUS);
    let q: [f64; NQ as usize] = [0.0; NQ as usize];

    // Thread-local storage dla dużych tablic roboczych
    thread_local! {
        static THREAD_X: RefCell<Vec<f64>> = RefCell::new(vec![0.0; NK_PLUS]);
    }

    // Główna pętla równoległa - odpowiednik #pragma omp parallel for reduction
    let result = (1..NN+1)
        .collect::<Vec<_>>()
        .par_chunks(chunk_size)
        .fold(|| (0.0, 0.0), |mut acc, chunk| {
            for &k in chunk {
                THREAD_X.with(|x_cell| {
                    let mut x = x_cell.borrow_mut();
                    randdp::vranlc((2 * NK) as i32, &mut t1, A, &mut x);
                    
                    // Box-Muller transform w chunks dla lepszej wektoryzacji
                    for chunk_start in (0..NK).step_by(CHUNK_SIZE) {
                        let chunk_end = (chunk_start + CHUNK_SIZE).min(NK);
                        
                        for i in chunk_start..chunk_end {
                            let x1 = 2.0 * x[2 * i] - 1.0;
                            let x2 = 2.0 * x[2 * i + 1] - 1.0;
                            let t1 = x1 * x1 + x2 * x2;
                            
                            if t1 <= 1.0 {
                                let t2 = (-2.0 * t1.ln() / t1).sqrt();
                                let t3 = x1 * t2;
                                let t4 = x2 * t2;
                                let l = t3.abs().max(t4.abs()) as usize;
                                
                                if l < NQ as usize {
                                    local_counts[l] += 1;
                                    acc.0 += t3;
                                    acc.1 += t4;
                                }
                            }
                        }
                    }
                });
            }
            acc
        })
        .reduce(|| (0.0, 0.0), |mut acc1, acc2| {
            acc1.0 += acc2.0;
            acc1.1 += acc2.1;
            acc1
        });

    // Agregacja wyników
    sx = result.0;
    sy = result.1;
}
\end{lstlisting}

\subsection{Specyfika benchmarku CG}
\begin{lstlisting}[language=Rust, caption={Implementacja benchmarku CG w języku Rust}, label={lst:cg_rust}]
const CLASS: &str = "S";
const NA: i32 = 1400;
const NONZER: i32 = 7;
const NITER: i32 = 15;
const SHIFT: f64 = 10.0;
const NZ: i32 = NA * (NONZER + 1) * (NONZER + 1);

fn main() {
    let num_threads = if args.len() > 1 {
        args[1].parse::<usize>().unwrap_or(1)
    } else { 1 };

    rayon::ThreadPoolBuilder::new()
        .num_threads(num_threads)
        .build_global()
        .unwrap();

    // Globalne struktury danych - odpowiednik static arrays w C++
    let mut colidx: Vec<i32> = vec![0; NZ as usize];
    let mut rowstr: Vec<i32> = vec![0; (NA + 1) as usize];
    let mut a: Vec<f64> = vec![0.0; NZ as usize];
    let mut x: Vec<f64> = vec![1.0; (NA + 2) as usize];
    let mut z: Vec<f64> = vec![0.0; (NA + 2) as usize];
    let mut p: Vec<f64> = vec![0.0; (NA + 2) as usize];
    let mut q: Vec<f64> = vec![0.0; (NA + 2) as usize];
    let mut r: Vec<f64> = vec![0.0; (NA + 2) as usize];

    // Generowanie macierzy rzadkiej - standalone function
    makea(&mut naa, &mut nzz, &mut a, &mut colidx, &mut rowstr, 
            &firstrow, &lastrow, &firstcol, &lastcol, 
            &mut arow, &mut acol, &mut aelt, &mut iv, &mut tran, &amult);

    // Główna pętla iteracyjna
    for it in 1..=NITER {
        conj_grad(&mut colidx, &mut rowstr, &mut x, &mut z, &mut a, 
                    &mut p, &mut q, &mut r, &mut rnorm, &naa, &lastcol, 
                    &firstcol, &lastrow, &firstrow);
        
        // Równoległe dot products - odpowiednik reduction w OpenMP
        let len = (lastcol - firstcol + 1) as usize;
        
        norm_temp1 = x[..len]
            .par_iter()
            .zip(&z[..len])
            .map(|(&xi, &zi)| xi * zi)
            .sum();

        norm_temp2 = z[..len]
            .par_iter()
            .map(|&zi| zi * zi)
            .sum();

        // Równoległa aktualizacja wektora x - odpowiednik #pragma omp for
        x[..=(lastcol - firstcol) as usize].par_iter_mut()
            .zip(&z[..=(lastcol - firstcol) as usize])
            .for_each(|(xi, &zi)| {
                *xi = norm_temp2 * zi;
            });
    }
}

// Standalone function - odpowiednik funkcji C
fn conj_grad(colidx: &mut Vec<i32>, rowstr: &mut Vec<i32>, 
                x: &mut Vec<f64>, z: &mut Vec<f64>, a: &mut Vec<f64>, 
                p: &mut Vec<f64>, q: &mut Vec<f64>, r: &mut Vec<f64>, 
                rnorm: &mut f64, /* inne parametry */) {
    let cgitmax: i32 = 25;
    
    // Sekwencyjna inicjalizacja - jak w C++
    for j in 0..=*naa {
        let j = j as usize;
        q[j] = 0.0;
        z[j] = 0.0;
        r[j] = x[j];
        p[j] = r[j];
    }

    // Równoległe obliczenie rho - odpowiednik #pragma omp for reduction(+:rho)
    rho = (0..=(*lastcol - *firstcol))
        .into_par_iter()
        .map(|j| {
            let idx = j as usize;
            r[idx] * r[idx]
        })
        .sum();

    for _cgit in 1..=cgitmax {
        // Równoległe mnożenie macierz-wektor - odpowiednik #pragma omp for
        q.par_chunks_mut(1)
            .enumerate()
            .for_each(|(j, q_slice)| {
                if j <= (*lastrow - *firstrow) as usize {
                    let mut sum = 0.0;
                    for k in rowstr[j]..rowstr[j + 1] {
                        let k = k as usize;
                        let cidx = colidx[k];
                        if cidx >= 0 && (cidx as usize) < p.len() {
                            sum += a[k] * p[cidx as usize];
                        }
                    }
                    q_slice[0] = sum;
                }
            });

        // Równoległe obliczenie dot product
        d = (0..=(*lastcol - *firstcol))
            .into_par_iter()
            .map(|j| {
                let j = j as usize;
                p[j] * q[j]
            })
            .sum();

        // Równoległe aktualizacje wektorów
        let range = 0..=(*lastcol - *firstcol) as usize;
        z[range.clone()].par_iter_mut()
            .zip(&p[range.clone()])
            .for_each(|(z_val, &p_val)| {
                *z_val = *z_val + alpha * p_val;
            });

        r[range.clone()].par_iter_mut()
            .zip(&q[range.clone()])
            .for_each(|(r_val, &q_val)| {
                *r_val = *r_val - alpha * q_val;
            });

        // Nowe rho z równoległą redukcją
        rho = r[range.clone()].par_iter()
            .map(|&r_val| r_val * r_val)
            .sum();
    }
}
\end{lstlisting}
  
\subsection{Specyfika benchmarku IS}
\begin{lstlisting}[language=Rust, caption={Implementacja benchmarku IS w języku Rust}, label={lst:is_rust}]
const CLASS: &str = "S";
const TOTAL_KEYS_LOG_2: u32 = 16;
const MAX_KEY_LOG_2: u32 = 11;
const NUM_BUCKETS_LOG_2: u32 = 9;
const TOTAL_KEYS: usize = 1 << TOTAL_KEYS_LOG_2;
const MAX_KEY: usize = 1 << MAX_KEY_LOG_2;
const NUM_BUCKETS: usize = 1 << NUM_BUCKETS_LOG_2;

// Struktura enkapsulująca złożony stan bucket sort - wyjątek od proceduralnej reguły
struct ISBenchmark {
    key_array: Vec<KeyType>,
    key_buff1: Vec<KeyType>,
    key_buff2: Vec<KeyType>,
    bucket_size: Vec<Vec<KeyType>>,
    bucket_ptrs: Vec<KeyType>,
    num_threads: usize,
}

impl ISBenchmark {
    fn new(num_threads: usize) -> Self {
        ISBenchmark {
            key_array: vec![0; TOTAL_KEYS],
            key_buff1: vec![0; MAX_KEY],
            key_buff2: vec![0; TOTAL_KEYS],
            bucket_size: vec![vec![0; NUM_BUCKETS]; num_threads],
            bucket_ptrs: vec![0; NUM_BUCKETS + 1],
            num_threads,
        }
    }
    
    fn rank(&mut self, iteration: i32) {
        let shift = MAX_KEY_LOG_2 - NUM_BUCKETS_LOG_2;
        
        // Równoległe bucket counting - odpowiednik #pragma omp parallel for
        let chunk_size = (TOTAL_KEYS + self.num_threads - 1) / self.num_threads;
        
        self.key_array
            .par_chunks(chunk_size)
            .zip(&mut self.bucket_size)
            .for_each(|(chunk, thread_bucket)| {
                for &key in chunk {
                    let bucket_idx = (key >> shift) as usize;
                    if bucket_idx < NUM_BUCKETS {
                        thread_bucket[bucket_idx] += 1;
                    }
                }
            });

        // Równoległe bucket sort z atomics
        (0..NUM_BUCKETS).into_par_iter().for_each(|i| {
            let key_buff1_shared = Arc::new(Mutex::new(&mut self.key_buff1));
            // Complex parallel bucket redistribution
        });
    }
}

fn main() {
    let num_threads = if args.len() > 1 {
        args[1].parse::<usize>().unwrap_or(1)
    } else { 1 };
    
    rayon::ThreadPoolBuilder::new()
        .num_threads(num_threads)
        .build_global()
        .unwrap();

    let mut benchmark = ISBenchmark::new(num_threads);
    
    // Proceduralne wywołania na strukturze
    benchmark.create_seq();
    
    for iteration in 1..=MAX_ITERATIONS {
        benchmark.rank(iteration);
    }
    
    benchmark.full_verify();
}
\end{lstlisting}
\begin{itemize}
    \item Losowy dostęp do pamięci – nieprzewidywalny wzorzec odwołań do pamięci, który utrudnia efektywne wykorzystanie pamięci podręcznej procesora.
    
    \item Nierównomierne obciążenie – niejednorodne rozłożenie kluczy pomiędzy kubełki (buckets), prowadzące do nieefektywnego wykorzystania zasobów obliczeniowych.
    
    \item Narzut synchronizacji – konieczność stosowania operacji atomowych podczas równoczesnej aktualizacji kubełków, co może obniżać wydajność.
  \end{itemize}
  