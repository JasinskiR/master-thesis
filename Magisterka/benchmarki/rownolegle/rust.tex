\section{Implementacje w języku Rust}
\subsection{Struktura i organizacja kodu}
Implementacje w języku Rust charakteryzują się proceduralną strukturą podobną do implementacji C++, wykorzystującą globalne stałe i funkcje standalone. Wszystkie benchmarki (EP, CG, IS) następują jednolity wzorzec architektoniczny oparty na funkcji \texttt{main()}:

\begin{lstlisting}[language=Rust, caption={Struktura kodu benchmarków w języku Rust}, label={lst:rust_structure}]
pub struct NPBBenchmark {
// Rust: stałe kompilacyjne + proceduralna struktura
const CLASS: &str = "S";
const TOTAL_KEYS: usize = 1 << 16;

fn main() {
    let num_threads = parse_args();
    
    // Rust: declarative thread pool configuration
    rayon::ThreadPoolBuilder::new()
        .num_threads(num_threads)
        .build_global().unwrap();
    
    algorithm_implementation();
}

// Funkcje niezależne operujące na globalnych danych
fn algorithm_implementation() { /* ... */ }
fn verification_and_results() { /* ... */ }
\end{lstlisting}
Kluczowe cechy organizacji kodu Rust:
\begin{itemize}
\item Struktura proceduralna - główna logika programu umieszczana jest w funkcji \texttt{main()}, analogicznie do języka C++.
\item Globalne stałe - parametry problemu definiowane są jako stałe kompilacyjne (\texttt{const}).
\item Funkcje niezależne - logika algorytmu implementowana jest w postaci oddzielnych funkcji, a nie metod związanych z obiektami.
\item Automatyczne zarządzanie pamięcią - dynamiczne struktury danych (np. \texttt{Vec<T>}) zastępują ręczne zarządzanie pamięcią (np. \texttt{malloc}/\texttt{free}).
\item Integracja z biblioteką Rayon - jednolite zastosowanie biblioteki Rayon we wszystkich benchmarkach w celu realizacji współbieżności.
\end{itemize}


\subsection{Zarządzanie pamięcią}
Rust wykorzystuje system własności z automatycznym zarządzaniem pamięcią przez \texttt{Vec<T>}, eliminując ręczną alokację:
\begin{lstlisting}[language=Rust, caption={Zarządzanie pamięcią w benchmarkach NPB w języku Rust}, label={lst:rust_memory_management}]
impl NPBBenchmark {
// Rust: automatyczne zarządzanie pamięci przez Vec<T>
let mut key_array: Vec<i32> = vec![0; TOTAL_KEYS];
let mut working_data: Vec<f64> = vec![0.0; PROBLEM_SIZE];

// Rust: thread-safe storage bez jawnego blokowania
thread_local! {
    static THREAD_DATA: RefCell<Vec<f64>> = RefCell::new(vec![0.0; SIZE]);
}

fn algorithm_core(data: &mut Vec<i32>, working: &mut Vec<f64>) {
    // Pożyczanie pozwala na bezpieczny dostęp bez przenoszenia własności
    parallel_processing(data, working);
    // Kompilator gwarantuje brak wyścigów danych i użyciu po zwolnieniu
}
\end{lstlisting}
Zalety modelu własności:
\begin{itemize}
    \item Automatyczna dealokacja: \texttt{Vec<T>} zwalnia pamięć automatycznie po wyjściu z zakresu,
    \item Brak wycieków pamięci: gwarancja na poziomie kompilatora,
    \item Abstrakcje bez narzutu kosztów: brak narzutu wydajnościowego,
    \item Bezpieczeństwo wątków: mechanizm sprawdzania pożyczania eliminuje wyścigi danych.
\end{itemize}

\subsection{Mechanizmy równoległości}
Wszystkie implementacje Rust wykorzystują bibliotekę Rayon dla spójnego podejścia do równoległości:
\begin{lstlisting}[language=Rust, caption={Równoległość w benchmarkach NPB w języku Rust}, label={lst:rust_parallelism}]
use rayon::prelude::*;

fn parallel_computation(data: &mut Vec<f64>, num_threads: usize) {
    // Konfiguracja thread pool - odpowiednik task_scheduler_init w TBB
    rayon::ThreadPoolBuilder::new()
        .num_threads(num_threads)
        .build_global()
        .unwrap();

    // Równoległe przetwarzanie chunks - odpowiednik #pragma omp parallel for
    let results: Vec<_> = data
        .par_chunks_mut(optimal_chunk_size())
        .map(|chunk| process_chunk(chunk))
        .collect();
        
    // Równoległa redukcja - odpowiednik reduction(+:sum) w OpenMP
    let final_result = data.par_iter()
        .fold(|| 0.0, |acc, &x| acc + compute_element(x))
        .reduce(|| 0.0, |acc1, acc2| acc1 + acc2);
}

fn process_chunk(chunk: &mut [f64]) -> f64 {
    // Bezpieczne przetwarzanie bez mutexów dzięki par_chunks_mut
    // Rayon gwarantuje thread safety przez podział danych
    chunk.iter_mut().map(|x| *x * 2.0).sum()
}
\end{lstlisting}
\begin{itemize}
    \item Harmonogram z kradzieżą pracy \eng{work-stealing scheduler} - automatyczne równoważenie obciążenia pomiędzy wątkami poprzez dynamiczne przydzielanie zadań.
    
    \item Równoległość danych - naturalne ukierunkowanie na przetwarzanie kolekcji danych w sposób równoległy.
    
    \item Bezpieczeństwo w czasie kompilacji - brak warunków wyścigu  gwarantowany przez mechanizm pożyczania.
    
    \item Ergonomiczne API - intuicyjne przekształcenie kodu sekwencyjnego na równoległy bez znacznego zwiększania złożoności.
    
    \item Jednolita abstrakcja: ten sam wzorzec we wszystkich benchmarkach NPB
  \end{itemize}
  

\subsection{Specyfika benchmarku EP}
\begin{lstlisting}[language=Rust, caption={Implementacja benchmarku EP w języku Rust}, label={lst:ep_rust}]
const NN: u32 = 1 << 8;
const NK: usize = 1 << 16;

fn ep_main() {
    // Rust: deklaratywna iteracja równoległa + redukcja
    let (sx, sy) = (1..NN+1)
        .into_par_iter()
        .map(|_| {
            // Box-Muller transform
            let (x1, x2) = generate_random_pair();
            let t1 = x1 * x1 + x2 * x2;
            
            if t1 <= 1.0 {
                let t2 = (-2.0 * t1.ln() / t1).sqrt();
                (x1 * t2, x2 * t2)
            } else { (0.0, 0.0) }
        })
        .reduce(|| (0.0, 0.0), |a, b| (a.0 + b.0, a.1 + b.1));
}
\end{lstlisting}

\subsection{Specyfika benchmarku CG}
\begin{lstlisting}[language=Rust, caption={Implementacja benchmarku CG w języku Rust}, label={lst:cg_rust}]
fn cg_iteration(a: &[f64], p: &mut [f64], q: &mut [f64]) {
    // Rust: parallel matrix-vector multiply
    q.par_chunks_mut(CHUNK_SIZE)
        .enumerate()
        .for_each(|(i, q_chunk)| {
            q_chunk[0] = sparse_matvec(&a, &p, i);
        });
    
    // Rust: parallel dot product reduction
    let dot_result = p.par_iter()
        .zip(q.par_iter())
        .map(|(&p_val, &q_val)| p_val * q_val)
        .sum::<f64>();
}
\end{lstlisting}
  
\subsection{Specyfika benchmarku IS}
\begin{lstlisting}[language=Rust, caption={Implementacja benchmarku IS w języku Rust}, label={lst:is_rust}]
const TOTAL_KEYS: usize = 1 << 16;
const NUM_BUCKETS: usize = 1 << 9;

// Rust: encapsulated state vs C++ global variables
struct ISBenchmark {
    keys: Vec<u32>,
    buckets: Vec<Vec<u32>>,
}

impl ISBenchmark {
    fn parallel_bucket_sort(&mut self) {
        // Rust: safe parallel processing with ownership
        self.keys.par_sort_by_key(|&key| bucket_index(key));
    }
}
}
\end{lstlisting}
\begin{itemize}
    \item Mieszana architektura: \textbf{EP} i \textbf{CG} są proceduralne, natomiast \textbf{IS} posiada strukturę obiektową.
    \item Enkapsulacja stanu: Złożone struktury danych są definiowane jako \texttt{struct} z odpowiadającą im implementacją w \texttt{impl}.
    \item Zarządzanie własnością: Użycie \texttt{\&mut self} pozwala na bezpieczną modyfikację stanu bez naruszenia reguł współbieżności.
    \item Bezpieczeństwo wątków: Gwarantowane automatycznie przez system typów i mechanizm pożyczania.
\end{itemize}

  