\pdfbookmark[0]{Streszczenie}{streszczenie.1}

%\mbox{}\vspace{2cm} % mo¿na przesun¹æ, w zale¿noœci od d³ugoœci streszczenia
\begin{abstract}
Praca skupia się na porównaniu mechanizmów programowania współbieżnego i równoległego w językach Rust i C++. Pierwsza część obejmuje teoretyczne podstawy programowania równoległego oraz przegląd literatury przedmiotu, identyfikując luki badawcze w porównaniach tych języków na różnych architekturach sprzętowych.

Kolejne rozdziały przedstawiają mechanizmy współbieżności w analizowanych językach, omawiając biblioteki Rust Rayon i Tokio oraz C++ OpenMP i Intel TBB. Metodologia badań obejmuje opis środowiska testowego na architekturach ARM64 (Apple Silicon M1) i x86\_64 (Intel) z różnymi kompilatorami i systemami operacyjnymi.

Implementacja zawiera benchmarki NAS Parallel Benchmarks (CG, EP, IS) dla programowania równoległego oraz aplikacje testowe (producent-konsument, echo-serwer) dla programowania współbieżnego w obu językach. Analiza wyników obejmuje pomiary wydajności w MFLOPS, skalowanie względem liczby wątków, zużycie zasobów systemowych oraz efektywność komunikacji sieciowej.

Badania ujawniły systematyczną przewagę architektury ARM64 z współczynnikami przyspieszenia 1,55x-2,26x. Rust z Rayon osiągnął najwyższą wydajność w~obliczeniach równoległych, Intel TBB dominowała w zadaniach komunikacyjnych, a model bezpieczeństwa pamięci Rust nie wprowadzał narzutów wydajnościowych. Głównym ograniczeniem była heterogeniczność środowisk testowych.

Praca dostarcza pierwszego kompleksowego porównania tych języków na dwóch architekturach oraz praktycznych rekomendacji wyboru technologii w zależności od charakterystyki problemu obliczeniowego.

\end{abstract}
% \mykeywords{}

{
\selectlanguage{english}
\begin{abstract}
The thesis focuses on comparing concurrent and parallel programming mechanisms in Rust and C++ languages. The first part covers theoretical foundations of~parallel programming and literature review, identifying research gaps in comparisons of~these languages across different hardware architectures.

Subsequent chapters present concurrency mechanisms in analyzed languages, discussing Rust Rayon and Tokio libraries, and C++ OpenMP and Intel TBB. Research methodology encompasses testing environment description on ARM64 (Apple Silicon M1) and x86\_64 (Intel) architectures with different compilers and operating systems.

Implementation includes NAS Parallel Benchmarks (CG, EP, IS) for parallel programming and test applications (producer-consumer, echo-server) for concurrent programming in both languages. Results analysis covers MFLOPS performance measurements, thread scaling, system resource consumption, and network communication efficiency.

Research revealed systematic ARM64 superiority with speedup factors of~\mbox{1,55x-2,26x}. Rust with Rayon achieved highest performance in parallel computations, Intel TBB dominated communication-intensive tasks, and Rust's memory safety model introduced no performance overhead. The main limitation was testing environment heterogeneity.

The work provides the first comprehensive comparison of these languages across two architectures and practical recommendations for technology selection based on~computational problem characteristics.
\end{abstract}
% \mykeywords{}
}