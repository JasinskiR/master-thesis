\pdfbookmark[0]{Streszczenie}{streszczenie.1}

%\mbox{}\vspace{2cm} % mo¿na przesun¹æ, w zale¿noci od d³ugoci streszczenia
\begin{abstract}
Praca skupia się na porównaniu mechanizmów programowania współbieżnego i równoległego w językach Rust i C++. Pierwsza część obejmuje teoretyczne podstawy oraz przegląd literatury, identyfikując luki badawcze w porównaniach tych języków na różnych architekturach sprzętowych.

Kolejne rozdziały przedstawiają mechanizmy współbieżności, omawiając biblioteki Rust Rayon i Tokio oraz C++ OpenMP i Intel TBB. Metodologia badań obejmuje środowisko testowe na architekturach ARM64 (Apple Silicon M1) i x86\_64 (Intel) z różnymi kompilatorami i systemami operacyjnymi.

Implementacja zawiera benchmarki NAS Parallel Benchmarks (CG, EP, IS) oraz aplikacje testowe (producent-konsument, echo-serwer) w obu językach. Analiza wyników obejmuje pomiary wydajności w MFLOPS, skalowanie względem liczby wątków oraz zużycie zasobów systemowych.

Badania ujawniły systematyczną przewagę architektury ARM64 z współczynnikami przyspieszenia 1,55x-2,26x. Rust z Rayon osiągnął najwyższą wydajność w~obliczeniach równoległych, Intel TBB dominowała w zadaniach komunikacyjnych.

Praca dostarcza pierwszego kompleksowego porównania tych języków na dwóch architekturach oraz praktycznych rekomendacji wyboru technologii.

\end{abstract}
\mykeywords{}

{
\selectlanguage{english}
\begin{abstract}
The thesis focuses on comparing concurrent and parallel programming mechanisms in Rust and C++ languages. The first part covers theoretical foundations and literature review, identifying research gaps in comparisons across different hardware architectures.

Subsequent chapters present concurrency mechanisms, discussing Rust Rayon and Tokio libraries, and C++ OpenMP and Intel TBB. Research methodology encompasses testing environment on ARM64 (Apple Silicon M1) and x86\_64 (Intel) architectures with different compilers and operating systems.

Implementation includes NAS Parallel Benchmarks (CG, EP, IS) and test applications (producer-consumer, echo-server) in both languages. Results analysis covers MFLOPS performance measurements, thread scaling, and system resource consumption.

Research revealed systematic ARM64 superiority with speedup factors of~\mbox{1,55x-2,26x}. Rust with Rayon achieved highest performance in parallel computations, Intel TBB dominated communication-intensive tasks.

The work provides the first comprehensive comparison of these languages across two architectures and practical technology selection recommendations.

\end{abstract}
\mykeywords{}
}