%Porównanie Rust i C++
% \chapter{Porównanie mechanizmów w językach Rust oraz C++}
% \section{Programowanie współbieżne}
% Thread Creation and Management
% Thread creation time (microseconds)
% Memory overhead per thread (KB)
% Maximum number of threads before system failure
% Synchronization Performance
% Mutex lock/unlock operations per second
% Channel/message passing latency (microseconds)
% Context switching time between threads (microseconds)
% Safety Overhead
% Compilation time with concurrent code (seconds)
% Binary size comparison (KB)
% Memory usage during runtime (MB)

% \section{Programowanie równoległe}

% Computation Performance
% Execution time for parallel algorithms (milliseconds)
% Speedup ratio (T1/Tn) with different thread counts
% Efficiency (Speedup/Number of processors)
% GFLOP/s performance
% - Single-threaded GFLOP/s
% - Multi-threaded GFLOP/s
% - Scaling efficiency (GFLOP/s per core)
% Amdahl's law comparison
% Resource Usage
% CPU utilization percentage
% Memory consumption under load
% Cache miss rates
% I/O operations per second

% GFLOP/s is particularly useful:\\

% Provides a standardized way to measure computational performance
% Allows direct comparison of algorithmic efficiency
% Helps evaluate hardware utilization effectiveness
% To measure GFLOP/s:\\

% Matrix multiplication
% FFT (Fast Fourier Transform)
% N-body simulation

% \chapter{Porównanie mechanizmów w językach Rust oraz C++}
\chapter{Założenia i metodologia porównania mechanizmów w językach Rust oraz C++}
W tym rozdziale przedstawiono metryki oraz algorytmy do analizy wydajności mechanizmów programowania współbieżnego i równoległego w językach Rust oraz C++. 
\section{Programowanie współbieżne}
Dla mechanizmów programowania współbieżnego autor nie znalazł obecnie zunifikowanego, powszechnie uznanego zestawu benchmarków odpowiadający randze NPB. W związku z tym, w ramach niniejszej pracy, opracowano własny zestaw mini-aplikacji testowych, zaprojektowanych w taki sposób, aby odzwierciedlały typowe scenariusze współbieżności: komunikację między wątkami, synchronizację, obsługę asynchronicznych operacji wejścia/wyjścia, sytuacje ryzyka zakleszczenia, a także przypadki intensywnego przetwarzania danych z wykorzystaniem wielu wątków.

\subsection{Zarządzanie wątkami}
Porównanie tworzenia i zarządzania wątkami obejmuje następujące parametry:
\begin{itemize}
\item czas tworzenia wątku (w mikrosekundach),
\item narzut pamięciowy na wątek (w KB)
\end{itemize}

\subsection{Wydajność synchronizacji}
Kluczowe aspekty synchronizacji to:
\begin{itemize}
\item liczba operacji blokowania i odblokowywania mutexa na sekundę,
\item opóźnienie przesyłania wiadomości przez kanały (w mikrosekundach),
\item czas przełączania kontekstu pomiędzy wątkami (w mikrosekundach).
\end{itemize}

\subsection{Narzut bezpieczeństwa}
W kontekście narzutu związanego z mechanizmami bezpieczeństwa w językach Rust i C++ analizowane będą następujące metryki:
\begin{itemize}
\item czas kompilacji kodu współbieżnego (w sekundach),
\item rozmiar pliku binarnego (w KB),
\item zużycie pamięci podczas wykonywania programu (w MB).
\end{itemize}

\subsection{Wybrane algorytmy do analizy}
Dla porównania mechanizmów współbieżności wybrano następujące algorytmy:
\begin{itemize}
    \item Model producent-konsument (z wykorzystaniem aktorów),
    \item Problem filozofów (synchronizacja dostępu do zasobów).
\end{itemize}
Algorytmy te pozwalają na analizę zdolności języków Rust i C++ do efektywnego zarządzania współbieżnością w obliczeniach numerycznych.


\section{Programowanie równoległe}
W przypadku programowania równoległego, zdecydowano się na wykorzystanie uznanego zestawu testowego NAS Parallel Benchmarks (NPB) \cite{nasaParallelBenchmarks}. Zestaw ten jest szeroko stosowany w środowisku naukowym do oceny wydajności systemów wysokowydajnych (HPC) i stanowi wiarygodny punkt odniesienia przy analizie efektywności obliczeniowej.
\subsection{Wydajność obliczeniowa}
Główne metryki oceny wydajności algorytmów równoległych to:
\begin{itemize}
\item czas wykonania algorytmów równoległych (w milisekundach),
\item współczynnik przyspieszenia (T1/Tn) dla różnych liczby wątków,
\item efektywność (przyspieszenie/liczba procesorów).
\end{itemize}

\subsection{Wydajność sprzętowa (GFLOP/s)}
Wydajność obliczeniowa mierzona w jednostkach GFLOP/s (gigaflops per second) pozwala na ocenę efektywności wykorzystania sprzętu:
\begin{itemize}
\item wydajność w pojedynczym wątku,
\item wydajność wielowątkowa,
\item efektywność skalowania (GFLOP/s na rdzeń).
\end{itemize}
Dodatkowo przeprowadzona zostanie analiza zgodnie z prawem Amdahla w celu określenia teoretycznych ograniczeń przyspieszenia obliczeń.

\subsection{Zasoby systemowe}
Analiza zużycia zasobów przez algorytmy równoległe obejmuje:
\begin{itemize}
\item procentowe wykorzystanie CPU,
\item zużycie pamięci w warunkach obciążenia,
\item współczynnik nietrafień w cache,
\item liczbę operacji wejścia-wyjścia na sekundę.
\end{itemize}

\subsection{Wybrane algorytmy do analizy}
Dla porównania mechanizmów równoległości wybrano następujące algorytmy z zestawu testowego NPB:
\begin{itemize}
    \item CG - \eng{conjugate gradient} - gradient sprzężony
    \item EP - \eng{embarrassingly parallel} - problem trywialnie równoległy
    \item IS - \eng{intiger sorting} - sortowanie liczb całkowitych
\end{itemize}

Wybór powyższych benchmarków pozwoli na szczegółową analizę wydajności oraz stabilności obu języków w kontekście programowania współbieżnego i równoległego, umożliwiając sformułowanie rekomendacji dotyczących wyboru narzędzi w zależności od specyfiki projektu.

\subsubsection{CG - Gradient sprzężony}
Benchmark CG \eng{conjugate gradient} służy do oceny wydajności systemów wysokowydajnych (HPC) w kontekście rozwiązywania rzadkich układów równań liniowych metodą iteracyjną. Algorytm ten znajduje zastosowanie w wielu dziedzinach nauk obliczeniowych, takich jak mechanika płynów czy analiza strukturalna, gdzie układy równań wynikają z dyskretyzacji równań różniczkowych cząstkowych. W benchmarku CG generowana jest syntetyczna macierz rzadkich współczynników o dużych rozmiarach, a następnie przeprowadzana jest iteracyjna procedura wyznaczania przybliżonego rozwiązania układu równań. Test ten charakteryzuje się intensywnym wykorzystaniem operacji wektorowych i punktowych operacji na danych rozproszonych, co czyni go szczególnie użytecznym przy ocenie efektywności komunikacji między wątkami oraz przepustowości pamięci w systemach równoległych \cite{nasaParallelBenchmarks}.

\subsubsection{EP - Problem trywialnie równoległy}
Benchmark EP \eng{embarrassingly parallel} został zaprojektowany w celu oceny wydajności systemów obliczeniowych w scenariuszach, w których niemal całkowicie eliminuje się konieczność komunikacji między procesami lub wątkami. Test ten polega na generowaniu dużej liczby losowych punktów i przeprowadzaniu na nich niezależnych obliczeń statystycznych, takich jak estymacja wartości $\pi$ lub momentów rozkładu. Dzięki swojej naturze, EP umożliwia niemal idealną skalowalność równoległą i jest wykorzystywany przede wszystkim do pomiaru czystej mocy obliczeniowej procesorów, efektywności rozdziału zadań oraz narzutu wynikającego z zarządzania wątkami. Ze względu na minimalne wymagania względem synchronizacji i komunikacji, benchmark ten stanowi punkt odniesienia przy analizie teoretycznego maksimum wydajności danego systemu dla obciążeń równoległych \cite{nasaParallelBenchmarks}.

\subsubsection{IS - Sortowanie liczb całkowitych}
Benchmark IS \eng{integer sorting} służy do oceny wydajności systemów obliczeniowych w zakresie operacji nieciągłych i trudnych do zrównoleglenia, takich jak sortowanie i przemieszczanie danych w pamięci. Test polega na wygenerowaniu losowego zestawu liczb całkowitych, a następnie ich posortowaniu przy użyciu metody sortowania kubełkowego \eng{bucket sort} z zastosowaniem rozproszonej synchronizacji i komunikacji między wątkami lub procesami. Benchmark IS jest szczególnie przydatny do analizy przepustowości podsystemów pamięciowych, efektywności komunikacji w architekturach wieloprocesorowych oraz odporności systemu na nierównomierne rozłożenie danych. Ze względu na swoją nieregularną strukturę dostępu do danych i znaczną liczbę operacji porządkowania, IS stanowi istotne uzupełnienie pozostałych testów NPB, koncentrując się na problemach wymagających intensywnej pracy z pamięcią i synchronizacją \cite{nasaParallelBenchmarks}.

\section{Metodologia badań}
Badania eksperymentalne zostały zaprojektowane w taki sposób, aby umożliwić porównanie mechanizmów współbieżnych i równoległych w językach Rust oraz C++ przy wykorzystaniu dwóch odmiennych architektur sprzętowych: x86\_64 (architektura tradycyjna, Windows/Linux) oraz ARM64 (architektura Apple Silicon - M1, macOS). Dzięki temu możliwa będzie analiza wpływu typu procesora i systemu operacyjnego na wydajność oraz efektywność implementacji.

\subsection{Środowisko testowe}
W ramach środowiska testowego zostały wykorzystane następujące urządzenia wraz z oprogoramowaniem:
\subsubsection{Architektura ARM}
W ramach architektury ARM został wykorzystany laptop firmy Apple - MacBook Pro z następującymi specyfikacjami:
\begin{itemize}
    \item procesor Apple M1
    \item pamięć RAM 16 GB
    \item system - macOS Sonoma wersja 14.5 (23F79)
\end{itemize}
\subsubsection{Architektura x86\_64}
W ramach architektury x86\_64 został wykorzystany laptop firmy HP z następującymi specyfikacjami:
\begin{itemize}
    \item procesor
    \item pamięć RAM 32 GB
    \item system - Linux Ubuntu 24.04 LST
\end{itemize}
\subsection{Procedura testowa}
Procedura testowa będzie obejmowała uruchomienie zestawu benchmarków w różnych konfiguracjach, takich jak liczba wątków oraz różne klasy algorytmów w przypadku NPB oraz na dwóch architekturach procesora. Każdy test będzie uruchamiany wielokrotnie - 10 razy, aby uzyskać uśrednione wyniki. Algorytmy zawierają również swoje logi zdzarzeń, które zostaną użyte do weryfikacji poprawności działania samego algorytmu jak i do późniejszej analizy wyników.
\subsection{Narzędzia pomiarowe}
W ramach przeprowadzania testów zostaną wykorzystane następujące narzędzia pomiarowe:
\begin{itemize}
    \item perf - narzędzie do profilowania kodu, które pozwala na analizę wydajności aplikacji w czasie rzeczywistym (jednakże nie jest dostępne na systemach Apple)
    \item instruments (Xcode) - oficjalne narzędzie do profilowania od firmy Apple - zamiennik narzędzia \textit{perf} w tym przypadku,
    \item hwloc - narzędzie pozwalające zbadać zachowanie programu jeżeli chodzi o dostęp do podzespołów komputera
    \item threadsanitizer - flaga do kompilatora (dla języka C++ w tym przypadku), która pozwala sprawdzić czy w fazie kompilwoania nie zachodzi sytuacja wyścigów
\end{itemize}

