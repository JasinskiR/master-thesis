\section{Programowanie równoległe}
Rust oferuje nowoczesne podejście do programowania równoległego, które pozwala na bezpieczne i wydajne wykorzystanie wielu rdzeni procesora. Dzięki statycznemu systemowi typów, modelowi własności oraz bogatemu ekosystemowi bibliotek, programowanie równoległe w Rust jest zarówno ergonomiczne, jak i odporne na typowe błędy związane z współdzieleniem pamięci.
\subsection{Biblioteki}
Jednym z kluczowych komponentów wspierających programowanie równoległe w Rust jest biblioteka Rayon. Została ona zaprojektowana jako ergonomiczne narzędzie do równoległego przetwarzania kolekcji oraz rekurencyjnych algorytmów, takich jak mapowanie, filtrowanie czy redukcja. W przeciwieństwie do tradycyjnych podejść wymagających ręcznego tworzenia i zarządzania wątkami, Rayon oferuje wysokopoziomowe abstrakcje, które ukrywają złożoność alokacji wątków oraz synchronizacji, przy zachowaniu bezpieczeństwa typów i braku wyścigów danych.\\
Przykładowe wykorzystanie biblioteki Rayon może wyglądać następująco:
\begin{lstlisting}[language=Rust, caption=Przykład użycia par\_iter, label=pariter_example]
use rayon::prelude::*;

fn main() {
    let data = vec![1, 2, 3, 4, 5, 6, 7, 8];

    let result: i32 = data
        .par_iter()                  // <- równoległa wersja iteratora
        .map(|x| x * 2)              // <- równolegle podwajamy każdą wartość
        .reduce(|| 0, |a, b| a + b); // <- redukcja do sumy, start = 0

    println!("Wynik: {}", result); // Oczekiwany wynik: 72
}
    
\end{lstlisting}
Kod przedstawiony w listingu \ref{pariter_example} realizuje prostą operację podwajania wartości elementów wektora oraz ich sumowania, jednak kluczową cechą jest to, że wszystkie operacje wykonywane są równolegle. Funkcja \texttt{par\_iter()} zamienia standardowy iterator sekwencyjny na jego równoległy odpowiednik, co oznacza, że kolejne elementy będą przetwarzane na wielu rdzeniach procesora w sposób automatyczny i zoptymalizowany przez bibliotekę.

Następnie funkcja \texttt{map} pozwala na równoległe zastosowanie tej samej operacji - w tym przypadku mnożenia przez 2 - do każdego elementu kolekcji. Nie wymaga to żadnej ręcznej synchronizacji ani zarządzania współbieżnością — biblioteka zajmuje się podziałem pracy i wykonaniem w sposób optymalny względem dostępnych zasobów sprzętowych.

Ostatni etap przetwarzania to \texttt{reduce}, który agreguje wyniki cząstkowe w jeden wynik końcowy. Funkcja ta działa również równolegle: najpierw sumowane są wartości lokalnie (w ramach każdego wątku roboczego), a dopiero potem następuje łączenie tych częściowych sum w jedną finalną wartość. Neutralny element funkcji redukującej to 0, co jest standardowym przypadkiem przy sumowaniu liczb całkowitych.

Cały proces ilustruje typowy model MapReduce, w którym dane są:
\begin{itemize}
\item dzielone na fragmenty \eng{split},
\item transformowane \eng{map},
\item a następnie agregowane \eng{reduce}.
\end{itemize}