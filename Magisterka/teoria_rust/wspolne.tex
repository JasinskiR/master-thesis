\section{Mechanizmy wspólne dla współbieżności i równoległości}
\subsection{Wątki (std::thread)}
Jednym z podstawowych narzędzi oferowanych przez standardową bibliotekę Rust jest moduł std::thread, który umożliwia tworzenie niezależnych wątków wykonawczych. Pomimo że zapewnia on niski poziom abstrakcji i bezpośrednią kontrolę nad wątkami, jego użycie wymaga większej ostrożności w kontekście synchronizacji i zarządzania danymi współdzielonymi.
Przykładowa konstrukcja:
\begin{lstlisting}[language=Rust, caption=Przykład tworzenia wątku, label=thread_example]
use std::thread;

fn main() {
    let handle = thread::spawn(|| {
        // kod wykonywany równolegle
    });
    handle.join().unwrap();
}
\end{lstlisting}
W przedstawionym przykładzie listing \ref{thread_example} wykorzystano funkcję thread::spawn, która tworzy nowy wątek wykonawczy, umożliwiając równoległe przetwarzanie danych lub zadań. Ciało funkcji anonimowej przekazanej do spawn zawiera kod, który zostanie wykonany w kontekście nowo utworzonego wątku. Zmienna handle przechowuje uchwyt do tego wątku, umożliwiając synchronizację z jego wykonaniem.
Wywołanie handle.join().unwrap() służy do zablokowania głównego wątku programu do momentu zakończenia pracy wątku potomnego. Metoda join zwraca wynik zakończenia wątku.

Tworzenie dużej liczby wątków może być kosztowne zarówno pod względem zasobów systemowych, jak i czasu inicjalizacji. W związku z tym, Rust oferuje mechanizmy pul wątków \eng{thread pools}, które umożliwiają wielokrotne wykorzystywanie wcześniej zainicjalizowanych wątków do realizacji wielu zadań.

Popularnym rozwiązaniem wspierającym pule wątków jest biblioteka rayon, która automatyzuje proces zarządzania wątkami w kontekście równoległego przetwarzania danych. Jednakże, również inne biblioteki, takie jak tokio (dla asynchroniczności) czy async-std, implementują własne menedżery wątków, umożliwiające bardziej zaawansowane zarządzanie zadaniami.

W celu maksymalizacji wykorzystania zasobów obliczeniowych, wiele implementacji pul wątków w Rust stosuje strategię kradzieży zadań \eng{work stealing}. Mechanizm ten polega na dynamicznym równoważeniu obciążenia przez umożliwienie wątkom pobierania zadań z kolejek innych wątków, gdy ich własne kolejki są puste. Zwiększa to ogólną wydajność i skraca czas przetwarzania zadań.

Strategia ta znajduje zastosowanie m.in. w implementacji puli wątków biblioteki rayon, co czyni ją wysoce wydajną w przypadku zadań o nieregularnym czasie wykonania lub zróżnicowanym poziomie złożoności.

\subsection{Synchronizacja dostępu (Mutex, RwLock)}
Dla sytuacji wymagających współdzielenia pamięci Rust oferuje synchronizowane struktury, takie jak Mutex \eng{mutual exclusion} oraz RwLock. Umożliwiają one zarządzanie dostępem do danych w sposób bezpieczny, jednocześnie wymagając od programisty jawnego określenia momentów blokady i odblokowania zasobów.

\begin{lstlisting}[language=Rust, caption=Przykład użycia Mutex, label=mutex_example]
use std::sync::{Arc, Mutex};
use std::thread;

fn main() {
    let counter = Arc::new(Mutex::new(0));

    let mut handles = vec![];

    for _ in 0..10 {
        let counter = Arc::clone(&counter);
        let handle = thread::spawn(move || {
            let mut num = counter.lock().unwrap();
            *num += 1;
        });
        handles.push(handle);
    }

    for handle in handles {
        handle.join().unwrap();
    }

    println!("Wartość końcowa: {}", *counter.lock().unwrap());
}
\end{lstlisting}
Mutex<T> \eng{mutual exclusion}, czyli wzajemne wykluczanie, to mechanizm blokady umożliwiający bezpieczny dostęp do danych przez wiele wątków. W powyższym przykładzie dane typu i32 są opakowane w Mutex, a następnie udostępniane wielu wątkom za pomocą wskaźnika liczony atomowo Arc<T>. Każdy wątek dokonuje inkrementacji zmiennej wewnątrz sekcji krytycznej, co zapobiega wyścigom danych (ang. data races). Wywołanie lock().unwrap() blokuje wątek do czasu uzyskania wyłącznego dostępu do danych. Podejście to jest typowe w programowaniu zarówno współbieżnym (dla ochrony stanu globalnego), jak i równoległym (dla synchronizacji wyników obliczeń). 

\begin{lstlisting}[language=Rust, caption=Przykład użycia RwLock, label=rwlock_example]
use std::sync::{Arc, RwLock};
use std::thread;

fn main() {
    let data = Arc::new(RwLock::new(vec![1, 2, 3]));

    let reader = {
        let data = Arc::clone(&data);
        thread::spawn(move || {
            let read = data.read().unwrap();
            println!("Odczyt danych: {:?}", *read);
        })
    };

    let writer = {
        let data = Arc::clone(&data);
        thread::spawn(move || {
            let mut write = data.write().unwrap();
            write.push(4);
        })
    };

    reader.join().unwrap();
    writer.join().unwrap();
}
\end{lstlisting}
RwLock<T> \eng{read-write lock} czyli blokada odczytu-zapisu, umożliwia wielu wątkom jednoczesny odczyt danych przy zachowaniu wyłączności dla operacji zapisu. W przypadku często czytanych, rzadko modyfikowanych danych, takie rozwiązanie pozwala na lepszą wydajność niż klasyczny Mutex. Biblioteka standardowa Rust zapewnia gwarancje bezpieczeństwa pamięci, eliminując ryzyko naruszeń spójności danych nawet w środowiskach wielordzeniowych.  
\subsection{Wartości atomowe (Atomic*)}

Rust obsługuje także operacje na typach atomowych, które pozwalają na wykonywanie niepodzielnych operacji na współdzielonych zmiennych bez potrzeby stosowania bardziej zaawansowanych mechanizmów synchronizacji.

\begin{lstlisting}[language=Rust, caption=Przykład użycia Atomic, label=atomic_example]
use std::sync::atomic::{AtomicUsize, Ordering};
use std::thread;

fn main() {
    let counter = AtomicUsize::new(0);

    let handles: Vec<_> = (0..10)
        .map(|_| {
            let counter_ref = &counter;
            thread::spawn(move || {
                for _ in 0..1000 {
                    counter_ref.fetch_add(1, Ordering::Relaxed);
                }
            })
        })
        .collect();

    for handle in handles {
        handle.join().unwrap();
    }

    println!("Wynik: {}", counter.load(Ordering::Relaxed));
}
\end{lstlisting}
Typy Atomic* w Rust, takie jak AtomicUsize, AtomicBool, czy AtomicPtr, umożliwiają bezpieczne operacje na danych współdzielonych bez stosowania mechanizmów blokujących \eng{lock-free synchronization}. W przedstawionym przykładzie listing \ref{atomic_example}, funkcja fetch\_add wykonuje inkrementację wartości w sposób atomowy, zapewniając spójność danych nawet przy równoczesnym dostępie z wielu wątków. Choć Ordering::Relaxed zapewnia najmniejszy narzut, istnieją też silniejsze modele spójności pamięci (np. SeqCst, Acquire/Release), które mogą być konieczne przy bardziej złożonych zależnościach.

\subsection{Bariery}
Rust oferuje również podstawowe prymitywy synchronizacyjne, takie jak bariery, które umożliwiają synchronizację wątków w bardziej złożonych scenariuszach. Bariery pozwalają na zsynchronizowanie grupy wątków, które muszą osiągnąć określony punkt przed kontynuowaniem pracy, natomiast semafory kontrolują dostęp do ograniczonej liczby zasobów.
\begin{lstlisting}[language=Rust, caption=Przykład użycia bariery, label=barrier_example]
use std::sync::{Arc, Barrier};
use std::thread;

fn main() {
    let barrier = Arc::new(Barrier::new(3));
    let mut handles = vec![];

    for i in 0..3 {
        let c = Arc::clone(&barrier);
        let handle = thread::spawn(move || {
            println!("Wątek {}: przed barierą", i);
            c.wait(); // punkt synchronizacji
            println!("Wątek {}: po barierze", i);
        });
        handles.push(handle);
    }

    for handle in handles {
        handle.join().unwrap();
    }
}
\end{lstlisting}
Barrier (bariera synchronizacyjna) to mechanizm służący do synchronizacji wielu wątków w ustalonym punkcie programu. Każdy wątek, który osiąga barierę, zostaje zablokowany do momentu, aż dojdą do niej wszystkie pozostałe wątki zadeklarowane przy jej utworzeniu. Przykład ten przedstawia trójwątkową synchronizację – żaden z wątków nie przejdzie do fazy „po barierze”, dopóki wszystkie nie osiągną funkcji wait(). Bariera jest przydatna w systemach opartych na etapowym przetwarzaniu, np. w modelu SIMD czy w systemach obliczeń wielofazowych.