\section{Programowanie współbieżne}
Współbieżność to zdolność systemu do obsługi wielu zadań, które potencjalnie mogą się nakładać w czasie. Współbieżność odnosi się do zdolności systemu do jednoczesnego obsługiwania wielu zadań, które mogą mieć miejsce w tym samym czasie. Język Rust został zaprojektowany z uwzględnieniem bezpieczeństwa i wydajności w kontekście współbieżności, co czyni go niezwykle atrakcyjnym narzędziem dla programistów zajmujących się systemami wielowątkowymi.

\subsection{Model własności pamięci}
Centralnym elementem podejścia Rusta do współbieżności jest model własności pamięci, który umożliwia programistom tworzenie kodu współbieżnego bez ryzyka wystąpienia błędów związanych z niebezpiecznym dostępem do współdzielonej pamięci. Mechanizm ten, oparty na koncepcjach własności, pożyczania oraz systemu typów, pozwala kompilatorowi Rust na statyczne wykrywanie potencjalnych problemów, takich jak wyścigi danych (data races). Własność pamięci zapewnia, że tylko jeden wątek może posiadać mutowalny dostęp do danej zmiennej w danym czasie, eliminując tym samym możliwość niespodziewanej ingerencji ze strony innych wątków \ref{ownership_borrow}.

\subsection{Biblioteki}
\subsubsection{Tokio}
Tokio to biblioteka służąca do obsługi zadań asynchronicznych oraz wejścia-wyjścia bez blokad \eng{non-blocking I/O}. Wykorzystuje pętlę zdarzeń \eng{event loop}, dzięki której możliwe jest współbieżne przetwarzanie wielu zadań bez konieczności tworzenia oddzielnych wątków systemowych dla każdego z nich. Tokio wspiera zarówno jednowątkowy jak i wielowątkowy tryb działania, co pozwala na rozwój aplikacji zgodnie z potrzebami.

\begin{lstlisting}[language=Rust, caption=Przykład użycia Tokio, label=tokio_example]
use tokio::net::TcpListener;
use tokio::io::{AsyncReadExt, AsyncWriteExt};

#[tokio::main]
async fn main() {
    let listener = TcpListener::bind("127.0.0.1:8080").await.unwrap();
    loop {
        let (mut socket, _) = listener.accept().await.unwrap();
        tokio::spawn(async move {
            let mut buf = [0; 1024];
            let n = socket.read(&mut buf).await.unwrap();
            socket.write_all(&buf[0..n]).await.unwrap();
        });
    }
}
\end{lstlisting}
W powyższym przykładzie listing \ref{tokio_example} tokio::spawn inicjuje współbieżne zadanie, które przetwarza połączenie TCP bez blokowania głównej pętli programu.

\subsubsection{Crossbeam::channel - komunikacja między wątkami}
Crossbeam to zestaw narzędzi wspierających programowanie współbieżne. Jednym z kluczowych komponentów tej biblioteki są kanały komunikacyjne \eng{channels}, implementowane w module crossbeam::channel. Stanowią one bezpieczny i efektywny sposób przesyłania wiadomości pomiędzy wątkami, umożliwiając implementację wzorca producent-konsument.

\begin{lstlisting}[language=Rust, caption=Przykład użycia kanałów Crossbeam, label=crossbeam_channel_example]
use crossbeam::channel;
use std::thread;

fn main() {
    let (sender, receiver) = channel::unbounded();

    let producer = thread::spawn(move || {
        for i in 0..5 {
            sender.send(i).unwrap();
        }
    });

    let consumer = thread::spawn(move || {
        while let Ok(msg) = receiver.recv() {
            println!("Odebrano: {}", msg);
        }
    });

    producer.join().unwrap();
    consumer.join().unwrap();
}
\end{lstlisting}
W tym przypadku listing \ref{crossbeam_channel_example} channel::unbounded() tworzy kanał bez ograniczenia pojemności \eng{ unbounded channel}, który może służyć do swobodnej komunikacji między wątkami.
\subsubsection{Actix}
Actix to framework do tworzenia współbieżnych aplikacji opartych na modelu aktorowym. W tym podejściu jednostką obliczeniową jest aktor, który posiada własny stan i komunikuje się z innymi aktorami wyłącznie za pomocą komunikatów. Takie podejście eliminuje potrzebę stosowania blokad.
\begin{lstlisting}[language=Rust, caption=Przykład użycia Actix, label=actix_example]
use actix::prelude::*;

struct Ping;

impl Message for Ping {
    type Result = &'static str;
}

struct MyActor;

impl Actor for MyActor {
    type Context = Context<Self>;
}

impl Handler<Ping> for MyActor {
    type Result = &'static str;

    fn handle(&mut self, _: Ping, _: &mut Context<Self>) -> Self::Result {
        "pong"
    }
}

#[actix::main]
async fn main() {
    let addr = MyActor.start();
    let res = addr.send(Ping).await.unwrap();
    println!("{}", res);
}
\end{lstlisting}
W zaprezentowanym przykładzie aktor MyActor odbiera wiadomości typu Ping i odpowiada komunikatem pong. Każdy aktor działa w swoim własnym kontekście, co zapewnia izolację stanu i zwiększa bezpieczeństwo współbieżne bez użycia mutexów.

\subsection{Kanały}

Kanały \eng{channels} to mechanizm w języku Rust, który umożliwia bezpieczną i efektywną komunikację między wątkami. Kanały są jednym z kluczowych elementów modelu współbieżności w tym języku, zapewniając sposób przesyłania danych z jednego wątku do drugiego, jednocześnie minimalizując ryzyko problemów związanych z współdzieleniem pamięci. Kanały są oparte na wzorcu producenta-konsumenta, gdzie jeden wątek (producent) wysyła dane, a inny wątek (konsument) je odbiera.
\\

W Rust kanały są częścią standardowej biblioteki i implementują model komunikacji oparty na kolejkach \textbf{FIFO} (\textit{First In, First Out}). W kontekście Rust'a kanały są realizowane przez struktury \textbf{\textit{Sender}} (nadawca) i \textbf{\textit{Receiver}} (odbiorca), które stanowią mechanizm do przesyłania danych między wątkami. Kanały zapewniają synchronizację pomiędzy wątkami, eliminując potrzebę bezpośredniego współdzielenia pamięci w sposób, który mógłby prowadzić do niepożądanych efektów ubocznych, takich jak błędy związane z równoległym dostępem do tej samej przestrzeni pamięci.
\\

Kanały w Rust działają na zasadzie przekazywania wartości z wątku do wątku. Są one bezpieczne, ponieważ Rust zapewnia, że nie wystąpią wyścigi danych — Sender jest odpowiedzialny za wysyłanie danych, a Receiver za ich odbiór. Rust automatycznie zapewnia synchronizację, więc nie ma potrzeby stosowania dodatkowych mechanizmów, jak blokady mutexów, do zarządzania dostępem do pamięci.

\subsubsection{Tworzenie kanałów}
Kanał można stworzyć za pomocą funkcji \textit{mpsc::channel()} z modułu \textit{std::sync}. Ta funkcja zwraca dwie wartości: nadawcę (Sender) i odbiorcę (Receiver).

\begin{lstlisting}[language=Rust, caption=Przykład tworzenia kanału, label=channel_spawn]
use std::sync::mpsc;
use std::thread;

fn main() {
    // Tworzenie kanału
    let (tx, rx) = mpsc::channel();

    // Tworzenie wątku producenta
    thread::spawn(move || {
        let value = String::from("Hello from the producer!");
        tx.send(value).expect("Failed to send message");
    });

    // Odbiór wiadomości w wątku konsumenta
    let received = rx.recv().expect("Failed to receive message");
    println!("Received: {}", received);
}
\end{lstlisting}
Opis wykonanych kroków w celu utworzenia kanałów w listingu \ref{channel_spawn}:
\begin{enumerate}
    \item Tworzymy kanał za pomocą \textit{mpsc::channel()}, który zwraca parę (tx, rx) — tx jest nadawcą, a rx odbiorcą.
    \item Tworzymy nowy wątek (producenta), który wysyła wiadomość ("\textit{Hello from the producer!}") do kanału.
    \item W głównym wątku (konsument) odbieramy wiadomość za pomocą \textit{recv()} i drukujemy ją na ekranie.
\end{enumerate}

\subsubsection{Wysyłanie i odbieranie danych}
Nadawca (Sender) jest używany do wysyłania danych do kanału. Można wysłać dowolny typ, który jest przesyłany przez kanał. Funkcja send() jest używana do wysyłania wartości, a recv() w odbiorcy blokuje wątek do momentu otrzymania danych.\\
Odbiorca (Receiver) odbiera dane z kanału. Jest to blokująca operacja, co oznacza, że wątek odbiorcy będzie czekał, aż dane będą dostępne do odebrania.

\subsubsection{Przykład wielokrotnego odbioru}
Kanały w Rust są domyślnie jednokierunkowe, co oznacza, że tylko jeden odbiorca może odbierać wiadomości z kanału. Możemy jednak tworzyć wiele kanałów i różne wątki odbiorcze, aby efektywnie rozdzielać zadania.

\begin{lstlisting}[language=Rust, caption=Przykład z wieloma wątkami, label=multi_channel_spawn]
use std::sync::{mpsc, Arc, Mutex};
use std::thread;

fn main() {
    let (tx, rx) = mpsc::channel();
    let rx = Arc::new(Mutex::new(rx));

    // Tworzenie 3 wątków konsumentów
    for i in 0..3 {
        let rx = Arc::clone(&rx);  // Klonowanie Arc, nie Receiver
        thread::spawn(move || {
            let message = rx.lock().unwrap().recv().expect("Failed to receive message");
            println!("Consumer {} received: {}", i, message);
        });
    }

    // Wysyłanie wiadomości do konsumentów
    for i in 0..3 {
        let message = format!("Message {}", i);
        tx.send(message).expect("Failed to send message");
    }

    // Zatrzymanie wątku głównego, aby konsument mógł zakończyć pracę
    thread::sleep(std::time::Duration::from_secs(1));
}

\end{lstlisting}
W tym przykładzie \ref{multi_channel_spawn} tworzymy trzy wątki konsumentów, z których każdy odbiera wiadomości z tego samego kanału. Wątek główny wysyła trzy wiadomości, które są odbierane przez konsumentów.
\begin{itemize}
    \item Dodatkowo została użyta struktura \textbf{\textit{Arc<Mutex<Receiver> >}} \ref{ARC} do umożliwienia współdzielenia odbiorcy (Receiver) między wątkami. Mutex zapewnia synchronizację dostępu do kanału, dzięki czemu tylko jeden wątek w danej chwili może odbierać wiadomości.
    \item \textbf{\textit{Arc::clone}} klonuje Arc, a nie Receiver. \textit{Arc} tworzy nowy uchwyt do tego samego obiektu w pamięci, dzięki czemu wszystkie wątki mogą uzyskać dostęp do tego samego kanału.
    \item \textbf{\textit{lock()}} Każdy wątek przed odebraniem wiadomości blokuje mutex, aby uzyskać dostęp do kanału w bezpieczny sposób.
\end{itemize}
Przykładowa odpowiedź programu:
\begin{verbatim}
Consumer 0 received: Message 0
Consumer 2 received: Message 2
Consumer 1 received: Message 1
\end{verbatim}

\subsubsection{Zakończenie pracy z kanałami}
Kanał w Rust jest "ograniczony" — po wysłaniu wszystkich danych nadawca (sender) automatycznie zakończy swoje działanie, co powoduje, że funkcja recv() w odbiorcy zwróci błąd, jeśli nie będzie więcej wiadomości. Można także zakończyć kanał, jeśli w danym wątku nie ma już nadawców.
\begin{lstlisting}[language=Rust, caption=Zakończenie kanału, label=channel_ending]
use std::sync::mpsc;
use std::thread;

fn main() {
    let (tx, rx) = mpsc::channel();

    thread::spawn(move || {
        tx.send(String::from("End of messages")).expect("Send failed");
    });

    match rx.recv() {
        Ok(msg) => println!("Received: {}", msg),
        Err(_) => println!("No more messages!"),
    }
    match rx.recv() {
        Ok(msg) => println!("Received: {}", msg),
        Err(_) => println!("No more messages!"),
    }
}

\end{lstlisting}
W tym przykładzie \ref{channel_ending} po wysłaniu wiadomości przez nadawcę wątku głównego, kanał zostaje zamknięty, co sprawia, że dalsze wywołanie recv() w odbiorcy zwróci błąd.\\
Odpowiedź programu:
\begin{verbatim}
Received: End of messages
No more messages!
\end{verbatim}

\subsection{Asynchroniczność}
Chociaż współbieżność i asynchroniczność nie są tożsame, w Rust oba te podejścia są ze sobą ściśle powiązane. Rust implementuje asynchroniczność za pomocą konstrukcji takich jak async/await oraz bibliotek takich jak Tokio czy async-std. Podejście to pozwala na wykonywanie wielu operacji współbieżnie w ramach pojedynczego wątku, eliminując narzut związany z tworzeniem wielu wątków.
Podstawowe elementy asynchroniczności to:
\begin{itemize}
    \item async / await - konstrukcje językowe umożliwiające deklaratywne definiowanie funkcji asynchronicznych i ich późniejsze wywoływanie bez blokowania wątku.
    \item Futura - abstrakcja reprezentująca wartość obliczaną asynchronicznie, której rezultat będzie dostępny w przyszłości. Obliczenia tego typu są inicjowane leniwie i wymagają uruchomienia w ramach środowiska wykonawczego.
    \item Tokio i async-std - najpopularniejsze asynchroniczne runtimes, implementujące własne planisty i pętle zdarzeń (event loops).
    \item Asynchroniczne kanały (tokio::sync::mpsc, broadcast, oneshot) - umożliwiają komunikację bez blokowania wątków.
    \item Asynchroniczne semafory (tokio::sync::Semaphore) - pozwalają ograniczyć liczbę jednocześnie wykonywanych zadań asynchronicznych bez blokowania wątku. Są często stosowane do kontroli równoczesnych operacji na zasobach zewnętrznych, takich jak dostęp do bazy danych lub API.
\end{itemize}
W standardowej bibliotece Rust nie ma semafora, ale tokio udostępnia asynchroniczny semafor (tokio::sync::Semaphore), który można również użyć w środowisku wielowątkowym z tokio::main.

\begin{lstlisting}[language=Rust, caption=Przykład użycia semafora, label=semaphore_example]
use tokio::sync::Semaphore;
use std::sync::Arc;
use tokio::task;
use std::time::Duration;

#[tokio::main]
async fn main() {
    let semaphore = Arc::new(Semaphore::new(2)); // maks. 2 równoczesne zasoby
    let mut handles = vec![];

    for i in 0..5 {
        let sem_clone = Arc::clone(&semaphore);
        let handle = task::spawn(async move {
            let permit = sem_clone.acquire().await.unwrap();
            println!("Zadanie {}: rozpoczęło pracę", i);

            tokio::time::sleep(Duration::from_secs(1)).await;

            println!("Zadanie {}: kończy pracę", i);
            drop(permit); // zwolnienie zasobu
        });
        handles.push(handle);
    }

    for handle in handles {
        handle.await.unwrap();
    }
}
\end{lstlisting}
W tym przykładzie listing \ref{semaphore_example} semafor o pojemności 2 (Semaphore::new(2)) umożliwia maksymalnie dwóm zadaniom równoczesny dostęp do sekcji krytycznej. Pozostałe zadania oczekują w kolejce, aż jedno zezwolenie zostanie zwolnione. tokio::sync::Semaphore jest semaforem asynchronicznym (nie blokuje fizycznego wątku), dlatego nadaje się zarówno do środowisk asynchronicznych, jak i do ograniczania zasobów w systemach hybrydowych (np. współbieżno-równoległych).