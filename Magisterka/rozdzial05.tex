%Porównanie Rust i C++
\chapter{Wybrane mechanizmy w języku C++}
\section{Programowanie współbieżne}


\section{Programowanie równoległe}

\subsection{OpenMP}
Biblioteka ta pozwala uruchomić wybraną część kodu na wielu wątkach. Działa to podobnie do uruchamiania kody z wykorzystaniem GPU - kod, który zostanie uruchomiony na wielu wątkach umieszczany jest w specjalnym zakresie \eng{scope}. 
Kompilacja kodu z OpenMP odbywa się z wykorzystaniem flagi \texttt{-fopenmp} w przypadku kompilatora \texttt{g++}.\\
Użycie pamięci cache może pomóc w zwiększeniu wydajności czasowej programu kosztem pamięci. W przypadku OpenMP, zmienna \texttt{shared} oznacza, że zmienna jest współdzielona pomiędzy wątkami, a \texttt{private} oznacza, że każdy wątek ma swoją kopię zmiennej?\\