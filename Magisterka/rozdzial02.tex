\chapter{Cel oraz zakres pracy}
% \addcontentsline{toc}{chapter}{Cel i zakres pracy}  % Add unnumbered chapter to the table of contents if needed
Celem niniejszej pracy jest przeprowadzenie pogłębionej analizy oraz wszechstronnego porównania mechanizmów programowania współbieżnego i równoległego w dwóch językach programowania: Rust i C++. Celem jest przedstawienie kluczowych różnic oraz podobieństw w podejściu do zarządzania wielowątkowością, analizując jednocześnie efektywność, bezpieczeństwo oraz wygodę stosowania narzędzi dostępnych w obu językach.

W ramach pracy szczególną uwagę poświęcono omówieniu wybranych bibliotek i frameworków, które wspierają tworzenie aplikacji wielowątkowych w Rust (np. Tokio, Rayon) i C++ (np. std::thread, OpenMP, TBB). Przeanalizowane zostaną mechanizmy bezpieczeństwa oraz zarządzania pamięcią i wątkami, które odgrywają kluczową rolę w zapewnieniu stabilności i wydajności aplikacji współbieżnych i równoległych.

Dodatkowym celem jest przeprowadzenie analizy wydajności oraz efektywności implementacji aplikacji wielowątkowych, co pozwoli na ocenę szybkości działania i efektywnego zarządzania zasobami w obu językach. Badanie uwzględni również aspekty praktyczne, takie jak łatwość użycia narzędzi, dostępność wsparcia ze strony społeczności oraz dojrzałość ekosystemu każdego z języków.

Aby zilustrować wyniki teoretyczne w praktyce, przeprowadzona zostanie implementacja aplikacji współbieżnych i równoległych w obu językach, co umożliwi porównanie osiągniętych wyników wydajnościowych oraz analizę różnic w strukturze i stylu kodu. Efektem pracy będzie również identyfikacja scenariuszy, w których jeden z języków może przewyższać drugi pod względem wydajności, bezpieczeństwa, czy wygody stosowania, co pozwoli na sformułowanie rekomendacji dotyczących wyboru języka w zależności od specyficznych wymagań projektowych.