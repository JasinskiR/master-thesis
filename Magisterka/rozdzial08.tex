%Porównanie Rust i C++
\chapter{Porównanie międzyjęzykowe - programowanie współbieżne}

W ramach analizy programowania współbieżnego przeanalizowano implementacje wzorców producent-konsument oraz serwer echo w językach Rust i C++. Obie wersje korzystają z różnych podejść do zarządzania współbieżnością: Rust z biblioteką Tokio oraz C++ z mechanizmami standardowej biblioteki i rozszerzeniami języka C++20.

\section{Porównanie międzyjęzykowe}

\subsection{Struktura i organizacja kodu}

\begin{table}[H]
    \centering
    \caption{Porównanie aspektów struktury i organizacji kodu w implementacjach współbieżnych}
    \begin{tabularx}{\textwidth}{lXX}
        \toprule
        \textbf{Aspekt} &
        \textbf{Rust (Tokio)} &
        \textbf{C++ (std::thread + epoll/kqueue)} \\
        \midrule
        Architektura &
        Sterowana zdarzeniami z asynchronicznym środowiskiem wykonawczym, tworzenie zadań &
        Wątek na połączenie + opcjonalna pętla zdarzeń \\
        \hline
        Enkapsulacja &
        Moduły, cechy, system własności &
        Klasy, szablony, przestrzenie nazw \\
        \hline
        Obsługa błędów &
        \texttt{Result<T,E>}, operator \texttt{?}, wymuszona propagacja &
        \texttt{std::optional}, wyjątki, kody \texttt{errno} \\
        \hline
        Asynchroniczność &
        \texttt{async/await}, \texttt{tokio::select!}, wbudowane środowisko wykonawcze &
        \texttt{std::async}, ręczne pętle zdarzeń (epoll/kqueue) \\
        \hline
        Bezpieczeństwo typów &
        Sprawdzanie własności w czasie kompilacji, cechy \texttt{Send/Sync} &
        Sprawdzanie w czasie wykonania, ręczna synchronizacja \\
        \hline
        Konfiguracja &
        Cargo.toml, flagi funkcjonalności \eng{feature flags} &
        CMake/Make, dyrektywy preprocesora \\
        \hline
        Zarządzanie zależnościami &
        Scentralizowane z crates.io &
        Zdecentralizowane, pakiety systemowe \\
        \bottomrule
    \end{tabularx}
\end{table}

Implementacja w języku Rust cechuje się większą spójnością architektoniczną dzięki wbudowanym mechanizmom asynchroniczności oraz systemowi cech \eng{traits}. C++ zapewnia większą elastyczność kosztem złożoności zarządzania zgodnością między różnymi wersjami standardu.


\subsection{Zarządzanie pamięcią}

\begin{table}[H]
    \centering
    \caption{Porównanie modeli zarządzania pamięcią w badanych implementacjach}
    \renewcommand{\arraystretch}{1.3}
    \begin{tabularx}{\textwidth}{lX X}
        \toprule
        \textbf{Aspekt} &
        \textbf{Rust} &
        \textbf{C++} \\
        \midrule
        Model podstawowy &
        Własność z automatycznym zwalnianiem; sprawdzanie pożyczania w czasie kompilacji &
        RAII z ręcznym zarządzaniem czasem życia obiektów \\
        \midrule
        Współdzielenie w wątkach &
        \texttt{Arc<T>} dla danych niemutowalnych, \texttt{Arc<Mutex<T>>} dla mutowalnych &
        \texttt{std::shared\_ptr<T>}, ręczne opakowanie muteksem \\
        \midrule
        Bezpieczeństwo pamięci &
        Gwarantowane na etapie kompilacji, eliminacja wyścigów danych &
        Błędy ujawniane w czasie wykonania; ryzyko użycia po zwolnieniu \\
        \midrule
        Zwalnianie zasobów &
        Automatyczne dzięki cechom \texttt{Drop} &
        Destruktory RAII, często konieczne jawne zwolnienie (np. \texttt{close()}) \\
        \midrule
        Pomiar pamięci &
        Ograniczony dostęp do systemu plików \texttt{/proc} &
        Pełna analiza wykorzystania pamięci (RSS, sterta) \\
        \midrule
        Zarządzanie buforami &
        Bezpieczne wskaźniki i struktury (\texttt{Vec<T>}, \texttt{slice}) ze sprawdzaniem granic &
        Surowe wskaźniki i tablice, ryzyko przepełnień \\
        \midrule
        Porządek pamięci &
        \texttt{Ordering::Relaxed}, \texttt{Acquire}, \texttt{Release} &
        Odpowiedniki w \texttt{std::memory\_order} \\
        \midrule
        Wykrywanie wycieków &
        Rzadkie (np. cykliczne referencje z \texttt{Arc}); wspierane przez system typów &
        Częsty problem, wymagane zewnętrzne narzędzia (np. Valgrind) \\
        \bottomrule
    \end{tabularx}
\end{table}



System własności w języku Rust eliminuje całe klasy błędów pamięciowych na etapie kompilacji. C++ wymaga większej ostrożności programisty oraz wykorzystania narzędzi diagnostycznych.

\subsection{Mechanizmy współbieżności}

\begin{table}[H]
    \centering
    \caption{Porównanie mechanizmów współbieżności w badanych implementacjach}
    \renewcommand{\arraystretch}{1.3}
    \begin{tabularx}{\textwidth}{l>{\raggedright\arraybackslash}X>{\raggedright\arraybackslash}X}
        \toprule
        \textbf{Mechanizm} &
        \textbf{Rust (Tokio)} &
        \textbf{C++ (std + wywołania systemowe)} \\
        \midrule
        Model wykonania &
        Wątki M:N z planistą opartym na kradzieży zadań (work-stealing scheduler) &
        Wątki 1:1 zarządzane przez systemowy planista \\
        \midrule
        Tworzenie zadań &
        \texttt{tokio::spawn(async move \{\})} - lekki i szybki sposób uruchamiania zadań asynchronicznych &
        \texttt{std::thread::spawn()} z opcjonalnym \texttt{.detach()} lub synchronizacją przez \texttt{join()} \\
        \midrule
        Komunikacja &
        Kanały wieloproducent-pojedynczykonsument: \texttt{mpsc::channel} z \texttt{Arc<Mutex<Receiver>>} &
        Własna implementacja \texttt{Channel<T>} oparta na \texttt{std::condition\_variable} \\
        \midrule
        Synchronizacja &
        Primitives: \texttt{tokio::sync::Mutex}, \texttt{Semaphore}, \texttt{RwLock} - dostosowane do async &
        Klasyczne prymitywy: \texttt{std::mutex}, \texttt{condition\_variable}, atomiki z \texttt{std::atomic} \\
        \midrule
        Operacje wejścia/wyjścia &
        Nieblokujące I/O z użyciem await, np. \texttt{TcpListener::bind().await} &
        Blokujące systemowe wywołania: \texttt{accept()}, \texttt{recv()}, \texttt{send()} \\
        \midrule
        Obsługa zdarzeń &
        Asynchroniczne makro \texttt{tokio::select!} do oczekiwania na wiele zdarzeń &
        Ręczne pętle zdarzeń z wykorzystaniem \texttt{epoll\_wait()} lub \texttt{kevent()} \\
        \midrule
        Ograniczenia zasobów &
        Semafory asynchroniczne, np. \texttt{Semaphore::try\_acquire\_owned()} &
        Własne liczniki z użyciem \texttt{std::atomic<size\_t>} \\
        \midrule
        Koordynacja zamknięcia &
        Kanały rozgłoszeniowe, np. \texttt{broadcast::channel} z łagodnym zamykaniem (graceful shutdown) &
        Flagi typu \texttt{std::atomic<bool>} sygnalizujące zakończenie wątku \\
        \midrule
        Zapobieganie wyścigom danych &
        Zapewnione na etapie kompilacji przez ograniczenia traitów \texttt{Send} i \texttt{Sync} &
        Potrzebna synchronizacja ręczna; ryzyko błędów wykonawczych i wyścigów danych \\
        \bottomrule
    \end{tabularx}
\end{table}


Porównując mechanizmy współbieżności w językach Rust i C++, można zauważyć fundamentalne różnice w filozofii projektowej i gwarancjach bezpieczeństwa. Rust przyjmuje podejście "bezpieczeństwo przede wszystkim", podczas gdy C++ oferuje większą elastyczność, ale wymaga większej dyscypliny od programisty.

\subsubsection{Rust - kanały typowane}
Implementacja Rust wykorzystuje kanały z różnymi semantykami komunikacji (\texttt{broadcast}):
\begin{lstlisting}[language=Rust, caption={Producent-Konsument w Rust z mpsc channels}, label={lst:rust_producer_consumer}]
// Rust: producent-konsument z kanałami mpsc
fn producer_consumer_benchmark() {
    let (tx, rx) = mpsc::channel();
    let rx = Arc::new(Mutex::new(rx));
    let shutdown = Arc::new(AtomicBool::new(false));
    
    // Wątki producentów
    thread::spawn(move || {
        for i in 0..ITEMS {
            tx.send(format!("Item-{}", i)).unwrap();
        }
        // Kanał zamyka się automatycznie po usunięciu tx
    });

    // Wątki konsumentów - współdzielony odbiornik z Arc<Mutex<>>
    let rx_clone = Arc::clone(&rx);
    thread::spawn(move || {
        while let Ok(msg) = rx_clone.lock().unwrap().try_recv() {
            process_message(msg);
        }
    });
}
\end{lstlisting}

\subsubsection{C++ - ręczna synchronizacja}
Wersja C++ opiera się na ręcznym zarządzaniu współbieżnością z użyciem mutexów i zmiennych warunkowych:
\begin{lstlisting}[language=C++, style=VS2017,  caption={Producent-Konsument w C++ z Channel}, label={lst:cpp_producer_consumer}]
// C++: Własny kanał z ręczną synchronizacją
template<typename T>
class Channel {
    std::queue<T> queue_;
    mutable std::mutex mutex_;
    std::condition_variable condition_;
    bool closed_ = false;
    
public:
    void send(T item) { 
        std::lock_guard lock(mutex_); 
        queue_.push(std::move(item)); condition_.notify_one(); 
    }
    
    bool recv(T& item) {
        std::unique_lock lock(mutex_);
        condition_.wait(lock, [this]{ return !queue_.empty() || closed_; });
        if (queue_.empty()) return false;
        item = std::move(queue_.front()); queue_.pop();
        return true;
    }
    
    void close() { std::lock_guard lock(mutex_); closed_ = true; condition_.notify_all(); }
};
\end{lstlisting}

\subsubsection{Serwera echo}

\subsubsection{Rust - wejście/wyjście sterowane zdarzeniami}
Implementacja Rust wykorzystuje nieblokujące I/O:
\begin{lstlisting}[language=Rust, caption={Echo Serwer w Rust z Tokio}, label={lst:rust_echo_server}]
// Rust: Asynchroniczny echo serwer z Tokio
async fn run_echo_server(addr: &str, max_connections: usize) {
    let listener = TcpListener::bind(addr).await.unwrap();
    let semaphore = Arc::new(Semaphore::new(max_connections));
    
    loop {
        tokio::select! {
            Ok((stream, addr)) = listener.accept() => {
                if let Ok(permit) = semaphore.clone().try_acquire_owned() {
                    tokio::spawn(async move {
                        handle_client(stream, addr).await;
                        drop(permit); // Auto connection limit
                    });
                }
            }
            _ = shutdown_signal() => break,
        }
    }
}

async fn handle_client(mut stream: TcpStream, addr: SocketAddr) {
    let mut buffer = [0; 1024];
    while let Ok(n) = stream.read(&mut buffer).await {
        if n == 0 { break; }
        stream.write_all(&buffer[..n]).await.unwrap(); // Echo back
    }
}
\end{lstlisting}

\subsubsection{C++ - jeden wątek na połączenie}
Wersja C++ tworzy osobny wątek dla każdego klienta:
\begin{lstlisting}[language=C++, style=VS2017,  caption={Echo Serwer w C++ z jednym wątkiem na połączenie}, label={lst:cpp_echo_server}]
// C++: Traditional Thread-per-Connection Model
class EchoServer {
    int server_socket;
    std::atomic<bool> running{true};
    std::atomic<size_t> current_connections{0};
    size_t max_connections;
    
public:
    EchoServer(const std::string& addr, int port, size_t max_conn) : max_connections(max_conn) {
        server_socket = socket(AF_INET, SOCK_STREAM, 0);
        sockaddr_in server_addr{AF_INET, htons(port), {INADDR_ANY}};
        bind(server_socket, (sockaddr*)&server_addr, sizeof(server_addr));
        listen(server_socket, SOMAXCONN);
    }
    
    void run() {
        while (running.load()) {
            int client = accept(server_socket, nullptr, nullptr);
            if (current_connections.load() < max_connections) {
                // C++: Each connection = new OS thread
                std::thread([this, client]() {
                    current_connections++;
                    char buffer[1024];
                    while (int n = recv(client, buffer, sizeof(buffer), 0)) {
                        send(client, buffer, n, 0); // Echo back
                    }
                    close(client);
                    current_connections--;
                }).detach();
            } else {
                close(client); // Reject if at capacity
            }
        }
    }
};
\end{lstlisting}

\subsubsection{C++ - wersja asynchroniczna z epoll/kqueue}
C++ oferuje także asynchroniczną implementację używającą \texttt{epoll} (Linux) lub \texttt{kqueue} \mbox{(macOS)}:
\begin{lstlisting}[language=C++, style=VS2017,  caption={Async Echo Serwer w C++ z event loop}, label={lst:cpp_async_echo_server}]
class AsyncEchoServer {
    int server_socket;
    std::atomic<bool> running{true};
    std::vector<int> active_clients;
    
#ifdef __APPLE__
    int kqueue_fd;  // macOS: kqueue for event multiplexing
#else
    int epoll_fd;   // Linux: epoll for event multiplexing  
#endif

public:
    AsyncEchoServer(const std::string& addr, int port) {
        server_socket = socket(AF_INET, SOCK_STREAM, 0);
        fcntl(server_socket, F_SETFL, O_NONBLOCK); // Non-blocking socket
        
        sockaddr_in server_addr{AF_INET, htons(port), {INADDR_ANY}};
        bind(server_socket, (sockaddr*)&server_addr, sizeof(server_addr));
        listen(server_socket, SOMAXCONN);
        
#ifdef __APPLE__
        kqueue_fd = kqueue();
        struct kevent ev;
        EV_SET(&ev, server_socket, EVFILT_READ, EV_ADD, 0, 0, nullptr);
        kevent(kqueue_fd, &ev, 1, nullptr, 0, nullptr);
#else
        epoll_fd = epoll_create1(0);
        struct epoll_event ev{EPOLLIN, {.fd = server_socket}};
        epoll_ctl(epoll_fd, EPOLL_CTL_ADD, server_socket, &ev);
#endif
    }
    
    void run_async() {
        while (running.load()) {
#ifdef __APPLE__
            struct kevent events[64];
            int nev = kevent(kqueue_fd, nullptr, 0, events, 64, nullptr);
            
            for (int i = 0; i < nev; ++i) {
                if ((int)events[i].ident == server_socket) {
                    handle_new_connection();
                } else {
                    handle_client_data((int)events[i].ident);
                }
            }
#else
            struct epoll_event events[64];
            int nfds = epoll_wait(epoll_fd, events, 64, -1);
            
            for (int i = 0; i < nfds; ++i) {
                if (events[i].data.fd == server_socket) {
                    handle_new_connection();
                } else {
                    handle_client_data(events[i].data.fd);
                }
            }
#endif
        }
    }
    
private:
    void handle_new_connection() {
        int client = accept(server_socket, nullptr, nullptr);
        fcntl(client, F_SETFL, O_NONBLOCK);
        active_clients.push_back(client);
        
        // Add client to event system
#ifdef __APPLE__
        struct kevent ev;
        EV_SET(&ev, client, EVFILT_READ, EV_ADD, 0, 0, nullptr);
        kevent(kqueue_fd, &ev, 1, nullptr, 0, nullptr);
#else
        struct epoll_event ev{EPOLLIN, {.fd = client}};
        epoll_ctl(epoll_fd, EPOLL_CTL_ADD, client, &ev);
#endif
    }
    
    void handle_client_data(int client_socket) {
        char buffer[1024];
        int n = recv(client_socket, buffer, sizeof(buffer), 0);
        if (n > 0) {
            send(client_socket, buffer, n, 0); // Echo back
        } else {
            // Client disconnected - remove from event system
#ifdef __APPLE__
            struct kevent ev;
            EV_SET(&ev, client_socket, EVFILT_READ, EV_DELETE, 0, 0, nullptr);
            kevent(kqueue_fd, &ev, 1, nullptr, 0, nullptr);
#else
            epoll_ctl(epoll_fd, EPOLL_CTL_DEL, client_socket, nullptr);
#endif
            close(client_socket);
            active_clients.erase(
                std::remove(active_clients.begin(), active_clients.end(), client_socket),
                active_clients.end()
            );
        }
    }
};
\end{lstlisting}

\subsection{Wydajność i bezpieczeństwo}

\begin{table}[H]
    \centering
    \caption{Porównanie wydajności i bezpieczeństwa implementacji współbieżnych}
    \begin{tabularx}{\textwidth}{lXX}
        \toprule
        \textbf{Aspekt} &
        \textbf{Rust} &
        \textbf{C++} \\
        \midrule
        Bezpieczeństwo kompilacji &
        Wysokie - system pożyczania eliminuje wyścigi danych &
        Średnie - wymaga zewnętrznych narzędzi (np. ThreadSanitizer) \\
        \hline
        Narzut w czasie wykonania &
        Niski - abstrakcje bezkosztowe \eng{zero-cost} &
        Średni - narzut tworzenia wątków \\
        \hline
        Efektywność pamięci &
        Wysoka - współdzielona własność z \texttt{Arc<T>} &
        Średnia - ryzyko wycieków pamięci \\
        \hline
        Skalowanie &
        Wysokie - model M:N, kradzież zadań &
        Ograniczone - model 1:1, ograniczenia systemowe \\
        \hline
        Odzyskiwanie po błędach &
        Strukturalne - propagacja błędów przez \texttt{Result<T, E>} &
        Oparte na wyjątkach - ryzyko wycieków zasobów \\
        \hline
        Trudność debugowania &
        Średnia - błędy kompilacji z \eng{borrow checker} &
        Wysoka - wyścigi, błędy pamięci \\
        \hline
        Dojrzałość ekosystemu &
        Rozwijający się - Tokio, async-std &
        Dojrzały - OpenMP, TBB, Boost.Asio \\
        \bottomrule
    \end{tabularx}
\end{table}

\subsection{Modele wykonania i zarządzanie zadaniami}

Implementacja Rust wykorzystuje bibliotekę Tokio, która dostarcza model wątków M:N z planistą kradzieży zadań. W praktyce oznacza to, że liczba wątków systemu operacyjnego jest niezależna od liczby zadań aplikacyjnych. Funkcja \texttt{tokio::spawn()} tworzy lekkie zadania zarządzane przez środowisko wykonawcze, co umożliwia efektywne wykorzystanie zasobów przy dużej liczbie współbieżnych operacji.

Implementacja C++ opiera się na modelu 1:1, gdzie każde połączenie obsługiwane jest przez dedykowany wątek systemu operacyjnego. Alternatywna wersja asynchroniczna wykorzystuje wywołania systemowe \texttt{epoll()} na Linux lub \texttt{kevent()} na macOS do implementacji pętli zdarzeń, jednak wymaga to jawnego zarządzania stanem połączeń.

\subsubsection{Komunikacja i synchronizacja}

W implementacji Rust komunikacja realizowana jest przez \texttt{mpsc::channel()} w konfiguracji \texttt{Arc<Mutex<Receiver<T> > >} dla wielu odbiorców. Kanał zapewnia bezpieczeństwo typów i automatyczne sprzątanie przy zamknięciu wszystkich nadawców.

Implementacja C++ definiuje własną klasę \texttt{Channel<T>} opartą na \texttt{std::queue} z synchronizacją przez \texttt{std::mutex} i \texttt{std::condition\_variable}. Klasa implementuje metody \texttt{send()}, \texttt{try\_recv()} oraz \texttt{recv()} z semantyką blokującą, wymagając ręcznego zarządzania stanem zamknięcia kanału.

\subsubsection{Obsługa I/O}

Rust wykorzystuje \texttt{TcpListener::bind().await} z nieblokującymi operacjami \mbox{wejścia/wyjścia}. Makro \texttt{tokio::select!} umożliwia równoległe oczekiwanie na wiele zdarzeń (nowe połączenia, sygnał zamknięcia). Ograniczenia połączeń realizowane są przez \texttt{Semaphore::try\_acquire\_owned()}.

C++ implementuje blokujące wejście/wyjście przez bezpośrednie wywołania systemowe: \texttt{socket()}, \texttt{bind()}, \texttt{listen()}, \texttt{accept()}. Każde połączenie obsługiwane jest w dedykowanym wątku z synchronicznymi operacjami \texttt{recv()}/\texttt{send()}. Wersja asynchroniczna wymaga ręcznego zarządzania stanem dla każdego deskryptora pliku.

\subsection{Zarządzanie błędami i zasobami}

Rust wymusza jawną obsługę błędów przez typy \texttt{Result<T, E>} i operator \texttt{?}. Automatyczne sprzątanie zasobów zapewniany jest przez cechę \texttt{Drop} - połączenia gniazd, alokacje pamięci oraz uchwyty wątków są automatycznie zwalniane.

C++ opiera się na wzorcu RAII z ręcznymi wywołaniami destruktorów. Sprzątanie gniazd wymaga explicite wywołań \texttt{close()}, a obsługa błędów realizowana jest przez rzucanie wyjątków lub sprawdzanie \texttt{errno}. Wycieki pamięci mogą wystąpić przy nieprawidłowym zarządzaniu cyklem życia objektów.


\subsubsection{Bezpieczeństwo współbieżności}

Najbardziej zasadnicza różnica między Rust a C++ dotyczy podejścia do bezpieczeństwa współbieżnego. Rust stosuje rygorystyczny system typów oraz cechy \texttt{Send} i \texttt{Sync}, które gwarantują bezpieczeństwo dostępu do danych współdzielonych już na etapie kompilacji. Kompilator odrzuca programy mogące prowadzić do wyścigów danych, co eliminuje całą klasę trudnych do wykrycia błędów.

W C++ bezpieczeństwo współbieżności opiera się na odpowiedzialności programisty. Wszelkie błędy synchronizacji (np. brak blokady, nadmierne współdzielenie danych) mogą prowadzić do wyścigów, które są trudne do wykrycia i debugowania. Choć dostępne są narzędzia analityczne, takie jak ThreadSanitizer, nie eliminują one problemów przed uruchomieniem programu i nie oferują gwarancji podobnych do Rust.



\subsection{Wnioski}

Analiza rzeczywistych implementacji ujawnia konkretne kompromisy między językami w kontekście programowania współbieżnego.
Rust z Tokio charakteryzuje się:
\begin{itemize}
    \item Bezpieczeństwem w czasie kompilacji przez system własności i cechy \texttt{Send/Sync}
    \item Architekturą sterowaną zdarzeniami z efektywnym wykorzystaniem zasobów
    \item Wymuszonym obsługą błędów eliminującym ciche awarie \eng{silent failures}
    \item Ograniczonymi możliwościami introspekcji środowiska wykonawczego
    \item Zależnością od zewnętrznych pakietów \eng{external crates} dla podstawowej funkcjonalności
\end{itemize}

C++ ze standardową biblioteką oferuje:
\begin{itemize}
    \item Bezpośrednią kontrolę nad zasobami systemowymi i układem pamięci
    \item Kompleksowe monitorowanie i profilowanie w czasie wykonania  
    \item Kompatybilność międzyplatformową z ręcznymi abstrakcjami platformowymi
    \item Model wątek na połączenie z przewidywalnym użyciem zasobów
    \item Ryzyko błędów w czasie wykonania wymagające ostrożnej synchronizacji
\end{itemize}

W praktyce wybór między językami zależy od priorytetów projektu: bezpieczeństwo a~kontrola, produktywność a elastyczność, nowoczesność a kompatybilność. Należy również uwzględnić kompetencje zespołu oraz wymagania środowiska docelowego.
