\section{Programowanie równoległe}
\subsection{OpenMP -  Open Multi-Processing}
OpenMP to biblioteka umożliwiająca programowanie równoległe w modelu pamięci współdzielonej. Jest dostępna dla języków C, C++ oraz Fortran i opiera się na użyciu dyrektyw preprocesora pozwalających na uproszczone rozproszenie zadań pomiędzy wątki w sposób deklaratywny.

Podstawowym celem OpenMP jest ułatwienie implementacji aplikacji równoległych poprzez maksymalne ograniczenie ręcznego zarządzania wątkami i synchronizacją. W kodzie C++ użycie tej technologii wymaga aktywacji flagi -fopenmp podczas kompilacji przy użyciu kompilatora \texttt{g++}.
Przedstawia podstawowe pojęcia:
\begin{itemize}
    \item \texttt{\#pragma omp parallel} — dyrektywa inicjująca blok kodu, który ma zostać wykonany równolegle przez wiele wątków,
    \item \texttt{shared} — oznacza zmienne współdzielone pomiędzy wszystkimi wątkami,
    \item \texttt{private} — każdemu wątkowi przypisywana jest prywatna kopia zmiennej,
    \item \texttt{cache locality} — pamięć podręczna \eng{cache} może znacznie poprawić wydajność przetwarzania, choć kosztem większego zużycia pamięci.
\end{itemize}

\begin{lstlisting}[language=C++, caption={Przykład użycia OpenMP w C++}, label={lst:openmp_example}]
#include <omp.h>
#include <iostream>

int main() {
    int sum = 0;
    #pragma omp parallel for reduction(+:sum)
    for (int i = 1; i <= 100; ++i) {
        sum += i;
    }
    std::cout << "Suma: " << sum << std::endl;
    return 0;
}
\end{lstlisting}    
W powyższym przykładzie w listingu \ref{lst:openmp_example} zmienna \texttt{sum} jest sumą liczb od 1 do 100 obliczaną równolegle. Dzięki użyciu dyrektywy \texttt{\#pragma omp parallel for} każda iteracja pętli może być wykonana w osobnym wątku. Atrybut \texttt{reduction(+:sum)} zapewnia bezpieczne sumowanie wyników lokalnych wątków do jednej wartości globalnej. OpenMP automatycznie zarządza synchronizacją i agregacją wyników, dzięki czemu użytkownik nie musi implementować ręcznego zarządzania zasobami współdzielonymi.

\subsection{Intel TBB (Threading Building Blocks)}
Intel Threading Building Blocks (TBB) to nowoczesna biblioteka programistyczna dla języka C++, przeznaczona do tworzenia aplikacji równoległych w sposób elastyczny i skalowalny. W~przeciwieństwie do OpenMP, TBB opiera się na programowaniu funkcyjnym i komponentowym, umożliwiając dekompozycję zadań \eng{ task-based parallelism}, a nie operacji niskiego poziomu.

Cechy charakterystyczne:
\begin{itemize}
    \item Deklaratywny styl programowania, który umożliwia oddelegowanie decyzji o wykonaniu do systemu planowania zadań.
    \item Dynamiczna alokacja wątków w oparciu o dostępność zasobów.
    \item Wbudowana obsługa synchronizacji oraz struktur danych przystosowanych do środowisk wielowątkowych (np. \texttt{concurrent\_vector}, \texttt{concurrent\_queue}).
\end{itemize}

\begin{lstlisting}[language=C++, caption={Przykład użycia Intel TBB w C++}, label={lst:TBB_example}]
#include <TBB/TBB.h>
#include <iostream>
#include <vector>

int main() {
    std::vector<int> data(1000, 1);
    int sum = 0;

    TBB::parallel_reduce(
        TBB::blocked_range<size_t>(0, data.size()),
        0,
        [&](const TBB::blocked_range<size_t>& r, int local_sum) {
            for (size_t i = r.begin(); i < r.end(); ++i)
                local_sum += data[i];
            return local_sum;
        },
        std::plus<int>()
    );

    std::cout << "Suma: " << sum << std::endl;
    return 0;
}
\end{lstlisting}
W powyższym przykładzie w listingu \ref{lst:TBB_example} funkcja \texttt{TBB::parallel\_reduce} automatycznie dzieli zakres danych na bloki \texttt{(blocked\_range)}, które przetwarzane są równolegle przez dostępne wątki. Funkcja lambda odpowiada za lokalne przetwarzanie danych (w tym przypadku sumowanie wartości), a następnie lokalne wyniki są agregowane przy użyciu funkcji \texttt{std::plus<int>}. TBB samodzielnie zarządza planowaniem zadań oraz synchronizacją, co czyni go potężnym narzędziem w budowie skalowalnych aplikacji równoległych.

