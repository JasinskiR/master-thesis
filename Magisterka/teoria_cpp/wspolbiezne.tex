\section{Programowanie współbieżne}
Współbieżność w języku C++ wspierana jest od standardu C++11, który wprowadził szereg struktur i mechanizmów umożliwiających tworzenie i synchronizację wątków. W kolejnych wersjach (C++14, C++17, C++20 i C++23) język został wzbogacony o kolejne narzędzia, zwiększające bezpieczeństwo, ekspresyjność i ergonomię programowania współbieżnego.

\subsection{Biblioteki i przestrzeń standardowa}
\subsubsection{\texttt{std::thread} oraz \texttt{std::jthread}}
Standardowa biblioteka języka C++ zawiera podstawowe komponenty do obsługi wątków w~module \texttt{<thread>}. Wraz z C++20 wprowadzono \texttt{std::jthread}, będący bezpieczniejszą alternatywą dla \texttt{std::thread}, ponieważ automatycznie dołącza wątek w destruktorze obiektu. Dzięki temu możliwe jest uniknięcie błędów takich jak niezakończony wątek  \eng{orphaned thread} lub przedwczesne zakończenie programu.

\begin{lstlisting}[language=C++, caption={Przykład użycia std::jthread}, label={jthread_example}]
#include <iostream>
#include <thread>

void worker() {
    std::cout << "Thread is running." << std::endl;
}

int main() {
    std::jthread t(worker); // wątek zarządzany automatycznie
    // brak konieczności wywoływania join() lub detach()
}
\end{lstlisting}    
W przykładzie listing \ref{jthread_example} utworzono nowy wątek wątek z użyciem \texttt{std::jthread}, który uruchamia funkcję worker. Dzięki automatycznemu zarządzaniu zasobami przez \texttt{std::jthread}, nie ma potrzeby ręcznego wywoływania \texttt{join()}, co zmniejsza ryzyko błędów synchronizacji.


\subsection{Komunikacja między wątkami}
C++ nie posiada wbudowanych kanałów \eng{channels} występujących w języku Rust, lecz umożliwia komunikację poprzez konstrukcje takie jak: kolejki, zmienne warunkowe \texttt{(std::condition\_variable)} oraz typy atomowe \texttt{(std::atomic)}. Jednym z najczęstszych wzorców komunikacyjnych jest użycie kolejki chronionej mutexem oraz sygnalizowanej zmienną warunkową.
\begin{lstlisting}[language=C++, caption={Przykład komunikacji między wątkami}, label={condition_variable_example}]
#include <iostream>
#include <thread>
#include <queue>
#include <mutex>
#include <condition_variable>

std::queue<int> buffer;
std::mutex mtx;
std::condition_variable cv;

void producer() {
    std::unique_lock<std::mutex> lock(mtx);
    buffer.push(100); // produkcja danych
    cv.notify_one();  // powiadom konsumenta
}

void consumer() {
    std::unique_lock<std::mutex> lock(mtx);
    cv.wait(lock, [] { return !buffer.empty(); }); // czekaj na dane
    std::cout << "Consumed: " << buffer.front() << std::endl;
    buffer.pop(); // usuń dane z kolejki
}
\end{lstlisting}
Powyższy przykład listing \ref{condition_variable_example} implementuje prosty scenariusz producent-konsument. Producent wstawia dane do kolejki i powiadamia wątek oczekujący. Konsument blokuje się, dopóki kolejka nie będzie zawierać danych. Zmienna warunkowa eliminuje konieczność aktywnego sprawdzania warunku (busy waiting), co poprawia efektywność systemu.


\subsection{Synchronizacja}
Synchronizacja w języku C++ opiera się głównie na mutexach \texttt{(std::mutex)} oraz ich odmianach. Od C++17 dostępny jest \texttt{std::scoped\_lock}, pozwalający na bezpieczne blokowanie wielu mutexów jednocześnie, a od C++20 wprowadzono bardziej zaawansowane konstrukty takie jak \texttt{std::latch} i \texttt{std::barrier}, które umożliwiają synchronizację wielu wątków na określonym etapie wykonania.

\begin{lstlisting}[language=C++, caption={Przykład użycia std::scoped\_lock}, label={scoped_lock_example}]
#include <iostream>
#include <thread>
#include <mutex>

int counter = 0;
std::mutex mtx;

void increment() {
    std::lock_guard<std::mutex> lock(mtx); // automatyczna blokada mutexu
    ++counter;
}

int main() {
    std::thread t1(increment);
    std::thread t2(increment);
    t1.join();
    t2.join();
    std::cout << "Counter: " << counter << std::endl;
}
\end{lstlisting}
W tym przykładzie - listing \ref{scoped_lock_example} dwa wątki próbują jednocześnie zwiększyć wartość zmiennej counter. Aby zapobiec wyścigowi danych, dostęp do zasobu jest chroniony przez \texttt{std::mutex}. Użycie \texttt{std::lock\_guard} zapewnia, że blokada zostanie zwolniona automatycznie po wyjściu z zakresu funkcji.

\subsubsection{\texttt{std::latch} oraz \texttt{std::barrier} (C++20)}
Mechanizmy \texttt{std::latch} oraz \texttt{std::barrier} wprowadzone w standardzie C++20 służą do synchronizacji wielu wątków w określonym punkcie programu:
\begin{itemize}
    \item \texttt{std::latch} - jednorazowy licznik synchronizacyjny, który pozwala wątkom oczekującym na rozpoczęcie działania, aż inne wątki zakończą przygotowanie.
    \item \texttt{std::barrier} - wielokrotnego użytku, synchronizuje grupę wątków po osiągnięciu „cyklu bariery”.
\end{itemize}
\begin{lstlisting}[language=C++, caption={Przykład użycia std::latch oraz std::barrier}, label={latch_barrier_example}]
#include <iostream>
#include <thread>
#include <latch>
#include <barrier>

constexpr int num_threads = 3;
std::latch start_latch(num_threads);
std::barrier sync_barrier(num_threads);

void worker(int id) {
    std::cout << "Thread " << id << " is initializing.\n";
    start_latch.arrive_and_wait(); // czekaj aż wszystkie wątki się przygotują
    for (int i = 0; i < 2; ++i) {
        std::cout << "Thread " << id << " is processing iteration " << i << ".\n";

        // Synchronizacja między iteracjami
        sync_barrier.arrive_and_wait();

        std::cout << "Thread " << id << " passed the barrier in iteration " << i << ".\n";
    }
}
int main() {
    std::thread threads[num_threads];
    for (int i = 0; i < num_threads; ++i)
    threads[i] = std::thread(worker, i + 1);
    for (auto& t : threads)
        t.join();
}
\end{lstlisting}    
W listingu \ref{latch_barrier_example} każdy wątek najpierw dochodzi do punktu synchronizacji \texttt{std::latch}, oczekując aż wszystkie inne wątki również zakończą fazę inicjalizacji. Następnie, w dwóch kolejnych iteracjach przetwarzania danych, zastosowany zostaje \texttt{std::barrier}, który gwarantuje, że wszystkie wątki ukończą daną iterację przed przejściem do kolejnej. Takie podejście zwiększa spójność przetwarzania i eliminuje potencjalne niespójności wynikające z wyścigów czasowych między wątkami.

\subsection{Asynchroniczność}
Programowanie asynchroniczne w C++ możliwe jest dzięki konstrukcjom takim jak \texttt{std::async}, \texttt{std::future} i \texttt{std::promise}. \texttt{std::async} uruchamia funkcję w tle i umożliwia jej obserwację za pomocą obiektu \texttt{future}. Podejście to ułatwia uruchamianie zadań bez konieczności jawnego zarządzania wątkiem.

\begin{lstlisting}[language=C++, caption={Przykład użycia std::async}, label={async_example}]
    #include <iostream>
    #include <future>
    
    int compute() {
        return 2 * 21;
    }
    
    int main() {
        std::future<int> result = std::async(std::launch::async, compute);
        std::cout << "Result: " << result.get() << std::endl;
    }
\end{lstlisting}
W powyższym listingu \ref{async_example} funkcja \texttt{compute()} zostaje uruchomiona asynchronicznie. \texttt{std::future} pozwala na uzyskanie wyniku, gdy ten będzie dostępny. W ten sposób możemy kontynuować inne działania, a wynik odebrać w późniejszym czasie — co jest przydatne w~aplikacjach wymagających wysokiej responsywności.

\subsubsection{std::promise}
Obiekt \texttt{std::promise} (obietnica) w języku C++ umożliwia przekazywanie wartości z jednego wątku do drugiego. Stanowi uzupełnienie mechanizmu \texttt{std::future}, ponieważ pozwala manualnie ustawić wartość, którą futurę później odbierze. Dzięki temu rozdzielona zostaje produkcja i konsumpcja danych między wątkami, umożliwiając bardziej elastyczne projektowanie asynchronicznych przepływów sterowania.
\begin{lstlisting}[language=C++, caption={Przykład użycia std::promise}, label={promise_example}]
#include <iostream>
#include <thread>
#include <future>

// Funkcja symulująca kosztowne obliczenie
int compute(int x) {
    return x * 2;
}

int main() {
    std::promise<int> promise; // utworzenie obiektu obietnicy
    std::future<int> result = promise.get_future(); // pobranie powiązanego future

    std::thread producer([&promise]() {
        int value = 21;
        int result = compute(value);
        promise.set_value(result); // ustawienie wartości, która zostanie odebrana przez future
    });

    std::cout << "Result: " << result.get() << std::endl; // odbiór wartości, blokuje do czasu jej ustawienia
    producer.join();

    return 0;
}
\end{lstlisting}    
W listingu \ref{promise_example} wątek główny tworzy promesę i pobiera powiązaną z nią futurę. Wątek producenta wykonuje obliczenie i przekazuje wynik przez promesę. Funkcja \texttt{result.get()} wstrzymuje główny wątek do czasu dostępności wyniku.
\subsubsection{std::packaged\_task}
\texttt{std::packaged\_task} to obiekt otaczający dowolną wywoływalną funkcję (np. funkcję, lambda, \texttt{std::bind}), który integruje się z future. Pozwala to uruchomić zadanie w wątku i obserwować jego wynik.
\begin{lstlisting}[language=C++, caption={Przykład użycia std::packaged\_task}, label={packaged_task_example}]
#include <iostream>
#include <thread>
#include <future>

// Funkcja do opakowania w packaged_task
int compute(int x) {
    return x * 2;
}

int main() {
    std::packaged_task<int(int)> task(compute); // utworzenie zapakowanego zadania
    std::future<int> result = task.get_future(); // pobranie powiązanego future

    std::thread worker(std::move(task), 21); // uruchomienie zadania w wątku z parametrem 21

    std::cout << "Result: " << result.get() << std::endl; // odbiór wyniku
    worker.join();

    return 0;
}
\end{lstlisting}
W powyższym przykładzie listing \ref{packaged_task_example} funkcja \texttt{compute} została opakowana w \texttt{packaged\_task}, a następnie uruchomiona w osobnym wątku z argumentem 21. Wynik trafia do futury, który umożliwia jego odbiór w wątku głównym. przetwarzanie zadań.

\subsubsection{C++23 - \texttt{std::task} i \texttt{std::execution}}
Standard C++23 (najnowszy w chwili tworzenia niniejszej pracy) wprowadza nowe pojęcia:
\begin{itemize}
    \item \texttt{std::task} - reprezentuje zadanie, które można uruchomić z użyciem określonego planisty wykonania.
    \item \texttt{std::execution} - zestaw polityk (strategii) określających sposób wykonywania zadań, takich jak sekwencyjnie, współbieżnie, równolegle.
\end{itemize}

Mechanizmy te są częścią trwającej transformacji C++ w kierunku deklaratywnego modelu programowania współbieżnego i równoległego. Mają one zostać wprowadzone wstępnie w wersji C++26 \cite{cpp26} Ponieważ wsparcie dla tych mechanizmów jest na etapie wdrażania, nie będą one szczegółowo omawiane w tej pracy. Warto jednak zauważyć, że ich celem jest uproszczenie i~ujednolicenie podejścia do programowania współbieżnego, podobnie jak ma to miejsce w~języku Rust.