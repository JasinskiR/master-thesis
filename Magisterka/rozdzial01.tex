\chapter[Wstęp]{Wstęp}
% \addcontentsline{toc}{chapter}{Wstęp}  % Add unnumbered chapter to the table of contents if needed
\section{Cel oraz zakres pracy}
% \addcontentsline{toc}{chapter}{Cel i zakres pracy}  % Add unnumbered chapter to the table of contents if needed
Celem niniejszej pracy jest przeprowadzenie pogłębionej analizy oraz wszechstronnego porównania mechanizmów programowania współbieżnego i równoległego w dwóch językach programowania: Rust i C++. Celem jest przedstawienie kluczowych różnic oraz podobieństw w podejściu do zarządzania wielowątkowością, analizując jednocześnie efektywność, bezpieczeństwo oraz wygodę stosowania narzędzi dostępnych w obu językach.

W ramach pracy szczególną uwagę poświęcono omówieniu wybranych bibliotek i frameworków, które wspierają tworzenie aplikacji wielowątkowych w Rust (np. Tokio, Rayon) i C++ (np. std::thread, OpenMP, TBB). Przeanalizowane zostaną mechanizmy bezpieczeństwa oraz zarządzania pamięcią i wątkami, które odgrywają kluczową rolę w zapewnieniu stabilności i wydajności aplikacji współbieżnych i równoległych.

Dodatkowym celem jest przeprowadzenie analizy wydajności oraz efektywności implementacji aplikacji wielowątkowych, co pozwoli na ocenę szybkości działania i efektywnego zarządzania zasobami w obu językach. Badanie uwzględni również aspekty praktyczne, takie jak łatwość użycia narzędzi, dostępność wsparcia ze strony społeczności oraz dojrzałość ekosystemu każdego z języków.

Aby zilustrować wyniki teoretyczne w praktyce, przeprowadzona zostanie implementacja aplikacji współbieżnych i równoległych w obu językach, co umożliwi porównanie osiągniętych wyników wydajnościowych oraz analizę różnic w strukturze i stylu kodu. Efektem pracy będzie również identyfikacja scenariuszy, w których jeden z języków może przewyższać drugi pod względem wydajności, bezpieczeństwa, czy wygody stosowania, co pozwoli na sformułowanie rekomendacji dotyczących wyboru języka w zależności od specyficznych wymagań projektowych.
\section{Problem badawczy}
Wraz z rozwojem nowoczesnych technologii informatycznych i rosnącą złożonością systemów obliczeniowych, znaczenia nabierają paradygmaty programowania, które pozwalają na maksymalne wykorzystanie zasobów współczesnego sprzętu komputerowego — w szczególności architektur wielordzeniowych. Programowanie współbieżne i równoległe stanowią obecnie podstawę projektowania wydajnych i niezawodnych aplikacji w wielu obszarach, od systemów operacyjnych, przez serwery wysokiej dostępności, aż po rozwiązania z zakresu sztucznej inteligencji czy gier komputerowych.

W kontekście tych wyzwań szczególnie interesujące staje się porównanie narzędzi, jakie oferują współczesne języki programowania. Niniejsza praca magisterska koncentruje się na analizie dwóch języków: Rust oraz C++, które - mimo odmiennej filozofii projektowej - są powszechnie wykorzystywane w systemach wymagających wysokiej wydajności. Rust, jako stosunkowo młody język \cite{}, zdobywa coraz większą popularność\cite{} ze względu na nowatorskie podejście do bezpieczeństwa pamięci i współbieżności, opierające się na systemie własności (\eng{ownership}) oraz sprawdzaniu poprawności kodu na etapie kompilacji. Dzięki temu minimalizuje ryzyko wycieków pamięci, błędów synchronizacji czy wyścigów danych. Z kolei C++ - język dojrzały, o długiej historii i ugruntowanej pozycji w przemyśle - oferuje niezwykle szeroki wachlarz możliwości, jeśli chodzi o zarządzanie zasobami i niskopoziomową optymalizację, jednak często kosztem większego ryzyka błędów programistycznych.

Wybór tych dwóch języków podyktowany jest ich rosnącym znaczeniem w obszarach wymagających efektywnego zarządzania współbieżnością i równoległością. Rust jest promowany jako bezpieczna alternatywa dla C i C++ w systemach krytycznych \cite{}, natomiast C++ nadal pozostaje filarem wielu aplikacji, w tym tych o kluczowym znaczeniu dla infrastruktury informatycznej. Porównanie ich możliwości w zakresie programowania współbieżnego i równoległego dostarcza cennych informacji dla praktyków inżynierii oprogramowania, projektantów systemów oraz badaczy eksplorujących nowe podejścia do zarządzania złożonością kodu.

W związku z powyższym, głównym problemem badawczym pracy są następujące pytania:
\begin{quote}
    \item \textbf{PB1}: 
    \emph{Jakie są różnice i podobieństwa w podejściu do programowania współbieżnego i równoległego w językach Rust oraz C++ pod względem efektywności, bezpieczeństwa oraz dostępnych narzędzi?}
    \item \textbf{PB2}:
    \emph{W jaki sposób wybór konkretnego języka wpływa na wydajność i stabilność aplikacji współbieżnych oraz równoległych?}
\end{quote}
Odpowiedź na to pytanie zostanie udzielona poprzez analizę teoretyczną, przegląd literatury oraz eksperymentalne porównanie konkretnych mechanizmów oferowanych przez oba języki. W ramach pracy przeprowadzone zostaną testy wydajnościowe oraz analiza, które pozwolą na identyfikację różnic w podejściu do programowania współbieżnego i równoległego w językach Rust oraz C++.

%%Układ dokumentu - Układ tego dokumentu przedstawia się następująco:
\section{Struktura pracy}
Struktura pracy została zaplanowana w sposób umożliwiający systematyczne przedstawienie zagadnienia oraz przeprowadzenie kompleksowej analizy porównawczej. Po niniejszym wprowadzeniu, rozdział drugi precyzuje cel oraz zakres pracy, określając, które aspekty mechanizmów współbieżności i równoległości będą poddane szczegółowej analizie. Następnie, w rozdziale trzecim, przedstawione zostały podstawowe pojęcia związane z programowaniem współbieżnym i równoległym - zarówno od strony teoretycznej, jak i praktycznej - w celu zbudowania wspólnego kontekstu dla dalszych rozważań.

Rozdział czwarty zawiera przegląd literatury oraz wcześniejszych badań dotyczących wykorzystania języków Rust i C++ w projektowaniu systemów wielowątkowych. Zidentyfikowano w nim również istniejące luki badawcze oraz przedstawiono różnice w podejściu do bezpieczeństwa, wydajności i zarządzania pamięcią.

W dalszej części pracy - odpowiednio w rozdziałach piątym i szóstym - zaprezentowano konkretne mechanizmy programowania współbieżnego i równoległego dostępne w językach Rust oraz C++. Każdy z tych rozdziałów zawiera szczegółowe omówienie modeli pamięci, używanych bibliotek (np. Tokio, Rayon, std::thread, OpenMP), metod synchronizacji (mutexy, kanały, wartości atomowe), a także wybranych konstrukcji językowych wspierających bezpieczne współdzielenie danych między wątkami.

W rozdziale siódmym dokonano bezpośredniego porównania omawianych mechanizmów, koncentrując się na takich kryteriach jak zarządzanie wątkami, efektywność synchronizacji, narzut związany z bezpieczeństwem, a także wydajność obliczeniowa i sprzętowa.

Rozdział ósmy poświęcony jest analizie wyników eksperymentów, w ramach których porównano działanie wybranych algorytmów zaimplementowanych w obu językach, ze szczególnym uwzględnieniem czasów wykonania, zużycia zasobów oraz stabilności działania.

W końcowej części pracy, rozdział dziewiąty prezentuje wnioski oraz rekomendacje dotyczące praktycznego zastosowania języków Rust i C++ w projektach wymagających wysokiej wydajności i bezpieczeństwa współbieżnego. Pracę zamyka rozdział dziesiąty, który podsumowuje najważniejsze osiągnięcia badawcze oraz wskazuje możliwe kierunki dalszych analiz i rozwijania zaproponowanych rozwiązań.

\newpage
\section{Słownik wybranych pojęć}

\begin{itemize}
    \item \textbf{licznik Redis} - to mechanizm wykorzystujący bazę danych Redis do przechowywania i aktualizowania liczników w czasie rzeczywistym. Redis, jako szybka baza typu klucz-wartość, pozwala na błyskawiczne operacje inkrementacji i dekrementacji wartości przypisanej do danego klucza.

    \item \textbf{LLVM} - \eng{Low Level Virtual Machine} - to zestaw narzędzi i bibliotek do budowania kompilatorów, który umożliwia generowanie, analizę i optymalizację kodu (zarówno w czasie kompilacji, jak i wykonania). LLVM nie jest maszyną wirtualną w tradycyjnym sensie, ale raczej infrastrukturą kompilatora, która operuje na pośrednim języku reprezentacji (LLVM IR), z którego może generować kod maszynowy dla różnych architektur.

    \item \textbf{nieustraszona współbieżność} - \eng{fearless concurrency} - to podejście do programowania współbieżnego, które eliminuje problemy związane z bezpieczeństwem pamięci i synchronizacją wątków. W Rust osiągnięto to dzięki systemowi własności, który zapewnia, że dane mogą być modyfikowane tylko przez jeden wątek na raz, eliminując ryzyko wyścigów danych i błędów synchronizacji.

    \item \textbf{odwołania do nieobecnych stron} - \eng{page fault} - zdarzenie w systemie operacyjnym, które występuje, gdy program próbuje uzyskać dostęp do strony pamięci, która nie znajduje się obecnie w pamięci RAM. Może to skutkować koniecznością załadowania tej strony z dysku (np. z pliku wymiany), co wpływa na wydajność programu.

    \item \textbf{zwolnienie stron pamięci} - \eng{page reclaims} - operacje systemowe polegające na odzyskiwaniu już załadowanych, ale nieaktywnych stron pamięci, aby umożliwić ich ponowne wykorzystanie przez inne procesy. Pomaga to zoptymalizować wykorzystanie pamięci fizycznej bez konieczności natychmiastowego odwoływania się do pamięci wirtualnej.

    \item \textbf{pożyczanie} \eng{borrow} - również występujący pod inną nazwą jako przenoszenie własności \cite{rustPolishNames}, jest to mechanizm pozwalający na używanie wartości bez przejmowania jej na własność. Dzięki temu możemy przekazywać dane do funkcji lub między częściami programu bez ich kopiowania czy przenoszenia.

    \item \textbf{proces} - to instancja programu, która jest wykonywana w systemie operacyjnym. Procesy są izolowane od siebie i mają własne zasoby, takie jak pamięć i przestrzeń adresowa.

    \item \textbf{programowanie równoległe} - to sposób wykonywania wielu zadań jednocześnie, co zwiększa wydajność programu. W odróżnieniu od programowania współbieżnego, programowanie równoległe polega na wykonywaniu zadań w tym samym czasie, a nie przeplataniu ich w czasie.

    \item \textbf{programowanie współbieżne} - technika programistyczna polegająca na jednoczesnym wykonywaniu wielu zadań lub ich przeplataniu w czasie, mająca na celu zwiększenie efektywności działania programu. Współbieżność może być realizowana z wykorzystaniem wielu wątków, procesów bądź mechanizmów programowania asynchronicznego, które wewnętrznie mogą operować na wątkach lub innych zasobach udostępnianych przez system operacyjny.

    \item \textbf{SIMD} - \eng{Single Instruction, Multiple Data} - pojedyncza instrukcja wykonywana na wielu danych jednocześnie. Jest to technika optymalizacji wydajności obliczeń, która wykorzystuje jednostki wektorowe dostępne w nowoczesnych procesorach.

    \item \textbf{wątek} - część programu wykonywana współbieżnie w obrębie jednego procesu - w jednym procesie może istnieć wiele wątków. Główna różnica między procesem a wątkiem polega na tym, że wszystkie wątki należące do tego samego procesu współdzielą przestrzeń adresową oraz inne zasoby systemowe, takie jak listy otwartych plików czy gniazda sieciowe. Natomiast każdy proces dysponuje własnym, odrębnym zestawem zasobów.

    \item \textbf{własność} \eng{ownership} - system zarządzania pamięcią, który eliminuje konieczność używania automatycznego odśmiecania, jednocześnie zapobiegając błędom takim jak użycie po zwolnieniu czy podwójne zwolnienie.

    \item \textbf{wyścigi danych} \eng{race conditions} - to sytuacja, w której dwa lub więcej wątków lub procesów próbuje modyfikować wspólną zmienną w tym samym czasie, co prowadzi do nieprzewidywalnych wyników.
\end{itemize}

