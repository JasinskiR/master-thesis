\chapter{Podsumowanie}
W ramach niniejszej pracy magisterskiej zrealizowano porównawczą analizę mechanizmów programowania współbieżnego i równoległego w językach Rust i C++ na architekturach ARM64 (Apple Silicon M1) oraz x86\_64 (Intel). Zastosowana metodologia eksperymentalna, wykorzystująca zestawy testowe NAS Parallel Benchmarks oraz implementacje aplikacji współbieżnych, umożliwiła systematyczne porównanie wydajności bibliotek Rust Rayon, Intel TBB oraz OpenMP.

Badania wykazały znaczącą przewagę architektury ARM64 nad x86\_64 we wszystkich scenariuszach testowych, ze współczynnikami przyspieszenia 1,55x-2,26x. Implementacja z~biblioteką Rayon osiągnęła najwyższą wydajność w obliczeniach trywialnie równoległych \mbox{(164 vs 109 MFLOPS)}, podczas gdy Intel TBB dominowała w zadaniach z intensywną komunikacją międzywątkową. Kluczowym odkryciem było wykazanie, że model bezpieczeństwa pamięci w Rust nie generuje istotnych narzutów wydajnościowych, a w niektórych przypadkach przyczynia się do lepszej efektywności. W aplikacjach sieciowych architektura ARM uzyskała przepustowość o rząd wielkości wyższą przy zmniejszonym zużyciu pamięci.

Główne ograniczenia metodologiczne obejmowały heterogeniczność środowisk testowych (macOS vs Linux) oraz ograniczenia narzędzi diagnostycznych na platformie macOS, zwłaszcza w kontekście kontroli przypisań wątków do rdzeni. Dodatkowo, wybrane benchmarki NAS reprezentują problemy o regularnej strukturze obliczeniowej, co ogranicza generalizację wniosków do aplikacji o nieregularnych wzorcach dostępu do pamięci.

Praca wypełnia istotną lukę badawczą, dostarczając kompleksowej analizy porównawczej mechanizmów programowania równoległego w kontekście nowoczesnych architektur sprzętowych. Wyniki potwierdzają porównywalną wydajność Rust względem C++, jednocześnie uwydatniając wpływ architektury na efektywność mechanizmów współbieżności.

W kontekście dalszych badań sugeruje się ujednolicenie środowiska testowego poprzez wykorzystanie maszyn Apple z procesorami Intel i Apple Silicon, rozszerzenie zakresu na architektury heterogeniczne oraz eksplorację nowych paradygmatów programowania, w tym modelu aktorowego i mechanizmów C++20 \texttt{coroutines} w porównaniu z Rust \texttt{async/await}.

Podsumowując, przeprowadzone badania pogłębiają wiedzę na temat charakterystyk wydajnościowych nowoczesnych języków programowania w obliczeniach równoległych oraz wskazują na rosnący potencjał architektury ARM64 w zastosowaniach wysokowydajnościowych. Praca stanowi solidną podstawę do dalszych badań nad optymalizacją mechanizmów współbieżności w systemach wymagających wysokiej wydajności i niezawodności.

%\show\chapter
%\show\section
%\show\subsection

%\showthe\secindent
%\showthe\beforesecskip
%\showthe\aftersecskip
%\showthe\secheadstyle
%\showthe\subsecindent
%\showthe\beforesubsecskip
%\showthe\aftersubsecskip
%\showthe\subseccheadstyle
%\showthe\parskip
