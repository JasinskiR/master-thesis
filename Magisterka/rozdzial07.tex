%Porównanie Rust i C++
% \chapter{Porównanie mechanizmów w językach Rust oraz C++}
% \section{Programowanie współbieżne}
% Thread Creation and Management
% Thread creation time (microseconds)
% Memory overhead per thread (KB)
% Maximum number of threads before system failure
% Synchronization Performance
% Mutex lock/unlock operations per second
% Channel/message passing latency (microseconds)
% Context switching time between threads (microseconds)
% Safety Overhead
% Compilation time with concurrent code (seconds)
% Binary size comparison (KB)
% Memory usage during runtime (MB)

% \section{Programowanie równoległe}

% Computation Performance
% Execution time for parallel algorithms (milliseconds)
% Speedup ratio (T1/Tn) with different thread counts
% Efficiency (Speedup/Number of processors)
% GFLOP/s performance
% - Single-threaded GFLOP/s
% - Multi-threaded GFLOP/s
% - Scaling efficiency (GFLOP/s per core)
% Amdahl's law comparison
% Resource Usage
% CPU utilization percentage
% Memory consumption under load
% Cache miss rates
% I/O operations per second

% GFLOP/s is particularly useful:\\

% Provides a standardized way to measure computational performance
% Allows direct comparison of algorithmic efficiency
% Helps evaluate hardware utilization effectiveness
% To measure GFLOP/s:\\

% Matrix multiplication
% FFT (Fast Fourier Transform)
% N-body simulation

\chapter{Porównanie mechanizmów w językach Rust oraz C++}

\section{Programowanie współbieżne}
Programowanie współbieżne odnosi się do organizacji zadań w sposób umożliwiający ich jednoczesne wykonywanie, przy czym zadania te mogą dzielić wspólne zasoby. W niniejszej sekcji przeprowadzono porównanie mechanizmów współbieżności w językach Rust oraz C++, koncentrując się na kluczowych aspektach takich jak zarządzanie wątkami, synchronizacja oraz wydajność.

\subsection{Zarządzanie wątkami}
Porównanie tworzenia i zarządzania wątkami obejmuje następujące parametry:
\begin{itemize}
\item czas tworzenia wątku (w mikrosekundach),
\item narzut pamięciowy na wątek (w KB),
\item maksymalna liczba wątków przed awarią systemu.
\end{itemize}

\subsection{Wydajność synchronizacji}
Kluczowe aspekty synchronizacji to:
\begin{itemize}
\item liczba operacji blokowania i odblokowywania mutexa na sekundę,
\item opóźnienie przesyłania wiadomości przez kanały (w mikrosekundach),
\item czas przełączania kontekstu pomiędzy wątkami (w mikrosekundach).
\end{itemize}

\subsection{Narzut bezpieczeństwa}
W kontekście narzutu związanego z mechanizmami bezpieczeństwa w językach Rust i C++ analizowane będą następujące metryki:
\begin{itemize}
\item czas kompilacji kodu współbieżnego (w sekundach),
\item rozmiar pliku binarnego (w KB),
\item zużycie pamięci podczas wykonywania programu (w MB).
\end{itemize}

\subsection{Wybrane algorytmy do analizy wydajności}
Dla porównania mechanizmów współbieżności wybrano następujące algorytmy:
\begin{itemize}
\item Mnożenie macierzy (Matrix Multiplication),
\item Transformata Fouriera (Fast Fourier Transform, FFT).
\end{itemize}
Algorytmy te pozwalają na analizę zdolności języków Rust i C++ do efektywnego zarządzania współbieżnością w obliczeniach numerycznych.

\section{Programowanie równoległe}
Programowanie równoległe odnosi się do wykonywania zadań na wielu rdzeniach procesora jednocześnie, co pozwala na istotne przyspieszenie obliczeń. Analiza porównawcza języków Rust oraz C++ w zakresie równoległości obejmuje następujące aspekty:

\subsection{Wydajność obliczeniowa}
Główne metryki oceny wydajności algorytmów równoległych to:
\begin{itemize}
\item czas wykonania algorytmów równoległych (w milisekundach),
\item współczynnik przyspieszenia (T1/Tn) dla różnych liczby wątków,
\item efektywność (przyspieszenie/liczba procesorów).
\end{itemize}

\subsection{Wydajność sprzętowa (GFLOP/s)}
Wydajność obliczeniowa mierzona w jednostkach GFLOP/s (gigaflops per second) pozwala na ocenę efektywności wykorzystania sprzętu:
\begin{itemize}
\item wydajność w pojedynczym wątku,
\item wydajność wielowątkowa,
\item efektywność skalowania (GFLOP/s na rdzeń).
\end{itemize}
Dodatkowo przeprowadzona zostanie analiza zgodnie z prawem Amdahla w celu określenia teoretycznych ograniczeń przyspieszenia obliczeń.

\subsection{Zasoby systemowe}
Analiza zużycia zasobów przez algorytmy równoległe obejmuje:
\begin{itemize}
\item procentowe wykorzystanie CPU,
\item zużycie pamięci w warunkach obciążenia,
\item współczynnik nietrafień w cache,
\item liczbę operacji wejścia-wyjścia na sekundę.
\end{itemize}

\subsection{Wybrane algorytmy do analizy wydajności}
Dla porównania mechanizmów równoległości wybrano następujące algorytmy:
\begin{itemize}
\item Model producent-konsument (z wykorzystaniem aktorów),
\item Problem filozofów (synchronizacja dostępu do zasobów).
\end{itemize}
Algorytmy te pozwolą na analizę efektywności zarządzania zasobami oraz synchronizacji w językach Rust i C++.

Wybór powyższych benchmarków pozwoli na szczegółową analizę wydajności oraz stabilności obu języków w kontekście programowania współbieżnego i równoległego, umożliwiając sformułowanie rekomendacji dotyczących wyboru narzędzi w zależności od specyfiki projektu.