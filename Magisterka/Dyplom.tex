% !TeX root = Dyplom.tex
% !TeX root = Dyplom.tex
\documentclass[a4paper,onecolumn,oneside,12pt,extrafontsizes]{memoir}

%    Do tego wystarczy posłużyć się poniższymi komendami (zamiast documentclass z pierwszej linijki):
%   \documentclass[a4paper,onecolumn,twoside,10pt]{memoir} 
%   \renewcommand{\normalsize}{\fontsize{8pt}{10pt}\selectfont}
\usepackage[utf8]{inputenc} % Proszę użyć zamiast powyższego, jeśli kodowanie edytowanych plików to UTF8
\usepackage[T1]{fontenc}
\usepackage[english,polish]{babel} % Tutaj ważna jest kolejność atrybutów (dla pracy po polsku polish powinno być na końcu)
%\DisemulatePackage{setspace}
\usepackage{setspace}
\usepackage{color,calc}
%\usepackage{soul} % pakiet z komendami do podkreślania, przekreślania, podświetlania tekstu (raczej niepotrzebny)
\usepackage{ebgaramond} % pakiet z czcionkami garamond, potrzebny tylko do strony tytułowej, musi wystąpić przed pakietem tgtermes
\usepackage{float}
\usepackage{etoolbox}

\usepackage{subcaption}
%% Aby uzyskać polskie literki w pdfie (a nie zlepki) korzystamy z pakietu czcionek tgterms. 
%% W pakiecie tym są zdefiniowane klony czcionek Times o kształtach: normalny, pogrubiony, italic, italic pogrubiony.
%% W pakiecie tym brakuje czcionki o kształcie: slanted (podobny do italic). 
%% Jeśli w dokumencie gdzieś zostanie zastosowana czcionka slanted (np. po użyciu komendy \textsl{}), to
%% latex dokona podstawienia na czcionkę standardową i zgłosi to w ostrzeżeniu (warningu).
%% Ponadto tgtermes to czcionka do tekstu. Wszelkie matematyczne wzory będą sformatowane domyślną czcionką do wzorów.
%% Jeśli wzory mają być sformatowane z wykorzystaniem innych czcionek, trzeba to jawnie zadeklarować.

\usepackage{tgtermes}   
\renewcommand*\ttdefault{txtt}


%%%%%%%%%%%%%%%%%%%%%%%%%%%%%%%%%%%%%%%%%%%%%%%%%%%%%%%%%%%%%%%%%%%%%%%%%%%%%%%%
%% Ustawienia odpowiedzialne za sposób łamania dokumentu
%% i ułożenie elementów pływających
%%%%%%%%%%%%%%%%%%%%%%%%%%%%%%%%%%%%%%%%%%%%%%%%%%%%%%%%%%%%%%%%%%%%%%%%%%%%%%%%
\clubpenalty=10000      % kara za sierotki
\widowpenalty=10000     % nie pozostawiaj wdów
\righthyphenmin=3			  % dziel minimum 3 litery

\renewcommand{\topfraction}{0.95}
\renewcommand{\bottomfraction}{0.95}
\renewcommand{\textfraction}{0.05}
\renewcommand{\floatpagefraction}{0.35}

%%%%%%%%%%%%%%%%%%%%%%%%%%%%%%%%%%%%%%%%%%%%%%%%%%%%%%%%%%%%%%%%%%%%%%%%%%%%%%%%
%%  Ustawienia rozmiarów: tekstu, nagłówka i stopki, marginesów
%%  dla dokumentów klasy memoir 
%%%%%%%%%%%%%%%%%%%%%%%%%%%%%%%%%%%%%%%%%%%%%%%%%%%%%%%%%%%%%%%%%%%%%%%%%%%%%%%%
\setlength{\headsep}{3mm} 
\setlength{\headheight}{13.6pt} % wartość baselineskip dla czcionki 11pt tj. \small wynosi 13.6pt
\setlength{\footskip}{\headsep+\headheight}
\setlength{\uppermargin}{\headheight+\headsep+1cm}
\setlength{\textheight}{\paperheight-\uppermargin-\footskip-1.5cm}
\setlength{\textwidth}{\paperwidth-5cm}
\setlength{\spinemargin}{2.5cm}
\setlength{\foremargin}{2.5cm}
\setlength{\marginparsep}{2mm}
\setlength{\marginparwidth}{2.3mm}
\checkandfixthelayout[fixed] % konieczne, aby się dobrze wszystko poustawiało
%%%%%%%%%%%%%%%%%%%%%%%%%%%%%%%%%%%%%%%%%%%%%%%%%%%%%%%%%%%%%%%%%%%%%%%%%%%%%%%%
%%  Ustawienia odległości linii, wcięć, odstępów
%%%%%%%%%%%%%%%%%%%%%%%%%%%%%%%%%%%%%%%%%%%%%%%%%%%%%%%%%%%%%%%%%%%%%%%%%%%%%%%%
\linespread{1}
%\linespread{1.241}
\setlength{\parindent}{14.5pt}


\usepackage{multicol} % pakiet umożliwiający stworzenie wielokolumnowego tekstu
%%%%%%%%%%%%%%%%%%%%%%%%%%%%%%%%%%%%%%%%%%%%%%%%%%%%%%%%%%%%%%%%%%%%%%%%%%%%%%%%
%% Pakiety do formatowania tabel
%%%%%%%%%%%%%%%%%%%%%%%%%%%%%%%%%%%%%%%%%%%%%%%%%%%%%%%%%%%%%%%%%%%%%%%%%%%%%%%%
\usepackage{tabularx}

%%%%%%%%%%%%%%%%%%%%%%%%%%%%%%%%%%%%%%%%%%%%%%%%%%%%%%%%%%%%%%%%%%%%%%%%%%%%%%%%
%% Pakiet do wstawiania fragmentów kodu
%%%%%%%%%%%%%%%%%%%%%%%%%%%%%%%%%%%%%%%%%%%%%%%%%%%%%%%%%%%%%%%%%%%%%%%%%%%%%%%%
\usepackage{listings} 
\usepackage{listingsutf8}
\usepackage[linesnumbered,ruled,vlined]{algorithm2e}
% Zmiana "Algorithm" na "Algorytm"
\renewcommand{\algorithmcfname}{Algorytm}
\usepackage{framed}
\usepackage{xpatch}
\makeatletter
\xpatchcmd\l@lstlisting{1.5em}{0em}{}{}
\makeatother
% Pakiet dostarcza otoczenia lstlisting. Jest ono wysoce konfigurowalne. 
% Konfigurować można indywidualnie każdy z listingów lub globalnie, w poleceniu \lstset{}.

% Zalecane jest, by kod źródłowy był wyprowadzany z użyciem czcionki maszynowej \ttfamily
% Ponieważ kod źródłowy, nawet po obcięciu do interesujących fragmentów, bywa obszerny, należy zmniejszyć czcionkę.
% Zalecane jest \small (dla krótkich fragmentów) oraz \footnotesize (dla dłuższych fragmentów).

% Ponadto podczas konfiguracji można zadeklarować sposób numerowania linii. Numerowanie linii zalecane jest jednak 
% tylko w przypadkach, gdy w redagowanym tekście znajdują się jakieś odwołania do konkretnych linii.
% Jeśli takich odwołań nie ma, numerowanie linii jest zbędne. Proszę wtedy go nie stosować.
% Przy włączaniu numerowania linii należy zwrócić uwagę na to, gdzie pojawią się te numery.
% Bez zmiany dodatkowych parametrów pojawiają się one na marginesie strony (co jest niepożądane).

\lstset{
  basicstyle=\small\ttfamily, % lub basicstyle=\footnotesize\ttfamily
  inputencoding=utf8,
  %%columns=fullflexible,
	%%showstringspaces=false,
	%%showspaces=false,
  breaklines=true,
  postbreak=\mbox{\textcolor{red}{$\hookrightarrow$}\space}, 
  %%numbers=left,  % ta i poniższe linie dotyczą ustawienia numerowania i sposobu jego wyprowadzania
  %%firstnumber=1, 
  %%numberfirstline=true, 
	%%xleftmargin=17pt,
  %%framexleftmargin=17pt,
  %%framexrightmargin=5pt,
  %%framexbottommargin=4pt,
	belowskip=.5\baselineskip,
	literate={\_}{{\_\allowbreak}}1 % ta deklaracja przydaje się, jeśli na listingu mają być łamane nazwy zawierające podkreślniki
}
\lstset{literate=%-
{ą}{{\k{a}}}1 {ć}{{\'c}}1 {ę}{{\k{e}}}1 {ł}{{\l{}}}1 {ń}{{\'n}}1 {ó}{{\'o}}1 {ś}{{\'s}}1 {ż}{{\.z}}1 {ź}{{\'z}}1 {Ą}{{\k{A}}}1 {Ć}{{\'C}}1 {Ę}{{\k{E}}}1 {Ł}{{\L{}}}1 {Ń}{{\'N}}1 {Ó}{{\'O}}1 {Ś}{{\'S}}1 {Ż}{{\.Z}}1 {Ź}{{\'Z}}1 
    {Ö}{{\"O}}1
    {Ä}{{\"A}}1
    {Ü}{{\"U}}1
    {ß}{{\ss}}1
    {ü}{{\"u}}1
    {ä}{{\"a}}1
    {ö}{{\"o}}1
    {~}{{\textasciitilde}}1
    {—}{{{\textemdash} }}1
}%{\ \ }{{\ }}1}

%% Inne języki muszą być dodefiniowane. Poniżej podano przykłady definicji języków i styli.

\definecolor{lightgray}{rgb}{.9,.9,.9}
\definecolor{darkgray}{rgb}{.4,.4,.4}
\definecolor{purple}{rgb}{0.65, 0.12, 0.82}
\definecolor{javared}{rgb}{0.6,0,0} % for strings
\definecolor{javagreen}{rgb}{0.25,0.5,0.35} % comments
\definecolor{javapurple}{rgb}{0.5,0,0.35} % keywords
\definecolor{javadocblue}{rgb}{0.25,0.35,0.75} % javadoc


\lstdefinestyle{JavaStyle}{
basicstyle=\footnotesize\ttfamily,
keywordstyle=\color{javapurple}\bfseries,
stringstyle=\color{javared},
commentstyle=\color{javagreen},
morecomment=[s][\color{javadocblue}]{/**}{*/},
numbers=none,%left,
numberstyle=\tiny\color{black},
stepnumber=2,
numbersep=10pt,
tabsize=4,
showspaces=false,
showstringspaces=false,
captionpos=t
}

\definecolor{BackgroundRust}{RGB}{255,255,255}
\definecolor{GrayCodeBlock}{RGB}{241,241,241}
\definecolor{BlackText}{RGB}{110,107,94}
\definecolor{RedTypename}{RGB}{182,86,17}
\definecolor{GreenString}{RGB}{96,172,57}
\definecolor{PurpleKeyword}{RGB}{184,84,212}
\definecolor{GrayComment}{RGB}{170,170,170}
\definecolor{GoldDocumentation}{RGB}{180,165,45}
\lstdefinelanguage{rust}
{
    columns=fullflexible,
    keepspaces=true,
    frame=single,
    framesep=0pt,
    framerule=0pt,
    framexleftmargin=4pt,
    framexrightmargin=4pt,
    framextopmargin=5pt,
    framexbottommargin=3pt,
    xleftmargin=4pt,
    xrightmargin=4pt,
    backgroundcolor=\color{BackgroundRust},
    basicstyle=\footnotesize\ttfamily\color{BlackText},
    keywords={
        true,false,
        unsafe,async,await,move,
        use,pub,crate,super,self,mod,
        struct,enum,fn,const,static,let,mut,ref,type,impl,dyn,trait,where,as,
        break,continue,if,else,while,for,loop,match,return,yield,in
    },
    keywordstyle=\color{PurpleKeyword},
    ndkeywords={
        bool,u8,u16,u32,u64,u128,i8,i16,i32,i64,i128,char,str,
        Self,Option,Some,None,Result,Ok,Err,String,Box,Vec,Rc,Arc,Cell,RefCell,HashMap,BTreeMap,
        macro_rules
    },
    ndkeywordstyle=\color{RedTypename},
    comment=[l][\color{GrayComment}\slshape]{//},
    morecomment=[s][\color{GrayComment}\slshape]{/*}{*/},
    morecomment=[l][\color{GoldDocumentation}\slshape]{///},
    morecomment=[s][\color{GoldDocumentation}\slshape]{/*!}{*/},
    morecomment=[l][\color{GoldDocumentation}\slshape]{//!},
    morecomment=[s][\color{RedTypename}]{\#![}{]},
    morecomment=[s][\color{RedTypename}]{\#[}{]},
    stringstyle=\color{GreenString},
    string=[b]"
}

\definecolor{delim}{RGB}{20,105,176}
\definecolor{numb}{RGB}{106, 109, 32}
\definecolor{string}{rgb}{0.64,0.08,0.08}
\lstdefinelanguage{json}{
    showspaces=false,
    showtabs=false,
    breaklines=true,
    postbreak=\raisebox{0ex}[0ex][0ex]{\ensuremath{\color{GrayCodeBlock}\hookrightarrow\space}},
    breakatwhitespace=true,
    basicstyle=\ttfamily\footnotesize,
    upquote=true,
    morestring=[b]",
    stringstyle=\color{string},
    literate=
     *{0}{{{\color{numb}0}}}{1}
      {1}{{{\color{numb}1}}}{1}
      {2}{{{\color{numb}2}}}{1}
      {3}{{{\color{numb}3}}}{1}
      {4}{{{\color{numb}4}}}{1}
      {5}{{{\color{numb}5}}}{1}
      {6}{{{\color{numb}6}}}{1}
      {7}{{{\color{numb}7}}}{1}
      {8}{{{\color{numb}8}}}{1}
      {9}{{{\color{numb}9}}}{1}
      {\{}{{{\color{delim}{\{}}}}{1}
      {\}}{{{\color{delim}{\}}}}}{1}
      {[}{{{\color{delim}{[}}}}{1}
      {]}{{{\color{delim}{]}}}}{1},
}

\definecolor{pblue}{rgb}{0.13,0.13,1}
\definecolor{pgreen}{rgb}{0,0.5,0}
\definecolor{pred}{rgb}{0.9,0,0}
\definecolor{pgrey}{rgb}{0.46,0.45,0.48}
\definecolor{dark-grey}{rgb}{0.4,0.4,0.4}

% VS2017 C++ color scheme
\definecolor{clr-background}{RGB}{255,255,255}
\definecolor{clr-text}{RGB}{0,0,0}
\definecolor{clr-string}{RGB}{163,21,21}
\definecolor{clr-namespace}{RGB}{0,0,0}
\definecolor{clr-preprocessor}{RGB}{128,128,128}
\definecolor{clr-keyword}{RGB}{0,0,255}
\definecolor{clr-type}{RGB}{43,145,175}
\definecolor{clr-variable}{RGB}{0,0,0}
\definecolor{clr-constant}{RGB}{111,0,138} % macro color
\definecolor{clr-comment}{RGB}{0,128,0}

\lstdefinestyle{VS2017}{
	backgroundcolor=\color{clr-background},
	basicstyle=\color{clr-text}, % any text
	stringstyle=\color{clr-string},
	identifierstyle=\color{clr-variable}, % just about anything that isn't a directive, comment, string or known type
	commentstyle=\color{clr-comment},
	directivestyle=\color{clr-preprocessor}, % preprocessor commands
	% listings doesn't differentiate between types and keywords (e.g. int vs return)
	% use the user types color
	keywordstyle=\color{clr-type},
	keywordstyle={[2]\color{clr-constant}}, % you'll need to define these or use a custom language
	tabsize=4,
	basicstyle=\ttfamily\footnotesize,
}
% styl json
\newcommand\JSONnumbervaluestyle{\color{blue}}
\newcommand\JSONstringvaluestyle{\color{red}}

\newif\ifcolonfoundonthisline

\makeatletter

\lstdefinestyle{json-style}  
{
	showstringspaces    = false,
	keywords            = {false,true},
	alsoletter          = 0123456789.,
	morestring          = [s]{"}{"},
	stringstyle         = \ifcolonfoundonthisline\JSONstringvaluestyle\fi,
	MoreSelectCharTable =%
	\lst@DefSaveDef{`:}\colon@json{\processColon@json},
	basicstyle          = \footnotesize\ttfamily,
	keywordstyle        = \ttfamily\bfseries,
	numbers				= left, % zakomentować, jeśli numeracja linii jest niepotrzebna
	numberstyle={\footnotesize\ttfamily\color{dark-grey}},
	xleftmargin			= 2em % zakomentować, jeśli numeracja linii jest niepotrzebna
}

\newcommand\processColon@json{%
	\colon@json%
	\ifnum\lst@mode=\lst@Pmode%
	\global\colonfoundonthislinetrue%
	\fi
}

\lst@AddToHook{Output}{%
	\ifcolonfoundonthisline%
	\ifnum\lst@mode=\lst@Pmode%
	\def\lst@thestyle{\JSONnumbervaluestyle}%
	\fi
	\fi
	\lsthk@DetectKeywords% 
}

\lst@AddToHook{EOL}%
{\global\colonfoundonthislinefalse}

\makeatother
\usepackage{memlays}     % extra layout diagrams, zastosowane w szblonie do 'debuggowania', używa pakietu layouts
%\usepackage{layouts}
\usepackage{printlen} % pakiet do wyświetlania wartości zdefiniowanych długości, stosowany do 'debuggowania'
\usepackage{enumitem} % pakiet do numerowania 1.1 1.2 w sekcji enumrate
\uselengthunit{pt}
\makeatletter
\newcommand{\showFontSize}{\f@size pt} % makro wypisujące wielkość bieżącej czcionki
\makeatother
% do pokazania ramek można byłoby użyć:
%\usepackage{showframe} 

%%%%%%%%%%%%%%%%%%%%%%%%%%%%%%%%%%%%%%%%%%%%%%%%%%%%%%%%%%%%%%%%%%%%%%%%%%%%%%%%
%%  Formatowanie list wyliczeniowych, wypunktowań i własnych otoczeń
%%%%%%%%%%%%%%%%%%%%%%%%%%%%%%%%%%%%%%%%%%%%%%%%%%%%%%%%%%%%%%%%%%%%%%%%%%%%%%%%

% Domyślnie wypunktowania mają zadeklarowane znaki, które nie występują w tgtermes
% Aby latex nie podstawiał w ich miejsca znaków z czcionki standardowej można zrobić podstawienie:
%    \DeclareTextCommandDefault{\textbullet}{\ensuremath{\bullet}}
%    \DeclareTextCommandDefault{\textasteriskcentered}{\ensuremath{\ast}}
%    \DeclareTextCommandDefault{\textperiodcentered}{\ensuremath{\cdot}}
% Jednak jeszcze lepszym pomysłem jest zdefiniowanie otoczeń z wykorzystaniem enumitem
\usepackage{enumitem} % pakiet pozwalający zarządzać formatowaniem list wyliczeniowych
\setlist{noitemsep,topsep=4pt,parsep=0pt,partopsep=4pt,leftmargin=*} % zadeklarowane parametry pozwalają uzyskać 'zwartą' postać wypunktowania bądź wyliczenia
\setenumerate{labelindent=0pt,itemindent=0pt,leftmargin=!,label=\arabic*.} % można zmienić \arabic na \alph, jeśli wyliczenia mają być z literkami
\setlistdepth{4} % definiujemy głębokość zagnieżdżenia list wyliczeniowych do 4 poziomów
\setlist[itemize,1]{label=$\bullet$}  % definiujemy, jaki symbol ma być użyty w wyliczeniu na danym poziomie
\setlist[itemize,2]{label=\normalfont\bfseries\textendash}
\setlist[itemize,3]{label=$\ast$}
\setlist[itemize,4]{label=$\cdot$}
\renewlist{itemize}{itemize}{4}


\makeatletter
\renewenvironment{quote}{
	\begin{list}{}
	{
	\setlength{\leftmargin}{1em}
	\setlength{\topsep}{0pt}%
	\setlength{\partopsep}{0pt}%
	\setlength{\parskip}{0pt}%
	\setlength{\parsep}{0pt}%
	\setlength{\itemsep}{0pt}
	}
	}{
	\end{list}}
\makeatother

 \usepackage{datetime2} % INFO: pakiet potrzeby do uzyskania i sformatowania daty 
 \usepackage[pdftex,bookmarks,breaklinks,unicode]{hyperref}
 \usepackage[pdftex]{graphicx}
 \DeclareGraphicsExtensions{.pdf,.jpg,.mps,.png} % po zadeklarowaniu rozszerzeń można będzie wstawiać pliki z grafiką bez konieczności podawania tych rozszerzeń w ich nazwach
\pdfcompresslevel=9
\pdfoutput=1

% Dobrze przygotowany dokument pdf to taki, który zawiera metadane.
% Poniżej zadeklarowano pola metadanych, jakie będą włączone do dokumentu pdf.
% Można je zmodyfikować w zależności od potrzeb
\makeatletter
\pdftrailerid{} %Remove ID
\pdfsuppressptexinfo15 %Suppress PTEX.Fullbanner and info of imported PDFs
\makeatother
%\else             % jeśli kompilacja jest inna niż pdflatex
\usepackage{graphicx}
\DeclareGraphicsExtensions{.eps,.ps,.jpg,.mps,.png}
%\fi
\sloppy

% INFO: dodane by lepiej łamać urle 
\def\UrlBreaks{\do\/\do-\do_} 


%%%%%%%%%%%%%%%%%%%%%%%%%%%%%%%%%%%%%%%%%%%%%%%%%%%%%%%%%%%%%%%%%%%%%%%%%%%%%%%%
%%  Formatowanie dokumentu
%%%%%%%%%%%%%%%%%%%%%%%%%%%%%%%%%%%%%%%%%%%%%%%%%%%%%%%%%%%%%%%%%%%%%%%%%%%%%%%%
% INFO: Deklaracja głębokościu numeracji
\setcounter{secnumdepth}{2}
\setcounter{tocdepth}{2}
\setsecnumdepth{subsection} 
% INFO: Dodanie kropek po numerach sekcji
\makeatletter
\def\@seccntformat#1{\csname the#1\endcsname.\quad}
\def\numberline#1{\hb@xt@\@tempdima{#1\if&#1&\else.\fi\hfil}}
\makeatother
% INFO: Numeracja rozdziałów i separatory
\renewcommand{\chapternumberline}[1]{#1.\quad}
\renewcommand{\cftchapterdotsep}{\cftdotsep}

\makeatletter % odstępy w spisie pomiędzy rozdziałami
\renewcommand*{\insertchapterspace}{%
  \addtocontents{lof}{\protect\addvspace{3pt}}%
  \addtocontents{lot}{\protect\addvspace{3pt}}%
	\addtocontents{toc}{\protect\addvspace{3pt}} %
  \addtocontents{lol}{\protect\addvspace{3pt}}}
\makeatother 


\setlength{\cftbeforechapterskip}{0pt} % odstępy w spisie treści przed rozdziałem, działa w korelacji z:
\renewcommand{\aftertoctitle}{\afterchaptertitle\vspace{-4pt}} % 

% INFO: Czcionka do podpisów tabel, rysunków, listingów
\captionnamefont{\small}
\captiontitlefont{\small}

% INFO: Sformatowanie podpisu nad dwukolumnowym listingiem
\newcommand{\listingcaption}[1]
{%
\vspace*{\abovecaptionskip}\small 
\refstepcounter{lstlisting}\hfill%
Listing \thelstlisting: #1\hfill%\hfill%
\addcontentsline{lol}{lstlisting}{\protect\numberline{\thelstlisting}#1}
}%

% INFO: Pomocnicze marko do wyróżniania tekstu w języku angielskim
\newcommand{\eng}[1]{(ang.~\emph{#1})}
% IFNO: Pomocnicze makro do dołączania podpisów do rysunków ze wskazaniem źródła (bez wypisywania tego źródła w spisie rysunków)
\newcommand*{\captionsource}[2]{%
  \caption[{#1}]{%
    #1 \emph{Źródło:} #2%
  }%
}

% INFO: definicje etykiet i tytułów spisów

%\AtBeginDocument{% 
        \addto\captionspolish{% 
        \renewcommand{\tablename}{Tabela}%% INFO: Przedefiniowanie etykiet w podpisach tabel 
}%} 

% Przedefiniowanie etykiet oraz nazw wykazu literatury, spisów, indeksu
%\AtBeginDocument{% 
        \addto\captionspolish{% 
        \renewcommand{\figurename}{Rys.}%% INFO: Przedefiniowanie etykiet w podpisach rysunków 
}%}

%\AtBeginDocument{% 
        \addto\captionspolish{% 
        \renewcommand{\lstlistlistingname}{Spis listingów}%% INFO: Przedefiniowanie nazwy spisu listingów
}%} 
\newlistof{lstlistoflistings}{lol}{\lstlistlistingname}


%\AtBeginDocument{% 
        \addto\captionspolish{% 
        \renewcommand{\bibname}{Bibliografia}%% INFO: Przedefiniowanie nazwy wykazu literatury 
}%}

%\AtBeginDocument{% 
        \addto\captionspolish{% 
        \renewcommand{\listfigurename}{Spis rysunków}%% INFO: Przedefiniowanie nazwy spisu rysunków 
}%}

%\AtBeginDocument{% 
        \addto\captionspolish{% 
        \renewcommand{\listtablename}{Spis tabel}%% INFO: Przedefiniowanie nazwy spisu tabel 
}%}

%\AtBeginDocument{% 
        \addto\captionspolish{% 
\renewcommand\indexname{Indeks rzeczowy}%% INFO: Przedefiniowanie nazwy indeksu 
}%}

  \makeatletter
  % \renewcommand{\ALG@name}{Algorytm}
  \makeatother

%\AtBeginDocument{% 
%    \addto\captionsenglish{
%\renewcommand\abstractname{Abstract} 
%}%}

\renewcommand{\abstractnamefont}{\normalfont\Large\bfseries}
\renewcommand{\abstracttextfont}{\normalfont}


%%%%%%%%%%%%%%%%%%%%%%%%%%%%%%%%%%%%%%%%%%%%%%%%%%%%%%%%%%%%%%%%%%%%%%%%%%%%%%%%
%% Definicje stopek i nagłówków
%%%%%%%%%%%%%%%%%%%%%%%%%%%%%%%%%%%%%%%%%%%%%%%%%%%%%%%%%%%%%%%%%%%%%%%%%%%%%%%%
\addtopsmarks{headings}{%
\nouppercaseheads % added at the beginning
}{%
\createmark{chapter}{both}{shownumber}{}{. \space}
%\createmark{chapter}{left}{shownumber}{}{. \space}
\createmark{section}{right}{shownumber}{}{. \space}
}%use the new settings

\makeatletter
\copypagestyle{outer}{headings}
\makeoddhead{outer}{}{}{\small\itshape\rightmark}
\makeevenhead{outer}{\small\itshape\leftmark}{}{}
\makeoddfoot{outer}{\small\@author:~\@titleShort}{}{\small\thepage}
\makeevenfoot{outer}{\small\thepage}{}{\small\@author:~\@title}
\makeheadrule{outer}{\linewidth}{\normalrulethickness}
\makefootrule{outer}{\linewidth}{\normalrulethickness}{2pt}
\makeatother

% fix plain
\copypagestyle{plain}{headings} % overwrite plain with outer
\makeoddhead{plain}{}{}{} % remove right header
\makeevenhead{plain}{}{}{} % remove left header
\makeevenfoot{plain}{}{}{}
\makeoddfoot{plain}{}{}{}

\copypagestyle{empty}{headings} % overwrite plain with outer
\makeoddhead{empty}{}{}{} % remove right header
\makeevenhead{empty}{}{}{} % remove left header
\makeevenfoot{empty}{}{}{}
\makeoddfoot{empty}{}{}{}

% INFO: deklaracja zmiennej logicznej wykorzystywanej do rozróżnienia pracy inżynierskiej i magisterskiej
\newif\ifMaster% domyślnie false (czyli domyślnie mamy pracę inżynierską)

%%%%%%%%%%%%%%%%%%%%%%%%%%%%%%%%%%%%%%%%%%%%%%%%%%%%%%%%%%%%%%%%%%%%%%%%%%%%%%%%
%% Definicja strony tytułowej 
%%%%%%%%%%%%%%%%%%%%%%%%%%%%%%%%%%%%%%%%%%%%%%%%%%%%%%%%%%%%%%%%%%%%%%%%%%%%%%%%
\makeatletter
%Uczelnia
\newcommand\uczelnia[1]{\renewcommand\@uczelnia{#1}}
\newcommand\@uczelnia{}
%Wydział
\newcommand\wydzial[1]{\renewcommand\@wydzial{#1}}
\newcommand\@wydzial{}
%Kierunek
\newcommand\kierunek[1]{\renewcommand\@kierunek{#1}}
\newcommand\@kierunek{}
%Specjalność
\newcommand\specjalnosc[1]{\renewcommand\@specjalnosc{#1}}
\newcommand\@specjalnosc{}
%Tytuł po angielsku
\newcommand\titleEN[1]{\renewcommand\@titleEN{#1}}
\newcommand\@titleEN{}
%Tytuł krótki
\newcommand\titleShort[1]{\renewcommand\@titleShort{#1}}
\newcommand\@titleShort{}
%Promotor
\newcommand\promotor[1]{\renewcommand\@promotor{#1}}
\newcommand\@promotor{}
%Słowa kluczowe
\newcommand\kvpl[1]{\renewcommand\@kvpl{#1}}
\newcommand\@kvpl{}
\newcommand\kven[1]{\renewcommand\@kven{#1}}
\newcommand\@kven{}
%Komenda wykorzystywana w streszczeniu
\newcommand\mykeywords{\hspace{\absleftindent}%
\parbox{\linewidth-2.0\absleftindent}{
       \iflanguage{polish}{\textbf{Słowa kluczowe:} \@kvpl}{%
			 \iflanguage{english}{\textbf{Keywords:} \@kven}}{}}
				}

\def\maketitle{%
  \pagestyle{empty}%
%%\garamond 
	\fontfamily{\ebgaramond@family}\selectfont % na stronie tytułowej czcionka garamond
%%%%%%%%%%%%%%%%%%%%%%%%%%%%%%%%%%%%%%%%%%%%%%%%%%%%%%%%%%%%%%%%%%%%%%%%%%%%%%	
%% Poniżej, w otoczniu picture, wstawiono tytuł i autora. 
%% Tytuł (z autorem) musi znaleźć się w obszarze 
%% odpowiadającym okienku 110mmx75mm, którego lewy górny róg 
%% jest w położeniu 77mm od lewej i 111mm od górnej  krawędzi strony 
%% (tak wynika z wycięcia na okładce). 
%% Poniższy kod musi być użyty dokładnie w miejscu gdzie jest.
%% Jeśli tytuł nie mieści się w okienku, to należy tak pozmieniać 
%% parametry użytych komend, aby ten przydługi tytuł jednak 
%% upakować do okienka.
%%
%% Sama okładka (kolorowa strona z wycięciem, kiedyś była do pobrania z dydaktyki) 
%% powinna być przycięta o 3mm od każdej z krawędzi.
%% Te 3mm pewnie zostawiono na ewentualne spady czy też specjalną oprawę.
%%%%%%%%%%%%%%%%%%%%%%%%%%%%%%%%%%%%%%%%%%%%%%%%%%%%%%%%%%%%%%%%%%%%%%%%%%%%%%
\newlength{\tmpfboxrule}
\setlength{\tmpfboxrule}{\fboxrule}
\setlength{\fboxsep}{2mm}
\setlength{\fboxrule}{0mm} 
%\setlength{\fboxrule}{0.1mm} %% INFO: Jeśli chcemy zobaczyć ramkę, wystarczy odmarkować tę linijkę
\setlength{\unitlength}{1mm}
\begin{picture}(0,0)
%\put(26,-124){\fbox{% ustawienie do "wyciętego okienka"
\put(20,-124){\fbox{% ustawienie na środku
\parbox[c][71mm][c]{104mm}{\centering%\lineskip=34pt 
{\fontsize{18pt}{20pt}\bfseries\selectfont \@title}\\[5mm]
% {\fontsize{18pt}{20pt}\bfseries\selectfont \@titleEN}\\[10mm] % INFO: wstawiono tytuł w języku angielskim, choć w obecnych oficjalnych zaleceniach tego nie ma
%\fontsize{16pt}{18pt}\selectfont AUTOR:\\[2mm]
{\fontsize{16pt}{18pt}\selectfont \@author}}
}
}
\end{picture}
\setlength{\fboxrule}{\tmpfboxrule} 
%%%%%%%%%%%%%%%%%%%%%%%%%%%%%%%%%%%%%%%%%%%%%%%%%%%%%%%%%%%%%%%%%%%%%%%%%%%%%%
%% Reszta strony z nazwą uczelni, wydziału, kierunkiem, specjalnością
%% promotorem, oceną pracy (zakomentowane), miastem i rokiem
	{\vskip 9pt\centering
		{\fontsize{20pt}{22pt}\bfseries\selectfont \@uczelnia}\\[5pt]
		{\fontsize{16pt}{18pt}\bfseries\selectfont \@wydzial}\\[1pt]
		  \hrule
	}
{\vskip 24pt\raggedright\fontsize{14pt}{16pt}\selectfont%
\begin{tabular}{@{}ll}
Kierunek: & {\bfseries \@kierunek}\\
Specjalność: & {\bfseries \@specjalnosc}\\
\end{tabular}\\[1.3cm]
}
{\vskip 29pt\centering{\fontsize{24pt}{26pt}\selectfont%
{\fontsize{26pt}{28pt}\selectfont P}RACA {\fontsize{26pt}{24pt}\selectfont D}YPLOMOWA\\[7pt]
\ifMaster \selectfont{\fontsize{26pt}{28pt}\selectfont M}AGISTERSKA\\[2.5cm]%
\else \selectfont{\fontsize{26pt}{28pt}\selectfont I}NŻYNIERSKA\\[2.5cm]\fi
}}
	\vfill
{\centering
		{\fontsize{14pt}{16pt}\selectfont Opiekun pracy}\\[2mm] 
		{\fontsize{14pt}{16pt}\bfseries\selectfont \@promotor}\\[10mm]%INFO: tutaj wstawiane ejst nazwisko promotora
%		&{\fontsize{16pt}{18pt}\selectfont OCENA PRACY:}\\[20mm] 
% INFO: linię powyższą zakomentowano, gdyż od czasu pandemii COVID-19 prace mogą być dostarczane bez podpisu promotora
}
\vspace{4cm}\noindent
{\fontsize{12pt}{14pt}\selectfont Słowa kluczowe: \@kvpl}% INFO: na stronę tytułową trafiają tylko słowa kluczowe w języku polskim (w jakim napisana jest praca)
\vspace{1.3cm}
\hrule\vspace*{0.3cm}
{\centering
{\fontsize{14pt}{16pt}\selectfont \@date}\\[0cm]
}
%\ungaramond
\normalfont
 \cleardoublepage
}

\makeatother

\usepackage{listings}
% \usepackage{algorithm}
\makeatletter
\def\ext@algorithm{lol}% algorithm captions will be written to the .lol file
% share the list making commands and redefine the heading
\AtBeginDocument{%
  \let\l@algorithm\l@lstlisting
  \let\c@algorithm\c@lstlisting
  \let\thealgorithm\thelstlisting
  \renewcommand{\lstlistlistingname}{Algorithms and program code}%
}
\makeatother
\usepackage{subcaption}
\doublehyphendemerits=100000 % Domyślna wartość to 10000
\brokenpenalty=1000 % Domyślna wartość to 10000
\usepackage{multirow}

% Przypisy dolne
\usepackage{threeparttable}

\Mastertrue % INFO: odkomentuj, jeśli to praca magisterska
\title{Porównanie wybranych mechanizmów programowania współbieżnego \linebreak i równoległego w językach Rust i C++} % INFO: tytuł pracy w języku polskim 
\titleShort{Porównanie mechanizmów programowania współbieżnego i równoległego}  % INFO: krótki tytuł pracy (do zamieszczenia w stopce, sklejony z imieniem i nazwiskiem autora nie powinien zająć więcej niż jedną linijkę)
\titleEN{Comparison of selected concurrent and parallel programming mechanisms in Rust and C++} % INFO: tytuł pracy w języku angielskim
\author{Rafał Jasiński}  % INFO: imię i nazwisko autora
\uczelnia{Politechnika Wrocławska} % INFO: nazwa uczelni
\wydzial{Wydział Informatyki i Telekomunikacji} % INFO: nazwa wydziału
\kierunek{Informatyka Stosowana (IST)} % IFO: nazwa kierunku
\specjalnosc{Inżynieria Oprogramowania (IO)} % INFO: nazwa specjalności
\promotor{dr inż. Zdzisław Spławski} % INFO: dane promotora 
\date{WROCŁAW 2025} % INFO: miejscowość, rok złożenia pracy dyplomowej

%%%%%%%%%%%%%%%%%%%%%%%%%%%%%%%%%%%%%%%%%%%%%%%%%%%%%%%%%%%%%%%%%%%%%%%%%%%%%%%%%%
%%
%%  Struktura dokumentu
%%  - tutaj należy wstawić własne rozdziały
%%
%%%%%%%%%%%%%%%%%%%%%%%%%%%%%%%%%%%%%%%%%%%%%%%%%%%%%%%%%%%%%%%%%%%%%%%%%%%%%%%%%%

%%%%%%%%%%%%%%%%%%%%%%%%%%%%%%%%%%%%%%%%%%%%%%%%%%%%%%%%%%%%%%%%%%%%%%%%%%%%%%%%%%
% INFO: Za pomocą polecenia \includeonly{} można dokonać selekcji  
%       tych części (plików z latexowym kodem), które mają być kompilowane. 
%       Przydaje się to szczególnie podczas pracy nad dużymi dokumentami. 
%       Bo im mniej części zostanie wyselekcjonowanych, tym szybsza będzie kompilacja.
%       Proszę nie mylić tej komendy z poleceniem \include{}, którą używa się 
%       do zadeklarowania pełnej struktury dokumentu (plików z latexowym kodem).
%\includeonly{skroty,rozdzial01}  

\begin{document}

\maketitle
\newpage
\thispagestyle{empty}
\mbox{}
\newpage
% Kolejne części dokumentu: streszczenie, spisy, skróty, rozdziały, dodatki
%\chapterstyle{noNumbered}
% STRESZCZENIE (proszę zajrzeć do środka na zakomentowane komendy)
\pdfbookmark[0]{Streszczenie}{streszczenie.1}

%\mbox{}\vspace{2cm} % mo¿na przesun¹æ, w zale¿noœci od d³ugoœci streszczenia
\begin{abstract}
TEMPLATE
Praca skupia się na projekcie i implementacji aplikacji wykorzystującej algorytmy genetyczne wraz z ich wizualizacją. Pierwsza część obejmuje teoretyczne podstawy tych algorytmów, porównując je do mechanizmów biologicznej genetyki. Omówiono schemat działania, historię oraz kluczowe elementy, takie jak osobnik, populacja, selekcja, krzyżowanie, mutacja i funkcja celu. Następnie przedstawiono założenia projektowe, obejmujące kodowanie osobnika, metody selekcji, operatory krzyżowania, opcje mutacji, funkcję celu, interfejs użytkownika, przykład użycia\linebreak i strukturę aplikacji. Zawierają one opis ich zasady działania.

Implementacja aplikacji została opisana w kolejnym etapie, prezentując użyte technologie, wybrany język programowania wraz z interfejsem użytkownika i inne narzędzia. Szczegółowo omówiono implementację osobnika w kodowaniu binarnym, wybór wariantów operacji, metody selekcji, krzyżowania, mutacji, funkcji przystosowania oraz wygląd interfejsu użytkownika wraz z opisem.

Analiza wyników pracy obejmuje testy na danych testowych oraz porównanie różnych metod selekcji, krzyżowania i mutacji. Wnioski z porównań są przedstawione dla każdej badanej metody, dostarczając czytelnikowi kompleksowego spojrzenia na skuteczność poszczególnych elementów algorytmów genetycznych.

Całość pracy zawiera podsumowanie, gdzie prezentowane są główne osiągnięcia oraz wnioski podczas pisania pracy. Praca dostarcza wartościowego spojrzenia na zastosowanie algorytmów genetycznych w projektowaniu aplikacji, a także oferuje praktyczne wskazówki dotyczące implementacji i optymalizacji tych algorytmów.

\end{abstract}


{
\selectlanguage{english}
\begin{abstract}
TEMPLATE
The thesis focuses on the design and implementation of an application utilizing genetic algorithms along with their visualization. The first part covers the theoretical foundations of these algorithms, comparing them to the mechanisms of biological genetics. The operation scheme, history, and key elements such as individual, population, selection, crossover, mutation, and fitness function are discussed. The design assumptions are then presented, including individual encoding, selection methods, crossover operators, mutation options, the fitness function, user interface, usage example, and application structure. They contain a description of their principles of operation.

The application implementation is described in the next stage, presenting the technologies used, the chosen programming language along with the user interface, and other tools. The implementation of the individual in binary encoding, the selection of operation variants, selection methods, crossover, mutation, fitness function, and the appearance of the user interface are discussed in detail.

The results analysis includes tests on test data and a comparison of different selection, crossover, and mutation methods. Conclusions from the comparisons are presented for each investigated method, providing the reader with a comprehensive view of the effectiveness of individual elements of genetic algorithms.

The entire thesis includes a conclusion where the main achievements and conclusions drawn during the writing process are presented. The paper provides \linebreak a valuable perspective on the application of genetic algorithms in application design and also offers practical guidance on the implementation and optimization of these algorithms.
\end{abstract}

}

\pagestyle{outer}
\clearpage
% SPIS TREŚCI (zostanie wygenerowany automatycznie)
\pdfbookmark[0]{Spis treści}{spisTresci.1}%
%%\phantomsection
%%\addcontentsline{toc}{chapter}{Spis treści}
\tableofcontents* 
\clearpage

\clearpage
% SKRÓTY (to opcjonalna część pracy)
% \pdfbookmark[0]{Skróty}{skroty.1}% 
%%\phantomsection
%%\addcontentsline{toc}{chapter}{Skróty}
\chapter*{Skróty}
\label{sec:skroty}
\noindent\vspace{-\topsep-\partopsep-\parsep} % Jeœli zaczyna siê od otoczenia description, to otoczenie to l¹duje lekko ni¿ej ni¿ wyl¹dowa³by zwyk³y tekst, dlatego wstawiano przesuniêcie w pionie
\begin{description}[labelwidth=*]
  \item [GA] (ang.\ \emph{Genetic Algorithm})

\end{description}
 
% ROZDZIAŁY (kolejne rozdziały dołączane są z kolejnych plików)
\chapterstyle{default}


\chapter[Wstęp]{Wstęp}
% \addcontentsline{toc}{chapter}{Wstęp}  % Add unnumbered chapter to the table of contents if needed
\section{Cel oraz zakres pracy}
% \addcontentsline{toc}{chapter}{Cel i zakres pracy}  % Add unnumbered chapter to the table of contents if needed
Celem niniejszej pracy jest przeprowadzenie pogłębionej analizy oraz wszechstronnego porównania mechanizmów programowania współbieżnego i równoległego w dwóch językach programowania: Rust i C++. Celem jest przedstawienie kluczowych różnic oraz podobieństw w~podejściu do zarządzania wielowątkowością, analizując jednocześnie efektywność, bezpieczeństwo oraz wygodę stosowania narzędzi dostępnych w obu językach.

W ramach pracy szczególną uwagę poświęcono omówieniu wybranych bibliotek i frameworków, które wspierają tworzenie aplikacji wielowątkowych w Rust (np. Tokio, Rayon) i C++ (np. std::thread, OpenMP, TBB). Przeanalizowane zostaną mechanizmy bezpieczeństwa oraz zarządzania pamięcią i wątkami, które odgrywają kluczową rolę w zapewnieniu stabilności i~wydajności aplikacji współbieżnych i równoległych.

Dodatkowym celem jest przeprowadzenie analizy wydajności oraz efektywności implementacji aplikacji wielowątkowych, co pozwoli na ocenę szybkości działania i efektywnego zarządzania zasobami w obu językach. Badanie uwzględni również aspekty praktyczne, takie jak łatwość użycia narzędzi, dostępność wsparcia ze strony społeczności oraz dojrzałość ekosystemu każdego z języków.

Aby zilustrować wyniki teoretyczne w praktyce, przeprowadzona zostanie implementacja aplikacji współbieżnych i równoległych w obu językach, co umożliwi porównanie osiągniętych wyników wydajnościowych oraz analizę różnic w strukturze i stylu kodu. Efektem pracy będzie również identyfikacja scenariuszy, w których jeden z języków może przewyższać drugi pod względem wydajności, bezpieczeństwa, czy wygody stosowania, co pozwoli na sformułowanie rekomendacji dotyczących wyboru języka w zależności od specyficznych wymagań projektowych.
\section{Problem badawczy}
Wraz z rozwojem nowoczesnych technologii informatycznych i rosnącą złożonością systemów obliczeniowych, znaczenia nabierają paradygmaty programowania, które pozwalają na maksymalne wykorzystanie zasobów współczesnego sprzętu komputerowego — w szczególności architektur wielordzeniowych \cite{10876950,Sesay2024Parallelism}. Programowanie współbieżne i równoległe stanowią obecnie podstawę projektowania wydajnych i niezawodnych aplikacji w wielu obszarach, od systemów operacyjnych, aż po rozwiązania z zakresu sztucznej inteligencji czy gier komputerowych.

W kontekście tych wyzwań szczególnie interesujące staje się porównanie narzędzi, jakie oferują współczesne języki programowania. Niniejsza praca magisterska koncentruje się na analizie dwóch języków: Rust oraz C++, które - mimo odmiennej filozofii projektowej - są powszechnie wykorzystywane w systemach wymagających wysokiej wydajności \cite{Hazarika2025RustVsCpp}. Rust, jako stosunkowo młody język, zdobywa coraz większą popularność ze względu na nowatorskie podejście do bezpieczeństwa pamięci i współbieżności, opierające się na systemie własności \eng{ownership} oraz sprawdzaniu poprawności kodu na etapie kompilacji, co potwierdzają zarówno badania naukowe, jak i praktyczne analizy \cite{Saligrama2019RustConcurrency,RustFC}. Dzięki temu minimalizuje ryzyko wycieków pamięci, błędów synchronizacji czy wyścigów danych. Z kolei C++ - język dojrzały, o długiej historii i~ugruntowanej pozycji w przemyśle - oferuje szeroki wachlarz dojrzałych bibliotek równoległych (OpenMP, Intel TBB), jednak często kosztem większego ryzyka błędów programistycznych i bardziej złożonego zarządzania pamięcią.

Przegląd literatury pokazuje, że większość dotychczasowych badań porównawczych koncentrowała się na klasycznych architekturach x86\_64, co skutkuje ograniczoną wiedzą na temat wpływu nowoczesnych architektur ARM64 na wydajność mechanizmów programowania współbieżnego i~równoległego. Choć pojawiają się już prace dotyczące tego zagadnienia, wciąż brakuje szeroko zakrojonych, systematycznych analiz obejmujących różne architektury sprzętowe i~zestawy benchmarkowe.

W kwestii porównania implementacji w języku Rust z dojrzałymi rozwiązaniami C++ przy użyciu uznanych zestawów benchmarkowych, takich jak NAS Parallel Benchmarks, literatura naukowa jest wciąż uboga. W ostatnim czasie pojawiły się jednak pierwsze publikacje, które próbują wypełnić tę lukę - przykładem jest praca \cite{martins2025npbrustnasparallelbenchmarks}, gdzie autorzy analizują wydajność Rust w~kontekście benchmarków równoległych, jednak nie obejmuje ona kompleksowego porównania z rozwiązaniami C++ na różnych platformach sprzętowych.

W efekcie, choć pojawiają się już pojedyncze próby porównania Rust i C++ w kontekście benchmarków równoległych, wciąż brakuje systematycznych, szeroko zakrojonych badań, które uwzględniałyby zarówno różne architektury (w tym ARM64), jak i różne zestawy benchmarków, a także zapewniałyby porównanie wydajności i niezawodności obu języków w praktycznych zastosowaniach.

W związku z powyższym, głównym problemem badawczym pracy są następujące pytania:
\begin{quote}
    \item \textbf{PB1}: 
    \emph{Jakie są różnice w wydajności i charakterystykach skalowania mechanizmów programowania równoległego między językami Rust (Rayon) a C++ (OpenMP, Intel TBB) na architekturach ARM64 i x86\_64, mierzone przy użyciu standardowych benchmarków NAS Parallel Benchmarks (CG, EP, IS)?}
    \item \textbf{PB2}:
    \emph{W jaki sposób wybór konkretnego języka i biblioteki wpływa na wydajność aplikacji współbieżnych pod względem przepustowości, zużycia pamięci oraz stabilności działania w~różnych środowiskach kompilacji?}
    \item \textbf{PB3}:
    \emph{Jaki jest rzeczywisty narzut wydajnościowy związany z modelem bezpieczeństwa pamięci w języku Rust w porównaniu z mechanizmami zarządzania zasobami dostępnymi w C++, i czy różnice te są zależne od architektury sprzętowej?}
\end{quote}
Odpowiedź na te pytania zostanie udzielona poprzez systematyczne eksperymentalne porównanie konkretnych implementacji w obu językach, z wykorzystaniem standardowych metodologii pomiarowych oraz analizy statystycznej wyników uzyskanych na dwóch reprezentatywnych architekturach sprzętowych. Badania obejmą zarówno benchmarki obliczeniowe wysokiej wydajności, jak i praktyczne aplikacje reprezentujące typowe scenariusze współbieżności, co pozwoli na sformułowanie rekomendacji dotyczących wyboru odpowiednich narzędzi w zależności od specyfiki projektu i środowiska docelowego.

Wybór tematu pracy został dodatkowo umotywowany aktualnymi inicjatywami w zakresie bezpieczeństwa cybernetycznego oraz osobistymi doświadczeniami zawodowymi autora. W grudniu 2023 roku Agencja Bezpieczeństwa Cybernetycznego i Infrastruktury (CISA), we współpracy z Agencją Bezpieczeństwa Narodowego (NSA), FBI oraz międzynarodowymi organami ds. cyberbezpieczeństwa z Australii, Kanady, Nowej Zelandii i Wielkiej Brytanii, opublikowała wspólny przewodnik "The Case for Memory Safe Roadmaps" w ramach kampanii "Secure by Design" \cite{DoD2023MemorySafe}. Dokument ten zdecydowanie zachęca kierownictwo producentów oprogramowania do priorytetowego traktowania języków programowania bezpiecznych pod względem pamięci, tworzenia i publikowania map drogowych bezpieczeństwa pamięci oraz wdrażania zmian w celu wyeliminowania tej klasy podatności.

Agencje te podkreślają, że błędy bezpieczeństwa pamięci stanowią najczęstszy typ ujawnianych podatności oprogramowania, przy czym Microsoft potwierdził, że około 70\% jego błędów (CVE) to podatności związane z bezpieczeństwem pamięci, a Google poświadczył podobną cyfrę dla projektu Chromium. W tym kontekście agencje zalecają organizacjom odejście od C/C++, ponieważ nawet przy szkoleniach w zakresie bezpieczeństwa (i ciągłych wysiłkach na rzecz wzmocnienia kodu C/C++), programiści nadal popełniają błędy.

Dodatkowo, doświadczenia zawodowe autora w obszarze programowania równoległego wskazują na praktyczną potrzebę obiektywnej oceny narzędzi dostępnych w obu językach. W~środowisku przemysłowym coraz częściej pojawia się pytanie o zasadność migracji z dojrzałych rozwiązań C++ w kierunku nowszych technologii oferowanych przez Rust, szczególnie w kontekście aplikacji krytycznych pod względem wydajności i bezpieczeństwa.

%%Układ dokumentu - Układ tego dokumentu przedstawia się następująco:
\section{Struktura pracy}
Niniejsza praca została zorganizowana w sposób umożliwiający systematyczne przedstawienie zagadnienia oraz przeprowadzenie kompleksowej analizy porównawczej mechanizmów programowania współbieżnego i równoległego w językach Rust i C++.
Rozdział pierwszy przedstawia cel oraz zakres pracy, definiuje główne problemy badawcze i określa metodologię badań. Rozdział drugi wprowadza czytelnika w podstawy programowania współbieżnego i równoległego, omawiając kluczowe różnice między tymi paradygmatami oraz ich zastosowania w kontekście współczesnych systemów wielordzeniowych. Trzeci rozdział zawiera przegląd literatury przedmiotu, analizując dotychczasowe badania porównawcze języków programowania w kontekście mechanizmów wielowątkowości oraz identyfikując luki badawcze, które niniejsza praca ma wypełnić.
Rozdziały czwarty i piąty skupiają się na teoretycznych aspektach mechanizmów programowania współbieżnego i równoległego w analizowanych językach, omawiając dostępne biblioteki, frameworki oraz narzędzia zarządzania wątkami i pamięcią. Rozdział szósty przedstawia metodologię badań oraz procedury pomiarowe wykorzystane w eksperymentach, w tym opis środowiska sprzętowo-programowego, konfiguracji kompilatorów oraz protokołu eksperymentalnego dla benchmarków NPB i aplikacji testowych. Rozdział siódmy zawiera opis i~implementację benchmarków NPB (Conjugate Gradient, Embarrassingly Parallel, Integer Sort) dla programowania równoległego w językach Rust i C++. Rozdział ósmy przedstawia szczegółowy opis implementacji aplikacji testowych dla programowania współbieżnego (producent-konsument, echo-serwer), koncentrując się na specyfice programistycznej oraz architekturalnych rozwiązaniach zastosowanych w obu językach. Rozdział dziewiąty prezentuje wyniki eksperymentów dotyczących programowania równoległego oraz ich szczegółową analizę, koncentrując się na porównaniu wydajności benchmarków NPB, wzorcach skalowania oraz wykorzystaniu zasobów systemowych przez implementacje na architekturach ARM64 i x86\_64. Rozdział dziesiąty zawiera analizę wyników eksperymentów dotyczących programowania współbieżnego, przedstawiając porównanie wydajności aplikacji testowych oraz charakterystyk behawioralnych implementacji w obu językach.
Rozdział jedenasty prezentuje wnioski oraz rekomendacje dotyczące praktycznego zastosowania języków Rust i C++ w projektach wymagających wysokiej wydajności i bezpieczeństwa współbieżnego, identyfikując scenariusze optymalnego wykorzystania każdego z języków. Rozdział dwunasty podsumowuje najważniejsze osiągnięcia badawcze oraz wskazuje możliwe kierunki dalszych analiz i rozwijania zaproponowanych rozwiązań.
Na samym końcu pracy znajduje się literatura, spis wykorzystanych rysunków, tabel oraz listingów kodu.
% \newpage
\section{Słownik wybranych pojęć}

\begin{itemize}
    \item \textbf{licznik Redis} - to mechanizm wykorzystujący bazę danych Redis do przechowywania i aktualizowania liczników w czasie rzeczywistym. Redis, jako szybka baza typu klucz-wartość, pozwala na błyskawiczne operacje inkrementacji i dekrementacji wartości przypisanej do danego klucza.

    \item \textbf{LLVM} - \eng{Low Level Virtual Machine} - to zestaw narzędzi i bibliotek do budowania kompilatorów, który umożliwia generowanie, analizę i optymalizację kodu (zarówno w czasie kompilacji, jak i wykonania). LLVM nie jest maszyną wirtualną w tradycyjnym sensie, ale raczej infrastrukturą kompilatora, która operuje na pośrednim języku reprezentacji (LLVM IR), z którego może generować kod maszynowy dla różnych architektur.

    \item \textbf{nieustraszona współbieżność} - \eng{fearless concurrency} - to podejście do programowania współbieżnego, które eliminuje problemy związane z bezpieczeństwem pamięci i synchronizacją wątków. W Rust osiągnięto to dzięki systemowi własności, który zapewnia, że dane mogą być modyfikowane tylko przez jeden wątek naraz, eliminując ryzyko wyścigów danych i błędów synchronizacji.

    \item \textbf{odwołania do nieobecnych stron} - \eng{page fault} - zdarzenie w systemie operacyjnym, które występuje, gdy program próbuje uzyskać dostęp do strony pamięci, która nie znajduje się obecnie w pamięci RAM. Może to skutkować koniecznością załadowania tej strony z dysku (np. z pliku wymiany), co wpływa na wydajność programu.

    \item \textbf{zwolnienie stron pamięci} - \eng{page reclaims} - operacje systemowe polegające na odzyskiwaniu już załadowanych, ale nieaktywnych stron pamięci, aby umożliwić ich ponowne wykorzystanie przez inne procesy. Pomaga to zoptymalizować wykorzystanie pamięci fizycznej bez konieczności natychmiastowego odwoływania się do pamięci wirtualnej.

    \item \textbf{pożyczanie} \eng{borrow} - również występujący pod inną nazwą jako przenoszenie własności \cite{rustPolishNames}, jest to mechanizm pozwalający na używanie wartości bez przejmowania jej na własność. Dzięki temu możemy przekazywać dane do funkcji lub między częściami programu bez ich kopiowania czy przenoszenia.

    \item \textbf{proces} - to instancja programu, która jest wykonywana w systemie operacyjnym. Procesy są izolowane od siebie i mają własne zasoby, takie jak pamięć i przestrzeń adresowa.

    \item \textbf{programowanie równoległe} - to sposób wykonywania wielu zadań jednocześnie, co zwiększa wydajność programu. W odróżnieniu od programowania współbieżnego, programowanie równoległe polega na wykonywaniu zadań w tym samym czasie, a nie przeplataniu ich w~czasie.

    \item \textbf{programowanie współbieżne} - technika programistyczna polegająca na jednoczesnym wykonywaniu wielu zadań lub ich przeplataniu w czasie, mająca na celu zwiększenie efektywności działania programu. Współbieżność może być realizowana z wykorzystaniem wielu wątków, procesów bądź mechanizmów programowania asynchronicznego, które wewnętrznie mogą operować na wątkach lub innych zasobach udostępnianych przez system operacyjny.

    \item \textbf{SIMD} - \eng{Single Instruction, Multiple Data} - pojedyncza instrukcja wykonywana na wielu danych jednocześnie. Jest to technika optymalizacji wydajności obliczeń, która wykorzystuje jednostki wektorowe dostępne w nowoczesnych procesorach.

    \item \textbf{wątek} - część programu wykonywana współbieżnie w obrębie jednego procesu - w jednym procesie może istnieć wiele wątków. Główna różnica między procesem a wątkiem polega na tym, że wszystkie wątki należące do tego samego procesu współdzielą przestrzeń adresową oraz inne zasoby systemowe, takie jak listy otwartych plików czy gniazda sieciowe. Natomiast każdy proces dysponuje własnym, odrębnym zestawem zasobów.

    \item \textbf{własność} \eng{ownership} - system zarządzania pamięcią, który eliminuje konieczność używania automatycznego odśmiecania, jednocześnie zapobiegając błędom takim jak użycie po zwolnieniu czy podwójne zwolnienie.

    \item \textbf{wyścigi danych} \eng{race conditions} - to sytuacja, w której dwa lub więcej wątków lub procesów próbuje modyfikować wspólną zmienną w tym samym czasie, co prowadzi do nieprzewidywalnych wyników.
\end{itemize}


\chapter{Cel oraz zakres pracy}
% \addcontentsline{toc}{chapter}{Cel i zakres pracy}  % Add unnumbered chapter to the table of contents if needed
Celem niniejszej pracy jest przeprowadzenie pogłębionej analizy oraz wszechstronnego porównania mechanizmów programowania współbieżnego i równoległego w dwóch językach programowania: Rust i C++. Celem jest przedstawienie kluczowych różnic oraz podobieństw w podejściu do zarządzania wielowątkowością, analizując jednocześnie efektywność, bezpieczeństwo oraz wygodę stosowania narzędzi dostępnych w obu językach.

W ramach pracy szczególną uwagę poświęcono omówieniu wybranych bibliotek i frameworków, które wspierają tworzenie aplikacji wielowątkowych w Rust (np. Tokio, Rayon) i C++ (np. std::thread, OpenMP, TBB). Przeanalizowane zostaną mechanizmy bezpieczeństwa oraz zarządzania pamięcią i wątkami, które odgrywają kluczową rolę w zapewnieniu stabilności i wydajności aplikacji współbieżnych i równoległych.

Dodatkowym celem jest przeprowadzenie analizy wydajności oraz efektywności implementacji aplikacji wielowątkowych, co pozwoli na ocenę szybkości działania i efektywnego zarządzania zasobami w obu językach. Badanie uwzględni również aspekty praktyczne, takie jak łatwość użycia narzędzi, dostępność wsparcia ze strony społeczności oraz dojrzałość ekosystemu każdego z języków.

Aby zilustrować wyniki teoretyczne w praktyce, przeprowadzona zostanie implementacja aplikacji współbieżnych i równoległych w obu językach, co umożliwi porównanie osiągniętych wyników wydajnościowych oraz analizę różnic w strukturze i stylu kodu. Efektem pracy będzie również identyfikacja scenariuszy, w których jeden z języków może przewyższać drugi pod względem wydajności, bezpieczeństwa, czy wygody stosowania, co pozwoli na sformułowanie rekomendacji dotyczących wyboru języka w zależności od specyficznych wymagań projektowych.
\chapter{Przegląd literatury}
Celem niniejszego rozdziału jest przedstawienie dotychczasowych badań i publikacji dotyczących mechanizmów programowania współbieżnego i równoległego w językach Rust i C++. Analiza literatury umożliwi zrozumienie aktualnego stanu wiedzy w tej dziedzinie, a także wskazanie na występujące luki badawcze, które niniejsza praca postara się wypełnić.
Na samym wstępie zostały postawione następujące pytania do przeglądu literatury, które pomogą zrozumieć oraz sprawdzić aktualny stan wiedzy jeżeli chodzi o porównanie języków Rust oraz C++:

\begin{quote}
    \item \textbf{PPL1:} \emph{Jakie główne koncepcje/teorie dominują w literaturze dotyczącej porównania języków Rust oraz C++?} \label{PPL1}
    \item \textbf{PPL2:} \emph{Jakie metody badawcze są najczęściej stosowane do analizy różnic pomiędzy językami?}
    \item \textbf{PPL3:} \emph{Jak wygląda porównanie dostępności i dojrzałości bibliotek do programowania współbieżnego i równoległego w obu językach?}
    \item \textbf{PPL4:} \emph{Czy istnieją systematyczne metodologie porównywania języków programowania w~kontekście współbieżności, które można zastosować do analizy Rust i C++?}
    \item \textbf{PPL5:} \emph{Jakie aspekty programowania współbieżnego i równoległego w Rust i C++ nie zostały dostatecznie zbadane w literaturze?}
    \item \textbf{PPL6:} \emph{Jaki jest stan wiedzy na temat wykorzystania programowania współbieżnego w ramach GPU w językach Rust i C++?}
\end{quote}

Odpowiedzi na powyższe pytania pozwolą na zidentyfikowanie kluczowych obszarów, które wymagają dalszych badań oraz na wskazanie na potencjalne kierunki rozwoju w dziedzinie programowania współbieżnego i równoległego w językach Rust i C++.
\subsection{Metodologia przeglądu literatury}
Proces przeglądu literatury został zrealizowany zgodnie z zasadami przeglądu systematycznego, co oznaczało zastosowanie jasno określonych kryteriów selekcji i wyłączenia. Główne źródła literaturowe obejmowały artykuły naukowe, materiały konferencyjne oraz dokumentację techniczną. Wyszukiwanie przeprowadzono w renomowanych bazach danych naukowych oraz repozytoriach zawierających publikacje z zakresu inżynierii oprogramowania i języków programowania. Dodatkowo zostały również uwzględnione źródła internetowe oraz dokumentacje techniczne.

Przegląd literatury odbywał się z wykorzystaniem narzędzi baz danych oferujących wyszukiwanie, filtrowanie oraz przegląd prac: Scopus, Google Scholar.

\subsection{Kryteria selekcji oraz wyłączenia}
\label{KryteriaSelekcji}
W procesie selekcji literatury uwzględniano przede wszystkim publikacje wydane po 2012 roku, co wynika z faktu, iż w tym właśnie roku zadebiutował język Rust \cite{wikipediaRustprogramming}. Wyjątek stanowiły prace o charakterze ogólnym lub takie, które nie odnosiły się bezpośrednio do języka Rust, lecz zawierały istotne informacje dla problematyki badawczej niniejszej pracy.

Analiza obejmowała literaturę w języku polskim oraz angielskim, przy czym zdecydowana większość źródeł stanowiły publikacje anglojęzyczne. Selekcja materiałów opierała się na zgodności tematycznej z zakresem badań. W przypadku wątpliwości co do adekwatności danej pozycji, decyzja o jej włączeniu do przeglądu podejmowana była na podstawie analizy streszczenia. Jeśli po tej analizie publikacja wydawała się istotna, przechodzono do pełnej oceny jej treści.

Publikacje, które po dogłębnej analizie okazywały się nieodpowiednie dla głównego problemu badawczego, nie były uwzględniane w zasadniczej części pracy. Niemniej jednak, jeśli przyczyniły się do lepszego zrozumienia badanego zagadnienia lub pomogły w odpowiedzi na pytania do przeglądu literatury \ref{PPL1}, były one odnotowywane jako materiały pomocnicze. Prace niespełniające powyższych kryteriów lub te, które nie są dostępne za pośrednictwem dostępnych metod (bądź też braku odpowiedzi twórców o prośbę udostępnienia pracy) były wykluczane z~dalszej analizy.


\subsection{Baza Scopus}
W ramach bazy Scopus wykorzystano następujące kwerendy do wyszukiwania - tabela \ref{table:literatureReviewQueries}

\begin{table}[H]
    \caption{Kwerendy użyte w bazie Scopus \protect \footnotemark}
    \label{table:literatureReviewQueries}
    \begin{tabular}{cp{11cm}c}
    \hline
    Lp. & Kwerenda & Liczba wyników \\ \hline
    1 & ALL ("concurrent programming"\ OR "parallel programming") AND (ALL ("Rust") AND ALL ("C++")) & 444 \\ \hline

    2 & ALL ("concurrent programming"\ OR "parallel programming") AND (ALL ("Rust") AND ALL ("C++") ) AND ( ALL ("compare")) & 28 \\ \hline

    3 & (TITLE-ABS-KEY(("concurrent programming"\ OR "parallel programming") AND ("Rust"\ AND "C++"))) AND (TITLE-ABS-KEY("comparison"\ OR "evaluation"\ OR "benchmark")) & 6 \\ \hline

    4 & (TITLE-ABS-KEY(("thread"\ OR "async"\ OR "future"\ OR "actor model"\ OR "message passing"\ OR "shared memory") AND ("Rust"\ AND "C++"))) AND (TITLE-ABS-KEY("comparison"\ OR "performance"\ OR "evaluation")) & 50 \\ \hline

    5 & (TITLE-ABS-KEY(("Rust"\ AND "C++") AND ("concurrency model"\ OR "parallel constructs"\ OR "multithreading"))) AND (TITLE-ABS-KEY("comparison"\ OR "study")) & 2 \\ \hline

    \end{tabular}
\end{table}
\footnotetext{Liczba wyników dla poszczególnych zapytań może się różnić w zależności od daty (wyszukiwanie przeprowadzono w okresie listopad-luty 2024/25).}

W celu identyfikacji literatury związanej z porównaniem wybranych mechanizmów programowania współbieżnego i równoległego w~językach Rust i C++, opracowano pięć zapytań w~bazie Scopus, z których każde miało określony cel badawczy. Pierwsze zapytanie miało na celu uzyskanie ogólnego przeglądu literatury, wyszukując wszystkie dokumenty, w których występują jednocześnie zagadnienia programowania współbieżnego lub równoległego oraz języki Rust i C++, niezależnie od kontekstu. Pozwoliło to oszacować ogólną skalę badań łączących te zagadnienia. Drugie zapytanie zawężało zakres wyszukiwania poprzez dodanie słowa kluczowego „compare”, co umożliwiło wyodrębnienie publikacji, w których dokonano bezpośredniego porównania języków Rust i C++ w kontekście współbieżności lub równoległości. Dzięki temu uzyskano bardziej ukierunkowany zbiór literatury odnoszącej się do analizy porównawczej. Trzecie zapytanie charakteryzowało się większą precyzją, ograniczając wyniki do tytułów, streszczeń oraz słów kluczowych, i uwzględniało wyłącznie publikacje zawierające odniesienia do ewaluacji, porównań bądź benchmarków języków Rust i C++. Takie podejście pozwoliło wyselekcjonować najbardziej tematycznie powiązane prace. Czwarte zapytanie miało charakter bardziej techniczny, koncentrując się na konkretnych mechanizmach współbieżności, takich jak wątki, asynchroniczność, obiekty typu futury, model aktorów, przesyłanie komunikatów czy pamięć współdzielona, w połączeniu z terminami dotyczącymi wydajności i oceny. Umożliwiło to dotarcie do badań analizujących niskopoziomowe aspekty działania tych mechanizmów w obu językach. Piąte zapytanie skupiało się na poziomie koncepcyjnym, wyszukując publikacje zawierające takie terminy jak model współbieżności, konstrukty równoległe czy wielowątkowość, wraz z frazami dotyczącymi porównań lub analiz. Celem było zidentyfikowanie prac badających różnice w podejściu do współbieżności na poziomie architektury języka i jego konstrukcji wewnętrznych.

Autor zdecydował się również użyć nowego, wbudowanego narzędzia w systemie \mbox{Scopus - Scopus AI}. Narzędzie to oparte na sztucznej inteligencji, wspomaga eksplorację akademicką w oparciu o dane z platformy Scopus. Dzięki integracji z narzędziem Copilot optymalizuje wyszukiwania, łącząc metody semantyczne i dopasowanie słów kluczowych. Choć Scopus AI ułatwia badania, jego wyniki warto weryfikować, ponieważ mogą zawierać nieścisłości lub stronniczość. \\
Po wprowadzeniu tytułu pracy w języku angielskim jako kwerendę, Scopus AI zwrócił 9 wyników, biorąc pod uwagę kwerendę stworzoną na podstawie tytułu pracy, zamieszczoną w listingu \ref{AIQuery}. Zwrócone prace pokrywają się z przeglądem umieszczonym w tabeli \ref{table:literatureReviewQueries} 

\lstset{breaklines=true}
\begin{lstlisting}[caption=Kwerenda wygenerowana przez AI, label=AIQuery]
("concurrent programming" OR "parallel programming" OR "multithreading" OR "asynchronous")
AND ("Rust" OR "C++" OR "programming languages" OR "software development")
AND ("performance" OR "efficiency" OR "scalability" OR "resource management")
AND ("synchronization" OR "thread safety" OR "deadlock" OR "race condition")
AND ("libraries" OR "frameworks" OR "tools" OR "APIs")
\end{lstlisting}

Na podstawie zapytań wykonanych w bazie Scopus zidentyfikowano publikacje odpowiadające tematyce współbieżności i równoległości w językach Rust i C++ (prace w ramach poszczególny zapytaniach się powtarzały). Jednak z uwagi na ograniczenia czasowe oraz objętość wyników, szczegółowej analizie poddano jedynie pierwsze 15 stron wyników, jeżeli było ich więcej, co odpowiada około 150 pracom.

Proces selekcji, obejmujący kolejne etapy oceny tematycznej, lektury streszczeń i wybranych pełnych tekstów zamieszczono w tabeli \ref{table:selectionProcessScopus}.

\begin{table}[H]
    \caption{Przebieg selekcji literatury (baza Scopus)} 
    \label{table:selectionProcessScopus}
    \begin{tabular}{cp{11.5cm}c}
    \hline
    Etap & Opis & Liczba prac  \\ \hline 
    1 & Prace wyjściowe (przejrzane - pierwsze 15 stron wyników) & 247 \\ \hline 
    2 & Wstępna selekcja tematyczna - usunięcie prac niezwiązanych bezpośrednio z tematem badania & 97 \\ \hline 
    3 & Lektura streszczeń - eliminacja pozycji bez wartości empirycznej lub porównawczej & 39 \\ \hline 
    4 & Analiza pełnych treści - wybór prac zawierających konkretne porównania, benchmarki lub studia przypadków & 12 \\ \hline 
    5 & Prace kluczowe dla problemu badawczego - porównania/omówienie mechanizmów w Rust i C++ & 9 \\ \hline 
    \end{tabular} 
\end{table}

\subsection{Baza Google Scholar}
\begin{table}[H]
    \caption{Kwerendy użyte w bazie Scopus \protect \footnotemark}
    \label{table:literatureReviewQueries}
    \begin{tabular}{cp{11cm}c}
    \hline
    Lp. & Kwerenda & Liczba wyników \\ \hline
    1 & Comparison of selected concurrent and parallel programming mechanisms in Rust and C++ & 321 \\ \hline

    \end{tabular}
\end{table}
Podobnie jak w przypadku bazy Scopus, ze względu na dużą ilośc wyników zostało wzięte pod uwagę pierwsze 10 stron bazy (100 wyników). Proces selekcji, obejmujący kolejne etapy oceny tematycznej, lektury streszczeń i wybranych pełnych tekstów, przebiegał według poniższego schematu zamieszczono w tabeli \ref{table:selectionProcessGoogle}.
\begin{table}[H]
    \caption{Przebieg selekcji literatury (baza Google Scholar)} 
    \label{table:selectionProcessGoogle}
    \begin{tabular}{cp{11.5cm}c}
    \hline
    Etap & Opis & Liczba prac  \\ \hline 
    1 & Prace wyjściowe (przejrzane - pierwsze 10 stron wyników) & 100 \\ \hline 
    2 & Wstępna selekcja tematyczna - usunięcie prac niezwiązanych bezpośrednio z tematem badania & 37 \\ \hline 
    3 & Lektura streszczeń - eliminacja pozycji bez wartości empirycznej lub porównawczej & 9 \\ \hline 
    4 & Analiza pełnych treści - wybór prac zawierających konkretne porównania, benchmarki lub studia przypadków & 8 \\ \hline 
    5 & Prace kluczowe dla problemu badawczego - porównania/omówienie mechanizmów w Rust i C++ & 4 \\ \hline 
    \end{tabular} 
\end{table}
\footnotetext{Liczba wyników dla poszczególnych zapytań może się różnić w zależności od daty (wyszukiwanie przeprowadzono w okresie marzec-kwiecień 2025).}

\section{Porównanie Rust oraz C++}
Porównanie języków programowania Rust i C++ jest przedmiotem licznych publikacji, które analizują ich różnorodne aspekty, takie jak struktura kodu, sposób kompilacji, bezpieczeństwo, wydajność oraz obsługa współbieżności i równoległości. Choć istnieje szeroka literatura porównawcza, wciąż stosunkowo nieliczne prace skupiają się w sposób bezpośredni i systematyczny na porównaniu mechanizmów współbieżnego i równoległego przetwarzania w tych dwóch językach. W ostatnich latach pojawiły się jednak wartościowe opracowania, również te akademickie, które podejmują to zagadnienie. Pomimo to, że liczba takich prac nadal jest ograniczona, ich jakość i rosnące zainteresowanie środowiska akademickiego wskazują na istotny potencjał badawczy w obszarze porównań paradygmatów współbieżnych w nowoczesnych językach systemowych. Dostępne opracowania stanowią wartościowe punkty odniesienia i uzasadniają potrzebę kontynuacji prac empirycznych w tym zakresie, co znajduje swoje odzwierciedlenie także w niniejszej pracy. \\
W literaturze można znaleźć prace, które analizują różnice między Rustem a C++ w kontekście bezpieczeństwa, wydajności, zarządzania pamięcią oraz obsługi błędów.

\subsection{Bezpieczeństwo}
\label{Bezpieczeństwo}
Bezpieczeństwo języków Rust i C++ jest jednym z najczęściej analizowanych tematów w literaturze. W przypadku Rusta duży nacisk kładziony jest na eliminację całych klas błędów, takich jak dereferencja pustego wskaźnika, wyścigi danych oraz wycieki pamięci. Mechanizmy takie jak sprawdzanie własności i pożyczki oraz jawna mutowalność zmiennych \eng{explicit mutability} są wymieniane jako kluczowe elementy zapewniające bezpieczeństwo oraz minimalizując ryzyko wycieków pamięci \cite{MigratingCtoRustforMemorySafety}. 

Z drugiej strony, C++ umożliwia większą kontrolę nad pamięcią, co może być zaletą w systemach wymagających maksymalnej wydajności, ale jednocześnie wiąże się z koniecznością samodzielnego zarządzania zasobami przez programistów. W literaturze \cite{RustDifferences, RustDifferences1} często podkreśla się, że to właśnie większa złożoność i ryzyko błędów w kodzie C++ skłoniły społeczność do stworzenia języków takich jak Rust.

Przykładowo, badania \cite{RustSafety1, RustSafety2, RustSafety3} wskazują, że aplikacje napisane w Rust są mniej podatne na błędy związane z wyścigami danych \eng{data races}, co ma szczególne znaczenie w środowiskach wielowątkowych. Z kolei w C++ stosowanie bibliotek takich jak \texttt{std::thread} czy frameworków typu OpenMP pozwala na osiągnięcie podobnych celów, choć wymaga od programistów większej uwagi w zakresie synchronizacji. Dodatkowo są również prace \cite{PPL1_1,PPL1_2}, które przedstawiają próby implementacji mechanizmów wbudowanych w język Rust (prawo własności, pożyczka) do języka C.
\subsection{Czas wykonania}
\label{CzasWykonania}
Porównania czasów wykonania programów napisanych w Rust i C++ są częstym tematem analiz \cite{RustPerformance1, RustPerformance2, RustPerformance3, RustPerformance4}. W badaniach tych zostało wykazane, że pod względem wydajności Rust jest konkurencyjny wobec C++, co wynika z mechanizmów kompilacji i optymalizacji kodu.

Jednak kluczową różnicą jest to, że Rust wprowadza pewne narzuty związane z kontrolą bezpieczeństwa w czasie kompilacji, które mogą wydłużyć czas budowania programu, ale nie wpływają znacząco na czas wykonania.\\
Rust i C++ są językami kompilowanymi, co oznacza, że dedykowany kompilator tłumaczy kod źródłowy na kod maszynowy przed jego wykonaniem. Dzięki temu możliwe jest uzyskanie wysokiej wydajności programów. W literaturze \cite{Lesiński} często podkreśla się, że Rust, w odróżnieniu od C++, kładzie większy nacisk na bezpieczeństwo pamięci oraz typów w czasie kompilacji, co ma kluczowe znaczenie w nowoczesnym oprogramowaniu. W kontekście C++ wskazuje się na jego większą elastyczność oraz bogaty ekosystem, który pozwala na szeroką gamę zastosowań, ale jednocześnie wymaga większej uwagi programistów w zakresie zarządzania pamięcią i synchronizacji wątków.
% two images next to each other
\begin{figure}[H]
    \centering
    \begin{minipage}{.5\textwidth}
        \centering
        \includegraphics[height=12cm]{images/RustBuildsSteps.png}
        \caption{Kroki kompilacji w języku Rust}
        \label{fig:rust_build_steps}
    \end{minipage}%
    \begin{minipage}{.5\textwidth}
        \centering
        \includegraphics[height=12cm]{images/CppBuildsSteps.png}
        \caption{Kroki kompilacji w języku C++}
        \label{fig:cpp_build_steps}
    \end{minipage}
\end{figure}

Informacje o procesie kompilacji pochodzą z \cite{Lesiński, rustPolishNames, TheRustProgrammingLanguage}, które opisują integrację z LLVM i różnice w sprawdzaniu bezpieczeństwa.
Na rysunku \ref{fig:rust_build_steps} drugi blok reprezentuje dodatkowe etapy sprawdzania bezpieczeństwa w Rust, które nie występują w C++. Z kolei na rysunku \ref{fig:cpp_build_steps} drugi blok pokazuje podstawowe sprawdzanie typów w C++, które jest mniej rygorystyczne niż system Rust. Ze względu na różnicę w procesie kompulacji kodu powstają główne różnice w~bezpieczeństwie i wydajności obu języków.

C++ nadal pozostaje językiem preferowanym w projektach o krytycznym znaczeniu wydajnościowym, takich jak gry komputerowe, symulacje fizyczne czy systemy wbudowane, choć Rust zaczyna zdobywać popularność w tych obszarach ze względu na większe bezpieczeństwo przy porównywalnej wydajności. 

\subsection{Programowanie współbieżne oraz równoległe}
\label{Współbieżność}
Współbieżność i równoległość to kluczowe elementy programowania w  językach Rust i C++. Oba języki oferują zaawansowane narzędzia i biblioteki do zarządzania wielowątkowością.

Rust wyróżnia się systemem własności \eng{ownership} i wbudowanym mechanizmem wykrywania błędów współbieżności, co eliminuje wyścigi danych w czasie kompilacji. Narzędzia takie jak Tokio i Rayon pozwalają na łatwe tworzenie i zarządzanie zadaniami asynchronicznymi i równoległymi \cite{HandsOnConcurrencywithRust, RapidInnovationMasteringRust}.
C++ z kolei oferuje wsparcie dla wielowątkowości poprzez standardową bibliotekę (std::thread) oraz zaawansowane szkielety aplikacyjne (frameworks), takie jak OpenMP czy TBB (Threading Building Blocks). Chociaż te narzędzia są niezwykle potężne, nie zapewniają automatycznej ochrony przed błędami współbieżności, co wymaga większej ostrożności ze strony programistów.

Strona \cite{parallelrustcppIntroductionComparing} szczegółowo analizuje różnice w podejściu do współbieżności w obu językach, podkreślając, że Rust dzięki swojemu modelowi zarządzania pamięcią oferuje większe bezpieczeństwo, podczas gdy C++ pozostaje bardziej elastyczny, co może być korzystne w bardziej specyficznych scenariuszach. Jednak jak sam autor \cite{parallelrustcppIntroductionComparing} wskazuje, należy zwrócić uwagę, że większość sztuczek optymalizacyjnych pokazanych w tym porównaniu to jedynie adaptacje oryginalnych rozwiązań C++ w języku Rust. Koncentruje się ona na praktycznym porównaniu języków Rust i C++ pod względem równoległego przetwarzania, szczególnie na poziomie niskopoziomowych operacji i synchronizacji wątków. Znajdują się tam benchmarki pokazujące różnice w czasie wykonywania programów, narzędzia diagnostyczne oraz zestawienie wydajności w kontekście SIMD, wątków, oraz komunikacji międzypamięciowej.

W pracy akademickiej Brandefelta i Heymana \cite{heyman2020comparison} dokonano porównania wydajności oraz złożoności implementacyjnej aplikacji wielowątkowej napisanej w trzech językach: Rust, C++ oraz Java. Badanie to wykazało, że Rust oferuje zbliżoną wydajność do C++, przy jednoczesnym znacznym uproszczeniu kodu (mniejsza liczba linii kodu), co przekłada się na łatwiejsze utrzymanie i potencjalnie mniejszą podatność na błędy. Autorzy wskazują, że Rust, dzięki swojemu systemowi własności \eng{ownership}, eliminuje całe klasy błędów związanych z zarządzaniem pamięcią, bez konieczności stosowania odśmiecania.

Kolejne istotne opracowanie przedstawiono w publikacji Martinsa et al. \cite{martins2025npbrustnasparallelbenchmarks}, gdzie zaprezentowano implementację zestawu NAS (NASA Advanced Supercomputing) Parallel Benchmarks (NPB) \cite{nasaParallelBenchmarks} w języku Rust. Zestaw NPB stanowi uznany w środowisku naukowym standard do oceny wydajności systemów równoległych. W badaniach wykazano, że Rust osiąga porównywalną, a miejscami wyższą wydajność względem C++ i Fortranu (w wersjach OpenMP i~sekwencyjnych), co wskazuje na jego rosnący potencjał w dziedzinie wysokowydajnych obliczeń (HPC - High-Performance Computing). Jednocześnie zwrócono uwagę na trudności implementacyjne, związane z restrykcyjnym modelem bezpieczeństwa języka Rust, które jednak mogą być złagodzone poprzez zastosowanie odpowiednich bibliotek (np. Rayon).

Równolegle, w pracy Besozziego \cite{Besozzi} zaprezentowano bibliotekę Parallelo Parallel Library (PPL) - autorskie rozwiązanie do strukturalnego programowania równoległego w języku Rust. Biblioteka ta opiera się na wzorcach programowania równoległego takich jak szkielety \eng{skeletons} i wzorce projektowe, oferując wysokopoziomowe abstrakcje ułatwiające implementację złożonych aplikacji równoległych. Przeprowadzone testy wykazały, że PPL dorównuje lub przewyższa popularne biblioteki takie jak Rayon, jednocześnie zachowując zgodność z idiomami języka Rust i zasadą \textit{fearless concurrency}. Zastosowanie takich abstrakcji sprzyja zwiększeniu czytelności i poprawności kodu, co może mieć szczególne znaczenie w kontekście projektów systemowych.

Można również znaleźć pracę, która przedstawia wykorzystanie biblioteki odpowiedzialnej za współbieżność FastFlow przez oba języki Rust oraz C++ \cite{FastFlow}. Pokazuje ona, że język Rust jest dobrą alternatywą dla języka C++ w kontekście współbieżności.

% W odniesieniu do artykułów, czasopism oraz materiałów konferencyjnych opublikowanych w latach 2012 \footnotemark-2024, zidentyfikowanie prac, które jednoznacznie koncentrują się na problematyce pracy, jest wyzwaniem. Można natomiast znaleźć publikacje, które wykorzystują wspomniane języki (Rust oraz C++) i ich porównanie w kontekście złożonym do problemu pracy - chociażby wykorzystanie języka Rust w programowaniu układów wbudowanych \cite{ZamiennikWEmbedded}, jako zamiennik dla dotychczasowych języków z rodziny C.\\
% 
% \footnotetext{Rust został zaprezentowany w 2012 roku \cite{wikipediaRustprogramming}, a C++ w 1985 roku \cite{wikipediaWikipedia}.}


\section{Podsumowanie}
Na podstawie przeglądu literatury oraz zadanych pytań do przeglądu literatury można wskazać następujące odpowiedzi:
\subsubsection{PPL1: Jakie główne koncepcje/teorie dominują w literaturze dotyczącej porównania języków Rust oraz C++?}
W literaturze dominują badania porównawcze bezpieczeństwa, wydajności oraz zarządzania pamięcią w językach Rust i C++ - bardziej szczegółowo opisane w podrozdziałach \ref{Bezpieczeństwo} oraz \ref{CzasWykonania}. W kontekście współbieżności i równoległości można zauważyć znaczący wzrost prac w ubiegłych latach, co przedstawia przegląd opisany w podrozdziale \ref{Współbieżność}.

\subsubsection{PPL2: Jakie metody badawcze są najczęściej stosowane do analizy różnic pomiędzy językami ?}
W literaturze oraz w dotychczasowych analizach \cite{LanguageComparison_1,LanguageComparison_2,LanguageComparison_3,LanguageComparison_4} wskazano szereg kryteriów istotnych przy porównywaniu lub ocenie języków programowania ogólnego przeznaczenia. Do najważniejszych należą: 
    \begin{itemize}
        \item Prostota konstrukcji języka, mająca bezpośredni wpływ na łatwość programowania i zrozumiałość kodu
        \item Czytelność kodu, związana z jego późniejszą konserwacją i rozwijaniem
        \item Dostosowanie języka do konkretnego zastosowania, co wpływa na wydajność i efektywność programów
        \item Szybkość kompilacji
        \item Efektywność działania programu, zarówno pod względem szybkości, jak i zużycia zasobów systemowych,
        \item Dostępność i jakość bibliotek, frameworków oraz narzędzi wspierających rozwój oprogramowania
        \item Wsparcie w procesie debugowania, profilowania i testowania kodu
        \item Bezpieczeństwo języka, związane z eliminacją błędów w czasie kompilacji oraz wykrywaniem zagrożeń w trakcie działania programu
        \item Długowieczność języka oraz narzędzi kompilacyjnych, co wpływa na stabilność i przewidywalność rozwoju oprogramowania
        \item Przenośność między różnymi platformami i architekturami sprzętowymi.
    \end{itemize}
    Wszystkie te kryteria mają istotny wpływ na całkowity koszt i nakład pracy związany z tworzeniem oraz utrzymaniem oprogramowania, a także na jego jakość i przydatność w długim okresie użytkowania.

    Doświadczenie wskazuje również, że te same kryteria można stosować do oceny innych komponentów wspomagających proces tworzenia oprogramowania, takich jak biblioteki klas obiektowych czy projekty typów abstrakcyjnych. Wykorzystanie odpowiednich, zewnętrznych komponentów oraz dobrze zaprojektowanych rozwiązań przyczynia się do poprawy czytelności, utrzymywalności i ogólnej jakości kodu, jednocześnie przyspieszając proces jego tworzenia.

\subsubsection{PPL3: Jak wygląda porównanie dostępności i dojrzałości bibliotek do programowania współbieżnego i równoległego w obu językach?}
Porównując dostępność i dojrzałość bibliotek do programowania współbieżnego i równoległego w językach Rust i C++, zauważyć można istotne różnice wynikające zarówno z historii rozwoju tych języków, jak i ich podejścia do bezpieczeństwa, abstrakcji oraz zarządzania zasobami.

W przypadku języka C++, biblioteki takie jak OpenMP, Intel TBB czy pthreads cechują się dużą dojrzałością oraz szerokim zastosowaniem w przemyśle, zwłaszcza w kontekście obliczeń naukowych, symulacji fizycznych i systemów o wysokiej wydajności. Są one dobrze udokumentowane, posiadają wsparcie komercyjne (np. TBB) oraz charakteryzują się dużą kompatybilnością z istniejącą infrastrukturą sprzętową i programową. Niemniej jednak, wymagają od programisty głębokiej wiedzy w zakresie zarządzania pamięcią oraz synchronizacji, co przekłada się na wyższy próg wejścia i podatność na błędy (np. wyścigi danych czy zakleszczenia).

Z kolei Rust, jako język nowszy, oferuje bardziej nowoczesny zestaw narzędzi do programowania równoległego, w tym biblioteki takie jak Tokio, Rayon czy Crossbeam. Mimo mniejszej liczby lat rozwoju, biblioteki te szybko dojrzewają i zdobywają popularność dzięki silnym gwarancjom bezpieczeństwa pamięci na poziomie kompilatora oraz ergonomicznemu API. Rust promuje bezpieczne podejście do współbieżności poprzez system własności \eng{ownership} i brak domyślnego współdzielenia zasobów, co w praktyce eliminuje całą klasę błędów typowych dla C++. Dzięki temu biblioteki w Rust, choć mniej rozpowszechnione w starszych zastosowaniach, zyskują przewagę w projektach tworzonych od podstaw, zwłaszcza w środowiskach wymagających wysokiej niezawodności.

\subsubsection{PPL4: Czy istnieją systematyczne metodologie porównywania języków programowania w kontekście problemu pracy, które można zastosować do analizy Rust i C++?}

W literaturze przedmiotu można znaleźć prace proponujące sformalizowane, systematyczne metodologie służące do porównywania języków programowania w kontekście ich wsparcia dla współbieżności i równoległości. Jednakzę zdecydowana większość badań skupia się na aspektach wydajnościowych, bezpieczeństwa pamięci lub ekspresyjności języka, dodatkowo często przyjmują one podejścia ad hoc, oparte na wybranych przypadkach użycia, bez spójnych ram metodologicznych.

Niektóre opracowania, jak np. porównania publikowane w ramach blogów technicznych czy artykułów na platformach takich jak Medium, wykorzystują podejścia oparte na konteneryzacji programów testowych i~ich uruchamianiu na jednorodnej infrastrukturze. Przykładowo, w~metodzie \textit{Rainbow} \cite{rainbow} badana jest przepustowość obliczeń przy wykorzystaniu kontenerów i~zewnętrznego systemu, jak Redis, do monitorowania wyników. Choć takie podejścia są ciekawe, narażone są na zakłócenia wynikające z różnic w czasie inicjalizacji kontenerów, rozmiarze obrazu czy użyciu pamięci, co może zafałszowywać końcowe wyniki porównawcze.

Znacznie bardziej sformalizowanym podejściem jest wykorzystanie ustandaryzowanych pakietów benchmarkowych, takich jak NAS Parallel Benchmarks (NPB), które zostały opracowane przez NASA w celu obiektywnej oceny wydajności systemów wieloprocesorowych. Przykładem zastosowania tej metodologii jest praca „NPB-Rust” \cite{martins2025npbrustnasparallelbenchmarks}, w której autorzy przeprowadzili systematyczne porównanie implementacji benchmarków NAS w języku Rust (z użyciem biblioteki Rayon) oraz C++ (z użyciem OpenMP). Ocenie poddano nie tylko wydajność obliczeniową, ale również skalowalność, zużycie pamięci oraz nakład implementacyjny, mierząc m.in. liczbę linii kodu i szacowany koszt harmonogramowania według modelu COCOMO - \eng{Constructive Cost Model} - to model szacowania kosztów związanych z rozwojem oprogramowania.

Takie podejścia, oparte na zweryfikowanych zestawach testowych, umożliwiają bardziej wiarygodne i powtarzalne porównania pomiędzy językami programowania w kontekście współbieżności i równoległości. W związku z tym, zastosowanie frameworków benchmarkowych takich jak NPB powinno być traktowane jako wzorzec dla przyszłych badań w tej dziedzinie.

\subsubsection{PPL5: Jakie aspekty programowania współbieżnego i równoległego w Rust i C++ nie zostały dostatecznie zbadane w literaturze?}
Dotychczasowe badania naukowe dotyczące programowania współbieżnego w językach Rust i C++ koncentrowały się przede wszystkim na ogólnych aspektach, takich jak bezpieczeństwo pamięci, kontrola dostępu do danych, ergonomia kodu oraz wydajność kompilacji i wykonania. W szczególności analizowano podejście języków do eliminacji błędów wykonawczych (np. data races), mechanizmy typizacji oraz zarządzanie zasobami systemowymi. Przykładem mogą być prace \cite{heyman2020comparison}, które zestawiają Rust i C++ w kontekście praktycznych implementacji aplikacji wielowątkowych, czy badania \cite{martins2025npbrustnasparallelbenchmarks}, porównujące implementacje benchmarków równoległych.

Niemniej jednak, najnowsze publikacje wskazują na wypełnienie wielu wcześniej istniejących luk, zwłaszcza w zakresie porównania konkretnych konstrukcji językowych \linebreak (np. \mbox{async/await} kanały, skeletons) oraz empirycznej weryfikacji wydajności bibliotek wspierających równoległość (np. PPL, Rayon, OpenMP).

Pomimo to, w literaturze wciąż można zidentyfikować kilka istotnych obszarów badawczych, które pozostają niewystarczająco zbadane:
\begin{itemize}
    \item \textbf{Wpływ architektury sprzętowej na zachowanie systemów współbieżnych:}\\
    Większość dotychczasowych badań prowadzono na klasycznych architekturach x86\_64. Brakuje natomiast analiz zachowania aplikacji wielowątkowych w Rust i C++ na alternatywnych platformach, takich jak ARM, RISC-V czy systemy heterogeniczne (np. SoC zawierające zarówno CPU, jak i akceleratory). Tego typu analizy są szczególnie istotne w kontekście rozwoju systemów wbudowanych oraz IoT, gdzie Rust zyskuje coraz większą popularność.
    \item \textbf{Koszt abstrakcji oraz narzut związany z modelem bezpieczeństwa Rust:}\\
    Choć bezpieczeństwo współbieżności w Rust jest jego kluczowym atutem, literatura rzadko podejmuje próbę precyzyjnego oszacowania narzutu wydajnościowego wynikającego z jego rygorystycznego modelu własności i sprawdzania pożyczeń. Istnieje potrzeba eksperymentalnych badań porównawczych, które wykazałyby, w jakim stopniu koszt ten wpływa na skalowalność aplikacji w środowiskach o dużym współczynniku równoległości.
    \item \textbf{Porównanie ergonomii rozwiązań współbieżnych na poziomie idiomatycznym i bibliotekowym:}\\
    Chociaż wiele prac omawia techniczne możliwości języków (np. \texttt{std::thread}, Tokio, OpenMP, \texttt{std::thread} w C++), nadal brakuje badań jakościowych dotyczących ergonomii kodu, łatwości diagnostyki błędów oraz odporności na błędy logiczne podczas implementacji systemów wielowątkowych. Takie analizy mogłyby obejmować porównania idiomatycznych podejść (np. actor model, CSP, fork-join), oferowanych przez biblioteki Rust i C++.
    \item \textbf{Zachowanie systemów współbieżnych w warunkach zmiennego obciążenia i konkurencyjnego dostępu:}\\
    Istnieje luka w badaniach dotyczących stabilności i odporności aplikacji na skoki obciążenia lub dynamiczną alokację wątków. Potrzebne są badania stresowe i profilowanie systemów w~warunkach rzeczywistej konkurencji (np. serwery HTTP, silniki obliczeniowe), co pozwoliłoby ocenić adaptacyjność strategii planowania i zarządzania wątkami w Rust i C++.
    \item \textbf{Weryfikacja formalna i modelowanie błędów współbieżności:}\\
    Pomimo iż język Rust oferuje silne gwarancje bezpieczeństwa na etapie kompilacji - statyczna gwarancja bezpieczeństwa, zagadnienia związane z formalną weryfikacją własności współbieżnych, takich jak żywotność \eng{liveness}, brak zagłodzenia \eng{starvation-freedom} czy deterministyczność wykonania, pozostają w literaturze stosunkowo słabo zbadane. Potencjalnie wartościowym kierunkiem dalszych analiz byłoby porównanie dostępnych narzędzi formalnych, takich jak weryfikatory modeli, w kontekście języka Rust oraz innych języków programowania współbieżnego. Tego rodzaju zestawienie mogłoby przyczynić się do lepszego zrozumienia możliwości i ograniczeń istniejących podejść do formalnej weryfikacji w~środowiskach wielowątkowych.
\end{itemize}


\subsubsection{PPL6: Jaki jest stan wiedzy na temat wykorzystania programowania współbieżnego w ramach GPU w językach Rust i C++?}
W literaturze przedmiotowej oraz w praktyce programistycznej, zagadnienie wykorzystania programowania współbieżnego na GPU ewoluuje w różnym tempie w zależności od języka programowania. W przypadku języka Rust obserwujemy dynamiczny rozwój narzędzi, które umożliwiają wykorzystanie mocy obliczeniowej kart graficznych przy jednoczesnym zachowaniu wysokich gwarancji bezpieczeństwa pamięci i wątków.

Projekty takie jak rust-gpu \cite{rustgpuRust} dążą do umożliwienia pisania programów cieniujących w języku Rust, oferując podejście, które integruje możliwości GPU z bezpiecznymi konstrukcjami języka. Dokumentacja dostępna na stronie github \cite{rustgpuRust} wskazuje na intensywne prace nad przeniesieniem wielu korzyści wynikających z systemu własności i typizacji Rust na środowisko GPU, co może przyczynić się do zmniejszenia ryzyka błędów współbieżności, które są szczególnie krytyczne w obliczeniach równoległych.

Równolegle, biblioteka Vulkano (dostępna m.in. poprzez dokumentację \cite{docsVulkanoRust}) stanowi wysokopoziomowy, bezpieczny interfejs do API Vulkan, które jest standardem w programowaniu GPU. Vulkano umożliwia abstrakcję złożoności niskopoziomowych interfejsów, jednocześnie oferując możliwość pełnego wykorzystania równoległości GPU. W podobnym nurcie znajduje się projekt wgpu, który implementuje standard WebGPU, umożliwiając przenośne aplikacje graficzne i obliczeniowe, a jednocześnie integrując współczesne podejścia do zarządzania zasobami i synchronizacji.

W przeciwieństwie do podejścia Rust, w środowisku C++ dominującym rozwiązaniem jest platforma CUDA, rozwijana i wspierana przez firmę NVIDIA \cite{nvidiaCUDAToolkit}. CUDA oferuje bardzo dojrzały, zoptymalizowany i szeroko stosowany framework, który pozwala na bezpośrednią implementację algorytmów równoległych na GPU. W odróżnieniu od narzędzi rozwijanych dla Rust, CUDA posiada ugruntowaną pozycję w środowisku przemysłowym, co przekłada się na bogatą dokumentację, szerokie wsparcie techniczne oraz liczne biblioteki wspomagające rozwój aplikacji wykorzystujących GPU.

Podsumowując, stan wiedzy dotyczący programowania współbieżnego na GPU w języku Rust znajduje się na etapie intensywnego rozwoju i eksperymentacji, gdzie projekty takie jak rust-gpu, Vulkano oraz wgpu \cite{wgpuWgpuPortable} pokazują potencjał tego podejścia, zwłaszcza w kontekście bezpieczeństwa i nowoczesnych abstrakcji. Z kolei w C++ platforma CUDA, dzięki swojej dojrzałości oraz szerokiemu wsparciu ze strony przemysłu, pozostaje głównym narzędziem wykorzystywanym do implementacji wysokowydajnych obliczeń równoległych na GPU.

\subsection{Kierunki dalszych badań}
Na podstawie przeglądu literatury oraz odpowiedzi na pytania do przeglądu literatury można wskazać na kilka kierunków dalszych badań w dziedzinie programowania współbieżnego i równoległego w językach Rust i C++:
\begin{itemize}
    \item \textbf{Rozszerzone porównania systematyczne mechanizmów współbieżnych i równoległych} — z uwzględnieniem nie tylko wydajności, ale także ergonomii, bezpieczeństwa pamięci, ekspresyjności kodu oraz kosztu implementacyjnego.
    \item \textbf{Analiza porównawcza konkretnych mechanizmów i bibliotek} — takich jak \texttt{std::thread} w C++ i Rust, \texttt{OpenMP} a \texttt{Rayon}, czy też mechanizmy asynchroniczne (\texttt{async/await} w Rust a \texttt{futures}, \texttt{std::coroutine} w C++20 i nowszych).
    \item \textbf{Analiza wpływu zastosowania unsafe code w Rust} — w kontekście uzyskiwanej wydajności i kompromisów względem bezpieczeństwa pamięci oraz czytelności kodu.
    \item \textbf{Zastosowania programowania równoległego na GPU} — porównanie możliwości Rust (np. poprzez biblioteki takie jak \texttt{rust-gpu} czy \texttt{wgpu}) z rozwiązaniami dostępnymi w ekosystemie C++ (np. CUDA, SYCL, OpenCL).
    \item \textbf{Badanie wpływu konstrukcji językowych na podatność na błędy współbieżności} — \mbox{np. warunki} wyścigu, zakleszczenia, użycie nieprawidłowych referencji lub wskaźników, z~uwzględnieniem poziomu ochrony oferowanego przez kompilator.
\end{itemize}

\chapter{Przegląd literatury}
Celem niniejszego rozdziału jest przedstawienie dotychczasowych badań i publikacji dotyczących mechanizmów programowania współbieżnego i równoległego w językach Rust i C++. Analiza literatury umożliwi zrozumienie aktualnego stanu wiedzy w tej dziedzinie, a także wskazanie na występujące luki badawcze, które niniejsza praca postara się wypełnić.
Na samym wstępie zostały postawione następujące pytania do przeglądu literatury, które pomogą zrozumieć oraz sprawdzić aktualny stan wiedzy jeżeli chodzi o porównanie języków Rust oraz C++:

\begin{quote}
    \item \textbf{PPL1:} \emph{Jakie główne koncepcje/teorie dominują w literaturze dotyczącej porównania języków Rust oraz C++?} \label{PPL1}
    \item \textbf{PPL2:} \emph{Jakie metody badawcze są najczęściej stosowane do analizy różnic pomiędzy językami?}
    \item \textbf{PPL3:} \emph{Jak wygląda porównanie dostępności i dojrzałości bibliotek do programowania współbieżnego i równoległego w obu językach?}
    \item \textbf{PPL4:} \emph{Czy istnieją systematyczne metodologie porównywania języków programowania w kontekście współbieżności, które można zastosować do analizy Rust i C++?}
    \item \textbf{PPL5:} \emph{Jakie aspekty programowania współbieżnego i równoległego w Rust i C++ nie zostały dostatecznie zbadane w literaturze?}
    \item \textbf{PPL6:} \emph{Jaki jest stan wiedzy na temat wykorzystania programowania współbieżnego w ramach GPU w językach Rust i C++?}
\end{quote}

Odpowiedzi na powyższe pytania pozwolą na zidentyfikowanie kluczowych obszarów, które wymagają dalszych badań oraz na wskazanie na potencjalne kierunki rozwoju w dziedzinie programowania współbieżnego i równoległego w językach Rust i C++.\\
\subsection{Metodologia przeglądu literatury}
Proces przeglądu literatury został zrealizowany zgodnie z zasadami przeglądu systematycznego, co oznaczało zastosowanie jasno określonych kryteriów selekcji i wyłączenia. Główne źródła literaturowe obejmowały artykuły naukowe, materiały konferencyjne oraz dokumentację techniczną. Wyszukiwanie przeprowadzono w renomowanych bazach danych naukowych oraz repozytoriach zawierających publikacje z zakresu inżynierii oprogramowania i języków programowania. Dodatkowo zostały również uwzględnione źródła internetowe oraz dokumentacje techniczne.\\
Przegląd literatury odbywał się z wykorzystaniem narzędzi baz danych oferujących wyszukiwanie, filtrowanie oraz przegląd prac: Scopus, Google Scholar.\\
\subsection{Baza Scopus}
W ramach bazy scopus wykorzystano następujące kwerendy do wyszukiwania - tabela \ref{table:literatureReviewQueries}

\begin{table}[H]
    \caption{Kwerendy użyte w bazie Scopus \protect \footnotemark}
    \label{table:literatureReviewQueries}
    \begin{tabular}{|c|p{11cm}|c|}
    \hline
    Lp. & Kwerenda & Liczba wyników \\ \hline
    1 & ALL ("concurrent programming"\ OR "parallel programming") AND (ALL ("Rust") AND ALL ("C++")) & 444 \\ \hline

    2 & ALL ("concurrent programming"\ OR "parallel programming") AND (ALL ("Rust") AND ALL ("C++") ) AND ( ALL ("compare")) & 28 \\ \hline

    3 & (TITLE-ABS-KEY(("concurrent programming"\ OR "parallel programming") AND ("Rust"\ AND "C++"))) AND (TITLE-ABS-KEY("comparison"\ OR "evaluation"\ OR "benchmark")) & 6 \\ \hline

    4 & (TITLE-ABS-KEY(("thread"\ OR "async"\ OR "future"\ OR "actor model"\ OR "message passing"\ OR "shared memory") AND ("Rust"\ AND "C++"))) AND (TITLE-ABS-KEY("comparison"\ OR "performance"\ OR "evaluation")) & 50 \\ \hline

    5 & (TITLE-ABS-KEY(("Rust"\ AND "C++") AND ("concurrency model"\ OR "parallel constructs"\ OR "multithreading"))) AND (TITLE-ABS-KEY("comparison"\ OR "study")) & 2 \\ \hline

    \end{tabular}
\end{table}
\footnotetext{Ilość wyników dla poszczególnych zapytań może się różnić w zależności od daty (wyszukiwanie przeprowadzono w okresie listopad-luty 2024/25).}

W celu identyfikacji literatury związanej z porównaniem wybranych mechanizmów programowania współbieżnego i równoległego w językach Rust i C++, opracowano pięć zapytań w bazie Scopus, z których każde miało określony cel badawczy. Pierwsze zapytanie miało na celu uzyskanie ogólnego przeglądu literatury, wyszukując wszystkie dokumenty, w których występują jednocześnie zagadnienia programowania współbieżnego lub równoległego oraz języki Rust i C++, niezależnie od kontekstu. Pozwoliło to oszacować ogólną skalę badań łączących te zagadnienia. Drugie zapytanie zawężało zakres wyszukiwania poprzez dodanie słowa kluczowego „compare”, co umożliwiło wyodrębnienie publikacji, w których dokonano bezpośredniego porównania języków Rust i C++ w kontekście współbieżności lub równoległości. Dzięki temu uzyskano bardziej ukierunkowany zbiór literatury odnoszącej się do analizy porównawczej. Trzecie zapytanie charakteryzowało się większą precyzją, ograniczając wyniki do tytułów, streszczeń oraz słów kluczowych, i uwzględniało wyłącznie publikacje zawierające odniesienia do ewaluacji, porównań bądź benchmarków języków Rust i C++. Takie podejście pozwoliło wyselekcjonować najbardziej tematycznie powiązane prace. Czwarte zapytanie miało charakter bardziej techniczny, koncentrując się na konkretnych mechanizmach współbieżności, takich jak wątki, asynchroniczność, obiekty typu futury, model aktorów, przesyłanie komunikatów czy pamięć współdzielona, w połączeniu z terminami dotyczącymi wydajności i oceny. Umożliwiło to dotarcie do badań analizujących niskopoziomowe aspekty działania tych mechanizmów w obu językach. Piąte zapytanie skupiało się na poziomie koncepcyjnym, wyszukując publikacje zawierające takie terminy jak model współbieżności, konstrukty równoległe czy wielowątkowość, wraz z frazami dotyczącymi porównań lub analiz. Celem było zidentyfikowanie prac badających różnice w podejściu do współbieżności na poziomie architektury języka i jego konstrukcji wewnętrznych.

Autor zdecydował się również użyć nowego, wbudowane narzędzia w systemie Scopus - Scopus AI. Narzędzie to oparte na sztucznej inteligencji, wspomaga eksplorację akademicką w oparciu o dane z platformy Scopus. Dzięki integracji z narzędziem Copilot optymalizuje wyszukiwania, łącząc metody semantyczne i dopasowanie słów kluczowych. Choć Scopus AI ułatwia badania, jego wyniki warto weryfikować, ponieważ mogą zawierać nieścisłości lub stronniczość. \\
Po wprowadzeniu tytułu pracy w języku angielskim jako kwerendę, Scopus AI zwrócił 9 wyników, biorąc pod uwagę kwerendę stworzoną na podstawie tytyułu pracy, zamieszczoną w listingu \ref{AIQuery}. Zwrócone prace pokrywają się z przeglądem umieszczonym w tabeli \ref{table:literatureReviewQueries} 

\lstset{breaklines=true}
\begin{lstlisting}[caption=Kwerenda wygenerowana przez AI, label=AIQuery]
("concurrent programming" OR "parallel programming" OR "multithreading" OR "asynchronous")
AND ("Rust" OR "C++" OR "programming languages" OR "software development")
AND ("performance" OR "efficiency" OR "scalability" OR "resource management")
AND ("synchronization" OR "thread safety" OR "deadlock" OR "race condition")
AND ("libraries" OR "frameworks" OR "tools" OR "APIs")
\end{lstlisting}

\subsection{Kryteria selekcji oraz wyłączenia}
W procesie selekcji literatury uwzględniano przede wszystkim publikacje wydane po 2012 roku, co wynika z faktu, iż w tym właśnie roku zadebiutował język Rust \cite{wikipediaRustprogramming}. Wyjątek stanowiły prace o charakterze ogólnym lub takie, które nie odnosiły się bezpośrednio do języka Rust, lecz zawierały istotne informacje dla problematyki badawczej niniejszej pracy.

Analiza obejmowała literaturę w języku polskim oraz angielskim, przy czym zdecydowana większość źródeł stanowiły publikacje anglojęzyczne. Selekcja materiałów opierała się na zgodności tematycznej z zakresem badań. W przypadku wątpliwości co do adekwatności danej pozycji, decyzja o jej włączeniu do przeglądu podejmowana była na podstawie analizy streszczenia. Jeśli po tej analizie publikacja wydawała się istotna, przechodzono do pełnej oceny jej treści.

Publikacje, które po dogłębnej analizie okazywały się nieodpowiednie dla głównego problemu badawczego, nie były uwzględniane w zasadniczej części pracy. Niemniej jednak, jeśli przyczyniły się do lepszego zrozumienia badanego zagadnienia lub pomogły w odpowiedzi na pytania do przeglądu literatury \ref{PPL1}, były one odnotowywane jako materiały pomocnicze. Prace niespełniające powyższych kryteriów lub te, które nie są dostępne za pośrednictwem dostępnych metod (bądź też braku odpowiedzi twórców o prośbę udostępnienia pracy) były wykluczane z dalszej analizy.

\section{Porównanie Rust oraz C++}
Porównanie języków programowania Rust i C++ jest przedmiotem licznych publikacji, które analizują ich różnorodne aspekty, takie jak struktura kodu, sposób kompilacji, bezpieczeństwo, wydajność oraz obsługa współbieżności i równoległości. Niestety wśród prac naukowych zidentyfikowanych w bazach danych Scopus oraz Google Scholar, nie znaleziono prac, które jednoznacznie porównywałyby oba języki w kontekście programowania współbieżnego i równoległego.\\
W literaturze można znaleźć prace, które analizują różnice między Rustem a C++ w kontekście bezpieczeństwa, wydajności, zarządzania pamięcią oraz obsługi błędów.

\subsection{Bezpieczeństwo}
\label{Bezpieczeństwo}
Bezpieczeństwo języków Rust i C++ jest jednym z najczęściej analizowanych tematów w literaturze. W przypadku Rusta duży nacisk kładziony jest na eliminację całych klas błędów, takich jak null pointer dereferencing, data races oraz wycieki pamięci. Mechanizmy takie jak ownership, borrow checker oraz obowiązkowa mutowalność zmiennych (explicit mutability) są wymieniane jako kluczowe elementy zapewniające bezpieczeństwo oraz minimalizując ryzyko wycieków pamięci \cite{MigratingCtoRustforMemorySafety}. 

Z drugiej strony, C++ umożliwia większą kontrolę nad pamięcią, co może być zaletą w systemach wymagających maksymalnej wydajności, ale jednocześnie wiąże się z koniecznością samodzielnego zarządzania zasobami przez programistów. W literaturze \cite{RustDifferences, RustDifferences1} często podkreśla się, że to właśnie większa złożoność i ryzyko błędów w kodzie C++ skłoniły społeczność do stworzenia języków takich jak Rust.

Przykładowo, badania \cite{RustSafety1, RustSafety2, RustSafety3} wskazują, że aplikacje napisane w Rusta są mniej podatne na błędy związane z wyścigami danych \eng{data races}, co ma szczególne znaczenie w środowiskach wielowątkowych. Z kolei w C++ stosowanie bibliotek takich jak std::thread czy frameworków typu OpenMP pozwala na osiągnięcie podobnych celów, choć wymaga od programistów większej uwagi w zakresie synchronizacji. Dodatkowo są również prace \cite{PPL1_1,PPL1_2}, które przedstawiają próby implementacji mechanizmów wbudowanych w język Rust (prawo własności, pożyczka) do języka C.
\subsection{Czas wykonania}
\label{CzasWykonania}
Porównania czasów wykonania programów napisanych w Rust i C++ są częstym tematem analiz \cite{RustPerformance1, RustPerformance2, RustPerformance3, RustPerformance4}. W badaniach tych zostało wykazane, że pod względem wydajności Rust jest konkurencyjny wobec C++, co wynika z mechanizmów kompilacji i optymalizacji kodu.

Jednak kluczową różnicą jest to, że Rust wprowadza pewne narzuty związane z kontrolą bezpieczeństwa w czasie kompilacji, które mogą wydłużyć czas budowania programu, ale nie wpływają znacząco na czas wykonania.\\
Rust i C++ są językami kompilowanymi, co oznacza, że dedykowany kompilator tłumaczy kod źródłowy na kod maszynowy przed jego wykonaniem. Dzięki temu możliwe jest uzyskanie wysokiej wydajności programów. W literaturze \cite{Lesiński} często podkreśla się, że Rust, w odróżnieniu od C++, kładzie większy nacisk na bezpieczeństwo pamięci oraz typów w czasie kompilacji, co ma kluczowe znaczenie w nowoczesnym oprogramowaniu. W kontekście C++ wskazuje się na jego większą elastyczność oraz bogaty ekosystem, który pozwala na szeroką gamę zastosowań, ale jednocześnie wymaga większej uwagi programistów w zakresie zarządzania pamięcią i synchronizacji wątków.
% two images next to each other
\begin{figure}[H]
    \centering
    \begin{minipage}{.5\textwidth}
        \centering
        \includegraphics[height=12cm]{images/RustBuildsSteps.png}
        \caption{Kroki kompilacji w języku Rust}
        \label{fig:rust_build_steps}
    \end{minipage}%
    \begin{minipage}{.5\textwidth}
        \centering
        \includegraphics[height=12cm]{images/CppBuildsSteps.png}
        \caption{Kroki kompilacji w języku C++}
        \label{fig:cpp_build_steps}
    \end{minipage}
    \caption{Porównanie kroków kompilacji w językach Rust i C++}
\end{figure}

Informacje o procesie kompilacji pochodzą z \cite{Lesiński, rustPolishNames, TheRustProgrammingLanguage}, które opisują integrację z LLVM i różnice w sprawdzaniu bezpieczeństwa.
Na diagramie \ref{fig:rust_build_steps} drugi blok reprezentuje dodatkowe etapy sprawdzania bezpieczeństwa w Rust, które nie występują w C++. Z kolei na diagramie \ref{fig:cpp_build_steps} drugi blok pokazuje podstawowe sprawdzanie typów w C++, które jest mniej rygorystyczne niż system Rust. Ze względu na różnicę w procesie kompulacji kodu powstają główne różnice w bezpieczeństwie i wydajności obu języków.

C++ nadal pozostaje językiem preferowanym w projektach o krytycznym znaczeniu wydajnościowym, takich jak gry komputerowe, symulacje fizyczne czy systemy wbudowane, choć Rust zaczyna zdobywać popularność w tych obszarach ze względu na większe bezpieczeństwo przy porównywalnej wydajności. 

\subsection{Programowanie współbieżne oraz równoległe}
Współbieżność i równoległość to kluczowe elementy programowania w  językach Rust i C++. Oba języki oferują zaawansowane narzędzia i biblioteki do zarządzania wielowątkowością.

Rust wyróżnia się systemem własności \eng{ownership} i wbudowanym mechanizmem wykrywania błędów współbieżności, co eliminuje wyścigi danych w czasie kompilacji. Narzędzia takie jak Tokio i Rayon pozwalają na łatwe tworzenie i zarządzanie zadaniami asynchronicznymi i równoległymi.
C++ z kolei oferuje wsparcie dla wielowątkowości poprzez standardową bibliotekę (std::thread) oraz zaawansowane szkielety aplikacyjne (frameworks), takie jak OpenMP czy TBB (Threading Building Blocks). Chociaż te narzędzia są niezwykle potężne, nie zapewniają automatycznej ochrony przed błędami współbieżności, co wymaga większej ostrożności ze strony programistów.

Strona \cite{parallelrustcppIntroductionComparing} szczegółowo analizuje różnice w podejściu do współbieżności w obu językach, podkreślając, że Rust dzięki swojemu modelowi zarządzania pamięcią oferuje większe bezpieczeństwo, podczas gdy C++ pozostaje bardziej elastyczny, co może być korzystne w bardziej specyficznych scenariuszach. Jednak jak sam autor \cite{parallelrustcppIntroductionComparing} wskazuje, należy zwrócić uwagę, że większość sztuczek optymalizacyjnych pokazanych w tym porównaniu to jedynie adaptacje oryginalnych rozwiązań C++ w języku Rust. Koncentruje się ona na praktycznym porównaniu języków Rust i C++ pod względem równoległego przetwarzania, szczególnie na poziomie niskopoziomowych operacji i synchronizacji wątków. Znajdują się tam benchmarki pokazujące różnice w czasie wykonywania programów, narzędzia diagnostyczne oraz zestawienie wydajności w kontekście SIMD, wątków, oraz komunikacji międzypamięciowej.

W odniesieniu do artykułów, czasopism oraz materiałów konferencyjnych opublikowanych w latach 2012 \footnotemark-2024, zidentyfikowanie prac, które jednoznacznie koncentrują się na problematyce pracy, jest wyzwaniem. Można natomiast znaleźć publikacje, które wykorzystują wspomniane języki (Rust oraz C++) i ich porównanie w kontekście złożonym do problemu pracy - chociażby wykorzystanie języka Rust w programowaniu układów wbudowanych \cite{ZamiennikWEmbedded}, jako zamiennik dla dotychczasowych języków z rodziny C.\\
Można również znaleźć pracę, która przedstawia wykorzystanie biblioteki odpowiedzialnej za współbieżność FastFlow przez oba języki Rust oraz C++ \cite{FastFlow}. Pokazuje ona, że język Rust jest dobrą alternatywą dla języka C++ w kontekście współbieżności.
\footnotetext{Rust został zaprezentowany w 2012 roku \cite{wikipediaRustprogramming}, a C++ w 1985 roku \cite{wikipediaWikipedia}.}


\section{Podsumowanie}
Na podstawie przeglądu literatury oraz zadanych pytań do przeglądu literatury można wskazać na następujące odpowiedzi
\subsubsection{PPL1: Jakie główne koncepcje/teorie dominują w literaturze dotyczącej porównania języków Rust oraz C++?}
W literaturze dominują badania porównawcze bezpieczeństwa, wydajności oraz zarządzania pamięcią w językach Rust i C++ \cite{PPL1_1} - szczegółowo opisane w podrozdziałach \ref{Bezpieczeństwo} oraz \ref{CzasWykonania}. W kontekście współbieżności i równoległości brakuje systematycznych analiz porównawczych.

\subsubsection{PPL2: Jakie metody badawcze są najcz
ęściej stosowane do analizy różnic pomiędzy językami ?}
W literaturze oraz w dotychczasowych analizach \cite{LanguageComparison_1,LanguageComparison_2,LanguageComparison_3,LanguageComparison_4} wskazano szereg kryteriów istotnych przy porównywaniu lub ocenie języków programowania ogólnego przeznaczenia. Do najważniejszych należą: 
    \begin{itemize}
        \item Prostota konstrukcji języka, mająca bezpośredni wpływ na łatwość programowania i zrozumiałość kodu
        \item Czytelność kodu, związana z jego późniejszą konserwacją i rozwijaniem
        \item Dostosowanie języka do konkretnego zastosowania, co wpływa na wydajność i efektywność programów
        \item Szybkość kompilacji
        \item Efektywność działania programu, zarówno pod względem szybkości, jak i zużycia zasobów systemowych,
        \item Dostępność i jakość bibliotek, frameworków oraz narzędzi wspierających rozwój oprogramowania
        \item Wsparcie w procesie debugowania, profilowania i testowania kodu
        \item Bezpieczeństwo języka, związane z eliminacją błędów w czasie kompilacji oraz wykrywaniem zagrożeń w trakcie działania programu
        \item Długowieczność języka oraz narzędzi kompilacyjnych, co wpływa na stabilność i przewidywalność rozwoju oprogramowania
        \item Przenośność między różnymi platformami i architekturami sprzętowymi.
    \end{itemize}
    Wszystkie te kryteria mają istotny wpływ na całkowity koszt i nakład pracy związany z tworzeniem oraz utrzymaniem oprogramowania, a także na jego jakość i przydatność w długim okresie użytkowania.

    Doświadczenie wskazuje również, że te same kryteria można stosować do oceny innych komponentów wspomagających proces tworzenia oprogramowania, takich jak biblioteki klas obiektowych czy projekty typów abstrakcyjnych. Wykorzystanie odpowiednich, zewnętrznych komponentów oraz dobrze zaprojektowanych rozwiązań przyczynia się do poprawy czytelności, utrzymywalności i ogólnej jakości kodu, jednocześnie przyspieszając proces jego tworzenia.

\subsubsection{PPL3: Jak wygląda porównanie dostępności i dojrzałości bibliotek do programowania współbieżnego i równoległego w obu językach?}
Porównując dostępność i dojrzałość bibliotek do programowania współbieżnego i równoległego w językach Rust i C++, zauważyć można istotne różnice wynikające zarówno z historii rozwoju tych języków, jak i ich podejścia do bezpieczeństwa, abstrakcji oraz zarządzania zasobami.

W przypadku języka C++, biblioteki takie jak OpenMP, Intel TBB czy pthreads cechują się dużą dojrzałością oraz szerokim zastosowaniem w przemyśle, zwłaszcza w kontekście obliczeń naukowych, symulacji fizycznych i systemów o wysokiej wydajności. Są one dobrze udokumentowane, posiadają wsparcie komercyjne (np. TBB) oraz charakteryzują się dużą kompatybilnością z istniejącą infrastrukturą sprzętową i programową. Niemniej jednak, wymagają od programisty głębokiej wiedzy w zakresie zarządzania pamięcią oraz synchronizacji, co przekłada się na wyższy próg wejścia i podatność na błędy (np. wyścigi danych czy zakleszczenia).

Z kolei Rust, jako język nowszy, oferuje bardziej nowoczesny zestaw narzędzi do programowania równoległego, w tym biblioteki takie jak Tokio, Rayon czy Crossbeam. Mimo mniejszej liczby lat rozwoju, biblioteki te szybko dojrzewają i zdobywają popularność dzięki silnym gwarancjom bezpieczeństwa pamięci na poziomie kompilatora oraz ergonomicznemu API. Rust promuje bezpieczne podejście do współbieżności poprzez system własności \eng{ownership} i brak domyślnego współdzielenia zasobów, co w praktyce eliminuje całą klasę błędów typowych dla C++. Dzięki temu biblioteki w Rust, choć mniej rozpowszechnione w starszych zastosowaniach, zyskują przewagę w projektach tworzonych od podstaw, zwłaszcza w środowiskach wymagających wysokiej niezawodności.

\subsubsection{PPL4: Czy istnieją systematyczne metodologie porównywania języków programowania w kontekście problemu pracy, które można zastosować do analizy Rust i C++?}
W literaturze istnieje ograniczona liczba systematycznych metodologii porównujących języki programowania pod kątem programowania współbieżnego i równoległego. Większość badań koncentruje się na analizie wydajności i bezpieczeństwa, nie przyjmując jednolitego podejścia metodologicznego.\\
W ramach porównania mechanizmów równoległości w obu językach, opisanych na stronie \cite{parallelrustcppIntroductionComparing} zastosowano konteneryzację obrazu programu a nastęonie uruchmoenie go na maszynie testowej. Jednak jak podaje arytkuł na stronie Medium \cite{rainbow} może to prowadzić do nie uwzględnienia wpływu takich czynników jak czas uruchamiania, rozmiar obrazu i zużycie pamięci. Podaje również metodę \textit{Rainbow}, która testuje przepustowość, uruchamiając kontenery na podobnej infrastrukturze i używając licznika Redis do śledzenia obliczeń.

\subsubsection{PPL5: Jakie aspekty programowania współbieżnego i równoległego w Rust i C++ nie zostały dostatecznie zbadane w literaturze?}
badania skupiały się przede wszystkim na ogólnych aspektach porównawczych bezpieczeństwa pamięci, zarządzania zasobami oraz wydajności kompilacji i wykonania w językach Rust i C++. Niemniej jednak, istnieją wyraźne luki badawcze, które wymagają pogłębionej analizy, w tym
\begin{itemize}
    \item Systematyczne porównanie specyficznych mechanizmów synchronizacji i zarządzania wątkami, takich jak mutexy, semafory, bariery czy kanały komunikacyjne. Brakuje kompleksowych studiów empirycznych, które umożliwiłyby ocenę efektywności oraz narzutu związanych z implementacją tych mechanizmów w obu językach.
    \item Niedostatecznie zbadany wpływ przyjętych modeli programowania współbieżnego i równoległego na skalowalność i stabilność systemów wielowątkowych, szczególnie w kontekście aplikacji o wysokiej konkurencyjności oraz dynamicznie zmieniających się obciążeń.
    \item Ograniczona liczba badań eksperymentalnych, które wykorzystują benchmarki do porównania realizacji konkretnych konstrukcji językowych (np. asynchroniczności, operacji atomowych) w Rust i C++. Taka analiza mogłaby uwzględniać zarówno aspekty wydajnościowe, jak i bezpieczeństwa wykonywania.
\end{itemize}


\subsubsection{PPL6: Jaki jest stan wiedzy na temat wykorzystania programowania współbieżnego w ramach GPU w językach Rust i C++?}
W literaturze przedmiotowej oraz w praktyce programistycznej, zagadnienie wykorzystania programowania współbieżnego na GPU ewoluuje w różnym tempie w zależności od języka programowania. W przypadku języka Rust obserwujemy dynamiczny rozwój narzędzi, które umożliwiają wykorzystanie mocy obliczeniowej kart graficznych przy jednoczesnym zachowaniu wysokich gwarancji bezpieczeństwa pamięci i wątków.

Projekty takie jak rust‑gpu \cite{rustgpuRust} dążą do umożliwienia pisania shaderów w języku Rust, oferując podejście, które integruje możliwości GPU z bezpiecznymi konstrukcjami języka. Dokumentacja dostępna na stronie github \cite{rustgpuRust} wskazuje na intensywne prace nad przeniesieniem wielu korzyści wynikających z systemu własności i typowania Rust na środowisko GPU, co może przyczynić się do zmniejszenia ryzyka błędów współbieżności, które są szczególnie krytyczne w obliczeniach równoległych.

Równolegle, biblioteka Vulkano (dostępna m.in. poprzez dokumentację \cite{docsVulkanoRust}) stanowi wysokopoziomowy, bezpieczny interfejs do API Vulkan, które jest standardem w programowaniu GPU. Vulkano umożliwia abstrakcję złożoności niskopoziomowych interfejsów, jednocześnie oferując możliwość pełnego wykorzystania równoległości GPU. W podobnym nurcie znajduje się projekt wgpu, który implementuje standard WebGPU, umożliwiając przenośne aplikacje graficzne i obliczeniowe, a jednocześnie integrując współczesne podejścia do zarządzania zasobami i synchronizacji.

W przeciwieństwie do podejścia Rust, w środowisku C++ dominującym rozwiązaniem jest platforma CUDA, rozwijana i wspierana przez firmę NVIDIA \cite{nvidiaCUDAToolkit}. CUDA oferuje bardzo dojrzały, zoptymalizowany i szeroko stosowany framework, który pozwala na bezpośrednią implementację algorytmów równoległych na GPU. W odróżnieniu od narzędzi rozwijanych dla Rust, CUDA posiada ugruntowaną pozycję w środowisku przemysłowym, co przekłada się na bogatą dokumentację, szerokie wsparcie techniczne oraz liczne biblioteki wspomagające rozwój aplikacji wykorzystujących GPU.

Podsumowując, stan wiedzy dotyczący programowania współbieżnego na GPU w języku Rust znajduje się na etapie intensywnego rozwoju i eksperymentacji, gdzie projekty takie jak rust‑gpu, Vulkano oraz wgpu \cite{wgpuWgpuPortable} pokazują potencjał tego podejścia, zwłaszcza w kontekście bezpieczeństwa i nowoczesnych abstrakcji. Z kolei w C++ platforma CUDA, dzięki swojej dojrzałości oraz szerokiemu wsparciu ze strony przemysłu, pozostaje głównym narzędziem wykorzystywanym do implementacji wysokowydajnych obliczeń równoległych na GPU.

\subsection{Kierunki dalszych badań}
Na podstawie przeglądu literatury oraz odpowiedzi na pytania do przeglądu literatury można wskazać na kilka kierunków dalszych badań w dziedzinie programowania współbieżnego i równoległego w językach Rust i C++:
\begin{itemize}
    \item Systematyczne porównanie mechanizmów współbieżnych i równoległych w językach Rust i C++
    \item Analiza wydajności i bezpieczeństwa konkretnych mechanizmów współbieżnych (np. std::thread vs std::thread w Rust, OpenMP vs Rayon, async w Rust vs futures w C++20)
    \item Badania eksperymentalne z wykorzystaniem benchmarków oraz analizy kodu źródłowego w celu porównania języków Rust i C++ w kontekście współbieżności i równoległości
    \item Analiza wykorzystania programowania współbieżnego w ramach GPU w językach Rust i C++
\end{itemize}

%Porównanie Rust i C++
\chapter{Wybrane mechanizmy w języku C++}

\section{Programowanie współbieżne}
Współbieżność w języku C++ wspierana jest od standardu C++11, który wprowadził szereg struktur i mechanizmów umożliwiających tworzenie i synchronizację wątków. W kolejnych wersjach (C++14, C++17, C++20 i C++23) język został wzbogacony o kolejne narzędzia, zwiększające bezpieczeństwo, ekspresyjność i ergonomię programowania współbieżnego.

\subsection{Biblioteki i przestrzeń standardowa}
\subsubsection{\texttt{std::thread} oraz \texttt{std::jthread}}
Standardowa biblioteka języka C++ zawiera podstawowe komponenty do obsługi wątków w~module \texttt{<thread>}. Wraz z C++20 wprowadzono \texttt{std::jthread}, będący bezpieczniejszą alternatywą dla \texttt{std::thread}, ponieważ automatycznie dołącza wątek w destruktorze obiektu. Dzięki temu możliwe jest uniknięcie błędów takich jak niezakończony wątek  \eng{orphaned thread} lub przedwczesne zakończenie programu.

\begin{lstlisting}[language=C++, style=VS2017,  caption={Przykład użycia std::jthread}, label={jthread_example}]
#include <iostream>
#include <thread>

void worker() {
    std::cout << "Thread is running." << std::endl;
}

int main() {
    std::jthread t(worker); // wątek zarządzany automatycznie
    // brak konieczności wywoływania join() lub detach()
}
\end{lstlisting}    
W przykładzie listing \ref{jthread_example} utworzono nowy wątek wątek z użyciem \texttt{std::jthread}, który uruchamia funkcję worker. Dzięki automatycznemu zarządzaniu zasobami przez \texttt{std::jthread}, nie ma potrzeby ręcznego wywoływania \texttt{join()}, co zmniejsza ryzyko błędów synchronizacji.


\subsection{Komunikacja między wątkami}
C++ nie posiada wbudowanych kanałów \eng{channels} występujących w języku Rust, lecz umożliwia komunikację poprzez konstrukcje takie jak: kolejki, zmienne warunkowe \texttt{(std::condition\_variable)} oraz typy atomowe \texttt{(std::atomic)}. Jednym z najczęstszych wzorców komunikacyjnych jest użycie kolejki chronionej mutexem oraz sygnalizowanej zmienną warunkową.
\begin{lstlisting}[language=C++, style=VS2017,  caption={Przykład komunikacji między wątkami}, label={condition_variable_example}]
#include <iostream>
#include <thread>
#include <queue>
#include <mutex>
#include <condition_variable>

std::queue<int> buffer;
std::mutex mtx;
std::condition_variable cv;

void producer() {
    std::unique_lock<std::mutex> lock(mtx);
    buffer.push(100); // produkcja danych
    cv.notify_one();  // powiadom konsumenta
}

void consumer() {
    std::unique_lock<std::mutex> lock(mtx);
    cv.wait(lock, [] { return !buffer.empty(); }); // czekaj na dane
    std::cout << "Consumed: " << buffer.front() << std::endl;
    buffer.pop(); // usuń dane z kolejki
}
\end{lstlisting}
Powyższy przykład listing \ref{condition_variable_example} implementuje prosty scenariusz producent-konsument. Producent wstawia dane do kolejki i powiadamia wątek oczekujący. Konsument blokuje się, dopóki kolejka nie będzie zawierać danych. Zmienna warunkowa eliminuje konieczność aktywnego sprawdzania warunku (busy waiting), co poprawia efektywność systemu.


\subsection{Synchronizacja}
Synchronizacja w języku C++ opiera się głównie na mutexach \texttt{(std::mutex)} oraz ich odmianach. Od C++17 dostępny jest \texttt{std::scoped\_lock}, pozwalający na bezpieczne blokowanie wielu mutexów jednocześnie, a od C++20 wprowadzono bardziej zaawansowane konstrukty takie jak \texttt{std::latch} i \texttt{std::barrier}, które umożliwiają synchronizację wielu wątków na określonym etapie wykonania.

\begin{lstlisting}[language=C++, style=VS2017,  caption={Przykład użycia std::scoped\_lock}, label={scoped_lock_example}]
#include <iostream>
#include <thread>
#include <mutex>

int counter = 0;
std::mutex mtx;

void increment() {
    std::lock_guard<std::mutex> lock(mtx); // automatyczna blokada mutexu
    ++counter;
}

int main() {
    std::thread t1(increment);
    std::thread t2(increment);
    t1.join();
    t2.join();
    std::cout << "Counter: " << counter << std::endl;
}
\end{lstlisting}
W tym przykładzie - listing \ref{scoped_lock_example} dwa wątki próbują jednocześnie zwiększyć wartość zmiennej counter. Aby zapobiec wyścigowi danych, dostęp do zasobu jest chroniony przez \texttt{std::mutex}. Użycie \texttt{std::lock\_guard} zapewnia, że blokada zostanie zwolniona automatycznie po wyjściu z zakresu funkcji.

\subsubsection{\texttt{std::latch} oraz \texttt{std::barrier} (C++20)}
Mechanizmy \texttt{std::latch} oraz \texttt{std::barrier} wprowadzone w standardzie C++20 służą do synchronizacji wielu wątków w określonym punkcie programu:
\begin{itemize}
    \item \texttt{std::latch} - jednorazowy licznik synchronizacyjny, który pozwala wątkom oczekującym na rozpoczęcie działania, aż inne wątki zakończą przygotowanie.
    \item \texttt{std::barrier} - wielokrotnego użytku, synchronizuje grupę wątków po osiągnięciu „cyklu bariery”.
\end{itemize}
\begin{lstlisting}[language=C++, style=VS2017,  caption={Przykład użycia std::latch oraz std::barrier}, label={latch_barrier_example}]
#include <iostream>
#include <thread>
#include <latch>
#include <barrier>

constexpr int num_threads = 3;
std::latch start_latch(num_threads);
std::barrier sync_barrier(num_threads);

void worker(int id) {
    std::cout << "Thread " << id << " is initializing.\n";
    start_latch.arrive_and_wait(); // czekaj aż wszystkie wątki się przygotują
    for (int i = 0; i < 2; ++i) {
        std::cout << "Thread " << id << " is processing iteration " << i << ".\n";

        // Synchronizacja między iteracjami
        sync_barrier.arrive_and_wait();

        std::cout << "Thread " << id << " passed the barrier in iteration " << i << ".\n";
    }
}
int main() {
    std::thread threads[num_threads];
    for (int i = 0; i < num_threads; ++i)
    threads[i] = std::thread(worker, i + 1);
    for (auto& t : threads)
        t.join();
}
\end{lstlisting}    
W listingu \ref{latch_barrier_example} każdy wątek najpierw dochodzi do punktu synchronizacji \texttt{std::latch}, oczekując aż wszystkie inne wątki również zakończą fazę inicjalizacji. Następnie, w dwóch kolejnych iteracjach przetwarzania danych, zastosowany zostaje \texttt{std::barrier}, który gwarantuje, że wszystkie wątki ukończą daną iterację przed przejściem do kolejnej. Takie podejście zwiększa spójność przetwarzania i eliminuje potencjalne niespójności wynikające z wyścigów czasowych między wątkami.

\subsection{Asynchroniczność}
Programowanie asynchroniczne w C++ możliwe jest dzięki konstrukcjom takim jak \texttt{std::async}, \texttt{std::future} i \texttt{std::promise}. \texttt{std::async} uruchamia funkcję w tle i umożliwia jej obserwację za pomocą obiektu \texttt{future}. Podejście to ułatwia uruchamianie zadań bez konieczności jawnego zarządzania wątkiem.

\begin{lstlisting}[language=C++, style=VS2017,  caption={Przykład użycia std::async}, label={async_example}]
    #include <iostream>
    #include <future>
    
    int compute() {
        return 2 * 21;
    }
    
    int main() {
        std::future<int> result = std::async(std::launch::async, compute);
        std::cout << "Result: " << result.get() << std::endl;
    }
\end{lstlisting}
W powyższym listingu \ref{async_example} funkcja \texttt{compute()} zostaje uruchomiona asynchronicznie. \texttt{std::future} pozwala na uzyskanie wyniku, gdy ten będzie dostępny. W ten sposób możemy kontynuować inne działania, a wynik odebrać w późniejszym czasie — co jest przydatne w~aplikacjach wymagających wysokiej responsywności.

\subsubsection{std::promise}
Obiekt \texttt{std::promise} (obietnica) w języku C++ umożliwia przekazywanie wartości z jednego wątku do drugiego. Stanowi uzupełnienie mechanizmu \texttt{std::future}, ponieważ pozwala manualnie ustawić wartość, którą futurę później odbierze. Dzięki temu rozdzielona zostaje produkcja i konsumpcja danych między wątkami, umożliwiając bardziej elastyczne projektowanie asynchronicznych przepływów sterowania.
\begin{lstlisting}[language=C++, style=VS2017,  caption={Przykład użycia std::promise}, label={promise_example}]
#include <iostream>
#include <thread>
#include <future>

// Funkcja symulująca kosztowne obliczenie
int compute(int x) {
    return x * 2;
}

int main() {
    std::promise<int> promise; // utworzenie obiektu obietnicy
    std::future<int> result = promise.get_future(); // pobranie powiązanego future

    std::thread producer([&promise]() {
        int value = 21;
        int result = compute(value);
        promise.set_value(result); // ustawienie wartości, która zostanie odebrana przez future
    });

    std::cout << "Result: " << result.get() << std::endl; // odbiór wartości, blokuje do czasu jej ustawienia
    producer.join();

    return 0;
}
\end{lstlisting}    
W listingu \ref{promise_example} wątek główny tworzy promesę i pobiera powiązaną z nią futurę. Wątek producenta wykonuje obliczenie i przekazuje wynik przez promesę. Funkcja \texttt{result.get()} wstrzymuje główny wątek do czasu dostępności wyniku.
\subsubsection{std::packaged\_task}
\texttt{std::packaged\_task} to obiekt otaczający dowolną wywoływalną funkcję (np. funkcję, lambda, \texttt{std::bind}), który integruje się z future. Pozwala to uruchomić zadanie w wątku i obserwować jego wynik.
\begin{lstlisting}[language=C++, style=VS2017,  caption={Przykład użycia std::packaged\_task}, label={packaged_task_example}]
#include <iostream>
#include <thread>
#include <future>

// Funkcja do opakowania w packaged_task
int compute(int x) {
    return x * 2;
}

int main() {
    std::packaged_task<int(int)> task(compute); // utworzenie zapakowanego zadania
    std::future<int> result = task.get_future(); // pobranie powiązanego future

    std::thread worker(std::move(task), 21); // uruchomienie zadania w wątku z parametrem 21

    std::cout << "Result: " << result.get() << std::endl; // odbiór wyniku
    worker.join();

    return 0;
}
\end{lstlisting}
W powyższym przykładzie listing \ref{packaged_task_example} funkcja \texttt{compute} została opakowana w \texttt{packaged\_task}, a następnie uruchomiona w osobnym wątku z argumentem 21. Wynik trafia do futury, który umożliwia jego odbiór w wątku głównym. przetwarzanie zadań.

\subsubsection{C++23 - \texttt{std::task} i \texttt{std::execution}}
Standard C++23 (najnowszy w chwili tworzenia niniejszej pracy) wprowadza nowe pojęcia:
\begin{itemize}
    \item \texttt{std::task} - reprezentuje zadanie, które można uruchomić z użyciem określonego planisty wykonania.
    \item \texttt{std::execution} - zestaw polityk (strategii) określających sposób wykonywania zadań, takich jak sekwencyjnie, współbieżnie, równolegle.
\end{itemize}

Mechanizmy te są częścią trwającej transformacji C++ w kierunku deklaratywnego modelu programowania współbieżnego i równoległego. Mają one zostać wprowadzone wstępnie w wersji C++26 \cite{cpp26} Ponieważ wsparcie dla tych mechanizmów jest na etapie wdrażania, nie będą one szczegółowo omawiane w tej pracy. Warto jednak zauważyć, że ich celem jest uproszczenie i~ujednolicenie podejścia do programowania współbieżnego, podobnie jak ma to miejsce w~języku Rust.

\section{Programowanie równoległe}
\subsection{OpenMP -  Open Multi-Processing}
OpenMP to biblioteka umożliwiająca programowanie równoległe w modelu pamięci współdzielonej. Jest dostępna dla języków C, C++ oraz Fortran i opiera się na użyciu dyrektyw preprocesora (ang. compiler directives) pozwalających na uproszczone rozproszenie zadań pomiędzy wątki w sposób deklaratywny.

Podstawowym celem OpenMP jest ułatwienie implementacji aplikacji równoległych poprzez maksymalne ograniczenie ręcznego zarządzania wątkami i synchronizacją. W kodzie C++ użycie tej technologii wymaga aktywacji flagi -fopenmp podczas kompilacji przy użyciu kompilatora g++.
Przedstawia podstawowe pojęcia:
\begin{itemize}
    \item \#pragma omp parallel — dyrektywa inicjująca blok kodu, który ma zostać wykonany równolegle przez wiele wątków,
    \item shared — oznacza zmienne współdzielone pomiędzy wszystkimi wątkami,
    \item private — każdemu wątkowi przypisywana jest prywatna kopia zmiennej,
    \item cache locality — pamięć podręczna (ang. cache) może znacznie poprawić wydajność przetwarzania, choć kosztem większego zużycia pamięci.
\end{itemize}

\begin{lstlisting}[language=C++, caption={Przykład użycia OpenMP w C++}, label={lst:openmp_example}]
#include <omp.h>
#include <iostream>

int main() {
    int sum = 0;
    #pragma omp parallel for reduction(+:sum)
    for (int i = 1; i <= 100; ++i) {
        sum += i;
    }
    std::cout << "Suma: " << sum << std::endl;
    return 0;
}
\end{lstlisting}    
W powyższym przykładzie w listingu \ref{lst:openmp_example} zmienna sum jest sumą liczb od 1 do 100 obliczaną równolegle. Dzięki użyciu dyrektywy \#pragma omp parallel for każda iteracja pętli może być wykonana w osobnym wątku. Atrybut reduction(+:sum) zapewnia bezpieczne sumowanie wyników lokalnych wątków do jednej wartości globalnej. OpenMP automatycznie zarządza synchronizacją i agregacją wyników, dzięki czemu użytkownik nie musi implementować ręcznego zarządzania zasobami współdzielonymi
.

\subsection{Intel TBB (Threading Building Blocks)}
Intel Threading Building Blocks (TBB) to nowoczesna biblioteka programistyczna dla języka C++, przeznaczona do tworzenia aplikacji równoległych w sposób wysoce elastyczny i skalowalny. W przeciwieństwie do OpenMP, TBB opiera się na programowaniu funkcyjnym i komponentowym, umożliwiając dekompozycję zadań \eng{ task-based parallelism}, a nie operacji niskiego poziomu.

Cechy charakterystyczne:
\begin{itemize}
    \item Deklaratywny styl programowania, który umożliwia oddelegowanie decyzji o wykonaniu do systemu planowania zadań.
    \item Dynamiczna alokacja wątków w oparciu o dostępność zasobów.
    \item Wbudowana obsługa synchronizacji oraz struktur danych przystosowanych do środowisk wielowątkowych (np. concurrent\_vector, concurrent\_queue).
\end{itemize}

\begin{lstlisting}[language=C++, caption={Przykład użycia Intel TBB w C++}, label={lst:tbb_example}]
#include <tbb/tbb.h>
#include <iostream>
#include <vector>

int main() {
    std::vector<int> data(1000, 1);
    int sum = 0;

    tbb::parallel_reduce(
        tbb::blocked_range<size_t>(0, data.size()),
        0,
        [&](const tbb::blocked_range<size_t>& r, int local_sum) {
            for (size_t i = r.begin(); i < r.end(); ++i)
                local_sum += data[i];
            return local_sum;
        },
        std::plus<int>()
    );

    std::cout << "Suma: " << sum << std::endl;
    return 0;
}
\end{lstlisting}
W powyższym przykładzie w listingu \ref{lst:tbb_example} funkcja tbb::parallel\_reduce automatycznie dzieli zakres danych na bloki (blocked\_range), które przetwarzane są równolegle przez dostępne wątki. Funkcja lambda odpowiada za lokalne przetwarzanie danych (w tym przypadku sumowanie wartości), a następnie lokalne wyniki są agregowane przy użyciu funkcji std::plus<int>. TBB samodzielnie zarządza planowaniem zadań oraz synchronizacją, co czyni go potężnym narzędziem w budowie skalowalnych aplikacji równoległych.


%Porównanie Rust i C++
% \chapter{Porównanie mechanizmów w językach Rust oraz C++}
% \section{Programowanie współbieżne}
% Thread Creation and Management
% Thread creation time (microseconds)
% Memory overhead per thread (KB)
% Maximum number of threads before system failure
% Synchronization Performance
% Mutex lock/unlock operations per second
% Channel/message passing latency (microseconds)
% Context switching time between threads (microseconds)
% Safety Overhead
% Compilation time with concurrent code (seconds)
% Binary size comparison (KB)
% Memory usage during runtime (MB)

% \section{Programowanie równoległe}

% Computation Performance
% Execution time for parallel algorithms (milliseconds)
% Speedup ratio (T1/Tn) with different thread counts
% Efficiency (Speedup/Number of processors)
% GFLOP/s performance
% - Single-threaded GFLOP/s
% - Multi-threaded GFLOP/s
% - Scaling efficiency (GFLOP/s per core)
% Amdahl's law comparison
% Resource Usage
% CPU utilization percentage
% Memory consumption under load
% Cache miss rates
% I/O operations per second

% GFLOP/s is particularly useful:\\

% Provides a standardized way to measure computational performance
% Allows direct comparison of algorithmic efficiency
% Helps evaluate hardware utilization effectiveness
% To measure GFLOP/s:\\

% Matrix multiplication
% FFT (Fast Fourier Transform)
% N-body simulation

% \chapter{Porównanie mechanizmów w językach Rust oraz C++}
\chapter{Założenia i metodologia porównania mechanizmów w językach Rust oraz C++}
W tym rozdziale przedstawiono metryki oraz algorytmy do analizy wydajności mechanizmów programowania współbieżnego i równoległego w językach Rust oraz C++. 
\section{Programowanie współbieżne}
Dla mechanizmów programowania współbieżnego autor nie znalazł obecnie zunifikowanego, powszechnie uznanego zestawu benchmarków odpowiadający randze NPB. W związku z tym, w ramach niniejszej pracy, opracowano własny zestaw mini-aplikacji testowych, zaprojektowanych w taki sposób, aby odzwierciedlały typowe scenariusze współbieżności: komunikację między wątkami, synchronizację, obsługę asynchronicznych operacji wejścia/wyjścia, sytuacje ryzyka zakleszczenia, a także przypadki intensywnego przetwarzania danych z wykorzystaniem wielu wątków.

\subsection{Zarządzanie wątkami}
Porównanie tworzenia i zarządzania wątkami obejmuje następujące parametry:
\begin{itemize}
\item czas tworzenia wątku (w mikrosekundach),
\item narzut pamięciowy na wątek (w KB)
\end{itemize}

\subsection{Wydajność synchronizacji}
Kluczowe aspekty synchronizacji to:
\begin{itemize}
\item liczba operacji blokowania i odblokowywania mutexa na sekundę,
\item opóźnienie przesyłania wiadomości przez kanały (w mikrosekundach),
\item czas przełączania kontekstu pomiędzy wątkami (w mikrosekundach).
\end{itemize}

\subsection{Narzut bezpieczeństwa}
W kontekście narzutu związanego z mechanizmami bezpieczeństwa w językach Rust i C++ analizowane będą następujące metryki:
\begin{itemize}
\item czas kompilacji kodu współbieżnego (w sekundach),
\item rozmiar pliku binarnego (w KB),
\item zużycie pamięci podczas wykonywania programu (w MB).
\end{itemize}

\subsection{Wybrane algorytmy do analizy}
Dla porównania mechanizmów współbieżności wybrano następujące algorytmy:
\begin{itemize}
    \item Model producent-konsument (z wykorzystaniem aktorów),
    \item Problem filozofów (synchronizacja dostępu do zasobów).
\end{itemize}
Algorytmy te pozwalają na analizę zdolności języków Rust i C++ do efektywnego zarządzania współbieżnością w obliczeniach numerycznych.


\section{Programowanie równoległe}
W przypadku programowania równoległego, zdecydowano się na wykorzystanie uznanego zestawu testowego NAS Parallel Benchmarks (NPB) \cite{nasaParallelBenchmarks}. Zestaw ten jest szeroko stosowany w środowisku naukowym do oceny wydajności systemów wysokowydajnych (HPC) i stanowi wiarygodny punkt odniesienia przy analizie efektywności obliczeniowej.
\subsection{Wydajność obliczeniowa}
Główne metryki oceny wydajności algorytmów równoległych to:
\begin{itemize}
\item czas wykonania algorytmów równoległych (w milisekundach),
\item współczynnik przyspieszenia (T1/Tn) dla różnych liczby wątków,
\item efektywność (przyspieszenie/liczba procesorów).
\end{itemize}

\subsection{Wydajność sprzętowa (GFLOP/s)}
Wydajność obliczeniowa mierzona w jednostkach GFLOP/s (gigaflops per second) pozwala na ocenę efektywności wykorzystania sprzętu:
\begin{itemize}
\item wydajność w pojedynczym wątku,
\item wydajność wielowątkowa,
\item efektywność skalowania (GFLOP/s na rdzeń).
\end{itemize}
Dodatkowo przeprowadzona zostanie analiza zgodnie z prawem Amdahla w celu określenia teoretycznych ograniczeń przyspieszenia obliczeń.

\subsection{Zasoby systemowe}
Analiza zużycia zasobów przez algorytmy równoległe obejmuje:
\begin{itemize}
\item procentowe wykorzystanie CPU,
\item zużycie pamięci w warunkach obciążenia,
\item współczynnik nietrafień w cache,
\item liczbę operacji wejścia-wyjścia na sekundę.
\end{itemize}

\subsection{Wybrane algorytmy do analizy}
Dla porównania mechanizmów równoległości wybrano następujące algorytmy z zestawu testowego NPB:
\begin{itemize}
    \item CG - \eng{conjugate gradient} - gradient sprzężony
    \item EP - \eng{embarrassingly parallel} - problem trywialnie równoległy
    \item IS - \eng{intiger sorting} - sortowanie liczb całkowitych
\end{itemize}

Wybór powyższych benchmarków pozwoli na szczegółową analizę wydajności oraz stabilności obu języków w kontekście programowania współbieżnego i równoległego, umożliwiając sformułowanie rekomendacji dotyczących wyboru narzędzi w zależności od specyfiki projektu.

\subsubsection{CG - Gradient sprzężony}
Benchmark CG \eng{conjugate gradient} służy do oceny wydajności systemów wysokowydajnych (HPC) w kontekście rozwiązywania rzadkich układów równań liniowych metodą iteracyjną. Algorytm ten znajduje zastosowanie w wielu dziedzinach nauk obliczeniowych, takich jak mechanika płynów czy analiza strukturalna, gdzie układy równań wynikają z dyskretyzacji równań różniczkowych cząstkowych. W benchmarku CG generowana jest syntetyczna macierz rzadkich współczynników o dużych rozmiarach, a następnie przeprowadzana jest iteracyjna procedura wyznaczania przybliżonego rozwiązania układu równań. Test ten charakteryzuje się intensywnym wykorzystaniem operacji wektorowych i punktowych operacji na danych rozproszonych, co czyni go szczególnie użytecznym przy ocenie efektywności komunikacji między wątkami oraz przepustowości pamięci w systemach równoległych \cite{nasaParallelBenchmarks}.

\subsubsection{EP - Problem trywialnie równoległy}
Benchmark EP \eng{embarrassingly parallel} został zaprojektowany w celu oceny wydajności systemów obliczeniowych w scenariuszach, w których niemal całkowicie eliminuje się konieczność komunikacji między procesami lub wątkami. Test ten polega na generowaniu dużej liczby losowych punktów i przeprowadzaniu na nich niezależnych obliczeń statystycznych, takich jak estymacja wartości $\pi$ lub momentów rozkładu. Dzięki swojej naturze, EP umożliwia niemal idealną skalowalność równoległą i jest wykorzystywany przede wszystkim do pomiaru czystej mocy obliczeniowej procesorów, efektywności rozdziału zadań oraz narzutu wynikającego z zarządzania wątkami. Ze względu na minimalne wymagania względem synchronizacji i komunikacji, benchmark ten stanowi punkt odniesienia przy analizie teoretycznego maksimum wydajności danego systemu dla obciążeń równoległych \cite{nasaParallelBenchmarks}.

\subsubsection{IS - Sortowanie liczb całkowitych}
Benchmark IS \eng{integer sorting} służy do oceny wydajności systemów obliczeniowych w zakresie operacji nieciągłych i trudnych do zrównoleglenia, takich jak sortowanie i przemieszczanie danych w pamięci. Test polega na wygenerowaniu losowego zestawu liczb całkowitych, a następnie ich posortowaniu przy użyciu metody sortowania kubełkowego \eng{bucket sort} z zastosowaniem rozproszonej synchronizacji i komunikacji między wątkami lub procesami. Benchmark IS jest szczególnie przydatny do analizy przepustowości podsystemów pamięciowych, efektywności komunikacji w architekturach wieloprocesorowych oraz odporności systemu na nierównomierne rozłożenie danych. Ze względu na swoją nieregularną strukturę dostępu do danych i znaczną liczbę operacji porządkowania, IS stanowi istotne uzupełnienie pozostałych testów NPB, koncentrując się na problemach wymagających intensywnej pracy z pamięcią i synchronizacją \cite{nasaParallelBenchmarks}.

\section{Metodologia badań}
Badania eksperymentalne zostały zaprojektowane w taki sposób, aby umożliwić porównanie mechanizmów współbieżnych i równoległych w językach Rust oraz C++ przy wykorzystaniu dwóch odmiennych architektur sprzętowych: x86\_64 (architektura tradycyjna, Windows/Linux) oraz ARM64 (architektura Apple Silicon – M1, macOS). Dzięki temu możliwa będzie analiza wpływu typu procesora i systemu operacyjnego na wydajność oraz efektywność implementacji.

\subsection{Środowisko testowe}
W ramach środowiska testowego zostały wykorzystane następujące urządzenia wraz z oprogoramowaniem:
\subsubsection{Architektura ARM}
W ramach architektury ARM został wykorzystany laptop firmy Apple - MacBook Pro z następującymi specyfikacjami:
\begin{itemize}
    \item procesor Apple M1
    \item pamięć RAM 16 GB
    \item system - macOS Sonoma wersja 14.5 (23F79)
\end{itemize}
\subsubsection{Architektura x86\_64}
W ramach architektury x86\_64 został wykorzystany laptop firmy HP z następującymi specyfikacjami:
\begin{itemize}
    \item procesor
    \item pamięć RAM 32 GB
    \item system - Linux Ubuntu 24.04 LST
\end{itemize}
\subsection{Procedura testowa}
Procedura testowa będzie obejmowała uruchomienie zestawu benchmarków w różnych konfiguracjach, takich jak liczba wątków oraz różne klasy algorytmów w przypadku NPB oraz na dwóch architekturach procesora. Każdy test będzie uruchamiany wielokrotnie - 10 razy, aby uzyskać uśrednione wyniki. Algorytmy zawierają również swoje logi zdzarzeń, które zostaną użyte do weryfikacji poprawności działania samego algorytmu jak i do późniejszej analizy wyników.
\subsection{Narzędzia pomiarowe}
W ramach przeprowadzania testów zostaną wykorzystane następujące narzędzia pomiarowe:
\begin{itemize}
    \item perf - narzędzie do profilowania kodu, które pozwala na analizę wydajności aplikacji w czasie rzeczywistym (jednakże nie jest dostępne na systemach Apple)
    \item instruments (Xcode) - oficjalne narzędzie do profilowania od firmy Apple - zamiennik narzędzia \textit{perf} w tym przypadku,
    \item hwloc - narzędzie pozwalające zbadać zachowanie programu jeżeli chodzi o dostęp do podzespołów komputera
    \item threadsanitizer - flaga do kompilatora (dla języka C++ w tym przypadku), która pozwala sprawdzić czy w fazie kompilwoania nie zachodzi sytuacja wyścigów
\end{itemize}

\section{Porównanie międzyjęzykowe}
\section{Programowanie równoległe}
W ramach programów równoległych wykorzystano jako wzorzec, gotowe implementacje problemów z zestawu NPB w ramach istniejącej pracy The NAS Parallel Benchmarks for evaluating C++ parallel programming frameworks on shared-memory architectures \cite{CPPNPB} oraz programów bazujących na nich w języku Rust, napisanych w ramach proejktu na studia przez G.Bessa et al. \cite{NPBRust}.
\section{Programowanie równoległe}
W ramach programów równoległych wykorzystano jako wzorzec, gotowe implementacje problemów z zestawu NPB w ramach istniejącej pracy The NAS Parallel Benchmarks for evaluating C++ parallel programming frameworks on shared-memory architectures \cite{CPPNPB} oraz programów bazujących na nich w języku Rust, napisanych w ramach proejktu na studia przez G.Bessa et al. \cite{NPBRust}.

\section{Programowanie równoległe}
W ramach programów równoległych wykorzystano jako wzorzec, gotowe implementacje problemów z zestawu NPB w ramach istniejącej pracy The NAS Parallel Benchmarks for evaluating C++ parallel programming frameworks on shared-memory architectures \cite{CPPNPB} oraz programów bazujących na nich w języku Rust, napisanych w ramach proejktu na studia przez G.Bessa et al. \cite{NPBRust}.
% \section{Programowanie równoległe}
W ramach programów równoległych wykorzystano jako wzorzec, gotowe implementacje problemów z zestawu NPB w ramach istniejącej pracy The NAS Parallel Benchmarks for evaluating C++ parallel programming frameworks on shared-memory architectures \cite{CPPNPB} oraz programów bazujących na nich w języku Rust, napisanych w ramach proejktu na studia przez G.Bessa et al. \cite{NPBRust}.
%Porównanie Rust i C++
\chapter{Porównanie międzyjęzykowe - programowanie współbieżne}

W ramach analizy programowania współbieżnego przeanalizowano implementacje wzorców producent-konsument oraz serwer echo w językach Rust i C++. Obie wersje korzystają z różnych podejść do zarządzania współbieżnością: Rust z biblioteką Tokio oraz C++ z mechanizmami standardowej biblioteki i rozszerzeniami języka C++20.

\section{Porównanie międzyjęzykowe}

\subsection{Struktura i organizacja kodu}

\begin{table}[H]
    \centering
    \caption{Porównanie aspektów struktury i organizacji kodu w implementacjach współbieżnych}
    \begin{tabularx}{\textwidth}{lXX}
        \toprule
        \textbf{Aspekt} &
        \textbf{Rust (Tokio)} &
        \textbf{C++ (std::thread + epoll/kqueue)} \\
        \midrule
        Architektura &
        Sterowana zdarzeniami z asynchronicznym środowiskiem wykonawczym, tworzenie zadań &
        Wątek na połączenie + opcjonalna pętla zdarzeń \\
        \hline
        Enkapsulacja &
        Moduły, cechy, system własności &
        Klasy, szablony, przestrzenie nazw \\
        \hline
        Obsługa błędów &
        \texttt{Result<T,E>}, operator \texttt{?}, wymuszona propagacja &
        \texttt{std::optional}, wyjątki, kody \texttt{errno} \\
        \hline
        Asynchroniczność &
        \texttt{async/await}, \texttt{tokio::select!}, wbudowane środowisko wykonawcze &
        \texttt{std::async}, ręczne pętle zdarzeń (epoll/kqueue) \\
        \hline
        Bezpieczeństwo typów &
        Sprawdzanie własności w czasie kompilacji, cechy \texttt{Send/Sync} &
        Sprawdzanie w czasie wykonania, ręczna synchronizacja \\
        \hline
        Konfiguracja &
        Cargo.toml, flagi funkcjonalności \eng{feature flags} &
        CMake/Make, dyrektywy preprocesora \\
        \hline
        Zarządzanie zależnościami &
        Scentralizowane z crates.io &
        Zdecentralizowane, pakiety systemowe \\
        \bottomrule
    \end{tabularx}
\end{table}

Implementacja w języku Rust cechuje się większą spójnością architektoniczną dzięki wbudowanym mechanizmom asynchroniczności oraz systemowi cech \eng{traits}. C++ zapewnia większą elastyczność kosztem złożoności zarządzania zgodnością między różnymi wersjami standardu.


\subsection{Zarządzanie pamięcią}

\begin{table}[H]
    \centering
    \caption{Porównanie modeli zarządzania pamięcią w badanych implementacjach}
    \renewcommand{\arraystretch}{1.3}
    \begin{tabularx}{\textwidth}{lX X}
        \toprule
        \textbf{Aspekt} &
        \textbf{Rust} &
        \textbf{C++} \\
        \midrule
        Model podstawowy &
        Własność z automatycznym zwalnianiem; sprawdzanie pożyczania w czasie kompilacji &
        RAII z ręcznym zarządzaniem czasem życia obiektów \\
        \midrule
        Współdzielenie w wątkach &
        \texttt{Arc<T>} dla danych niemutowalnych, \texttt{Arc<Mutex<T>>} dla mutowalnych &
        \texttt{std::shared\_ptr<T>}, ręczne opakowanie muteksem \\
        \midrule
        Bezpieczeństwo pamięci &
        Gwarantowane na etapie kompilacji, eliminacja wyścigów danych &
        Błędy ujawniane w czasie wykonania; ryzyko użycia po zwolnieniu \\
        \midrule
        Zwalnianie zasobów &
        Automatyczne dzięki cechom \texttt{Drop} &
        Destruktory RAII, często konieczne jawne zwolnienie (np. \texttt{close()}) \\
        \midrule
        Pomiar pamięci &
        Ograniczony dostęp do systemu plików \texttt{/proc} &
        Pełna analiza wykorzystania pamięci (RSS, sterta) \\
        \midrule
        Zarządzanie buforami &
        Bezpieczne wskaźniki i struktury (\texttt{Vec<T>}, \texttt{slice}) ze sprawdzaniem granic &
        Surowe wskaźniki i tablice, ryzyko przepełnień \\
        \midrule
        Porządek pamięci &
        \texttt{Ordering::Relaxed}, \texttt{Acquire}, \texttt{Release} &
        Odpowiedniki w \texttt{std::memory\_order} \\
        \midrule
        Wykrywanie wycieków &
        Rzadkie (np. cykliczne referencje z \texttt{Arc}); wspierane przez system typów &
        Częsty problem, wymagane zewnętrzne narzędzia (np. Valgrind) \\
        \bottomrule
    \end{tabularx}
\end{table}



System własności w języku Rust eliminuje całe klasy błędów pamięciowych na etapie kompilacji. C++ wymaga większej ostrożności programisty oraz wykorzystania narzędzi diagnostycznych.

\subsection{Mechanizmy współbieżności}

\begin{table}[H]
    \centering
    \caption{Porównanie mechanizmów współbieżności w badanych implementacjach}
    \renewcommand{\arraystretch}{1.3}
    \begin{tabularx}{\textwidth}{l>{\raggedright\arraybackslash}X>{\raggedright\arraybackslash}X}
        \toprule
        \textbf{Mechanizm} &
        \textbf{Rust (Tokio)} &
        \textbf{C++ (std + wywołania systemowe)} \\
        \midrule
        Model wykonania &
        Wątki M:N z planistą opartym na kradzieży zadań (work-stealing scheduler) &
        Wątki 1:1 zarządzane przez systemowy planista \\
        \midrule
        Tworzenie zadań &
        \texttt{tokio::spawn(async move \{\})} - lekki i szybki sposób uruchamiania zadań asynchronicznych &
        \texttt{std::thread::spawn()} z opcjonalnym \texttt{.detach()} lub synchronizacją przez \texttt{join()} \\
        \midrule
        Komunikacja &
        Kanały wieloproducent-pojedynczykonsument: \texttt{mpsc::channel} z \texttt{Arc<Mutex<Receiver>>} &
        Własna implementacja \texttt{Channel<T>} oparta na \texttt{std::condition\_variable} \\
        \midrule
        Synchronizacja &
        Primitives: \texttt{tokio::sync::Mutex}, \texttt{Semaphore}, \texttt{RwLock} - dostosowane do async &
        Klasyczne prymitywy: \texttt{std::mutex}, \texttt{condition\_variable}, atomiki z \texttt{std::atomic} \\
        \midrule
        Operacje wejścia/wyjścia &
        Nieblokujące I/O z użyciem await, np. \texttt{TcpListener::bind().await} &
        Blokujące systemowe wywołania: \texttt{accept()}, \texttt{recv()}, \texttt{send()} \\
        \midrule
        Obsługa zdarzeń &
        Asynchroniczne makro \texttt{tokio::select!} do oczekiwania na wiele zdarzeń &
        Ręczne pętle zdarzeń z wykorzystaniem \texttt{epoll\_wait()} lub \texttt{kevent()} \\
        \midrule
        Ograniczenia zasobów &
        Semafory asynchroniczne, np. \texttt{Semaphore::try\_acquire\_owned()} &
        Własne liczniki z użyciem \texttt{std::atomic<size\_t>} \\
        \midrule
        Koordynacja zamknięcia &
        Kanały rozgłoszeniowe, np. \texttt{broadcast::channel} z łagodnym zamykaniem (graceful shutdown) &
        Flagi typu \texttt{std::atomic<bool>} sygnalizujące zakończenie wątku \\
        \midrule
        Zapobieganie wyścigom danych &
        Zapewnione na etapie kompilacji przez ograniczenia traitów \texttt{Send} i \texttt{Sync} &
        Potrzebna synchronizacja ręczna; ryzyko błędów wykonawczych i wyścigów danych \\
        \bottomrule
    \end{tabularx}
\end{table}


Porównując mechanizmy współbieżności w językach Rust i C++, można zauważyć fundamentalne różnice w filozofii projektowej i gwarancjach bezpieczeństwa. Rust przyjmuje podejście "bezpieczeństwo przede wszystkim", podczas gdy C++ oferuje większą elastyczność, ale wymaga większej dyscypliny od programisty.

\subsubsection{Rust - kanały typowane}
Implementacja Rust wykorzystuje kanały z różnymi semantykami komunikacji (\texttt{broadcast}):
\begin{lstlisting}[language=Rust, caption={Producent-Konsument w Rust z mpsc channels}, label={lst:rust_producer_consumer}]
// Rust: producent-konsument z kanałami mpsc
fn producer_consumer_benchmark() {
    let (tx, rx) = mpsc::channel();
    let rx = Arc::new(Mutex::new(rx));
    let shutdown = Arc::new(AtomicBool::new(false));
    
    // Wątki producentów
    thread::spawn(move || {
        for i in 0..ITEMS {
            tx.send(format!("Item-{}", i)).unwrap();
        }
        // Kanał zamyka się automatycznie po usunięciu tx
    });

    // Wątki konsumentów - współdzielony odbiornik z Arc<Mutex<>>
    let rx_clone = Arc::clone(&rx);
    thread::spawn(move || {
        while let Ok(msg) = rx_clone.lock().unwrap().try_recv() {
            process_message(msg);
        }
    });
}
\end{lstlisting}

\subsubsection{C++ - ręczna synchronizacja}
Wersja C++ opiera się na ręcznym zarządzaniu współbieżnością z użyciem mutexów i zmiennych warunkowych:
\begin{lstlisting}[language=C++, style=VS2017,  caption={Producent-Konsument w C++ z Channel}, label={lst:cpp_producer_consumer}]
// C++: Własny kanał z ręczną synchronizacją
template<typename T>
class Channel {
    std::queue<T> queue_;
    mutable std::mutex mutex_;
    std::condition_variable condition_;
    bool closed_ = false;
    
public:
    void send(T item) { 
        std::lock_guard lock(mutex_); 
        queue_.push(std::move(item)); condition_.notify_one(); 
    }
    
    bool recv(T& item) {
        std::unique_lock lock(mutex_);
        condition_.wait(lock, [this]{ return !queue_.empty() || closed_; });
        if (queue_.empty()) return false;
        item = std::move(queue_.front()); queue_.pop();
        return true;
    }
    
    void close() { std::lock_guard lock(mutex_); closed_ = true; condition_.notify_all(); }
};
\end{lstlisting}

\subsubsection{Serwera echo}

\subsubsection{Rust - wejście/wyjście sterowane zdarzeniami}
Implementacja Rust wykorzystuje nieblokujące I/O:
\begin{lstlisting}[language=Rust, caption={Echo Serwer w Rust z Tokio}, label={lst:rust_echo_server}]
// Rust: Asynchroniczny echo serwer z Tokio
async fn run_echo_server(addr: &str, max_connections: usize) {
    let listener = TcpListener::bind(addr).await.unwrap();
    let semaphore = Arc::new(Semaphore::new(max_connections));
    
    loop {
        tokio::select! {
            Ok((stream, addr)) = listener.accept() => {
                if let Ok(permit) = semaphore.clone().try_acquire_owned() {
                    tokio::spawn(async move {
                        handle_client(stream, addr).await;
                        drop(permit); // Auto connection limit
                    });
                }
            }
            _ = shutdown_signal() => break,
        }
    }
}

async fn handle_client(mut stream: TcpStream, addr: SocketAddr) {
    let mut buffer = [0; 1024];
    while let Ok(n) = stream.read(&mut buffer).await {
        if n == 0 { break; }
        stream.write_all(&buffer[..n]).await.unwrap(); // Echo back
    }
}
\end{lstlisting}

\subsubsection{C++ - jeden wątek na połączenie}
Wersja C++ tworzy osobny wątek dla każdego klienta:
\begin{lstlisting}[language=C++, style=VS2017,  caption={Echo Serwer w C++ z jednym wątkiem na połączenie}, label={lst:cpp_echo_server}]
// C++: Traditional Thread-per-Connection Model
class EchoServer {
    int server_socket;
    std::atomic<bool> running{true};
    std::atomic<size_t> current_connections{0};
    size_t max_connections;
    
public:
    EchoServer(const std::string& addr, int port, size_t max_conn) : max_connections(max_conn) {
        server_socket = socket(AF_INET, SOCK_STREAM, 0);
        sockaddr_in server_addr{AF_INET, htons(port), {INADDR_ANY}};
        bind(server_socket, (sockaddr*)&server_addr, sizeof(server_addr));
        listen(server_socket, SOMAXCONN);
    }
    
    void run() {
        while (running.load()) {
            int client = accept(server_socket, nullptr, nullptr);
            if (current_connections.load() < max_connections) {
                // C++: Each connection = new OS thread
                std::thread([this, client]() {
                    current_connections++;
                    char buffer[1024];
                    while (int n = recv(client, buffer, sizeof(buffer), 0)) {
                        send(client, buffer, n, 0); // Echo back
                    }
                    close(client);
                    current_connections--;
                }).detach();
            } else {
                close(client); // Reject if at capacity
            }
        }
    }
};
\end{lstlisting}

\subsubsection{C++ - wersja asynchroniczna z epoll/kqueue}
C++ oferuje także asynchroniczną implementację używającą \texttt{epoll} (Linux) lub \texttt{kqueue} \mbox{(macOS)}:
\begin{lstlisting}[language=C++, style=VS2017,  caption={Async Echo Serwer w C++ z event loop}, label={lst:cpp_async_echo_server}]
class AsyncEchoServer {
    int server_socket;
    std::atomic<bool> running{true};
    std::vector<int> active_clients;
    
#ifdef __APPLE__
    int kqueue_fd;  // macOS: kqueue for event multiplexing
#else
    int epoll_fd;   // Linux: epoll for event multiplexing  
#endif

public:
    AsyncEchoServer(const std::string& addr, int port) {
        server_socket = socket(AF_INET, SOCK_STREAM, 0);
        fcntl(server_socket, F_SETFL, O_NONBLOCK); // Non-blocking socket
        
        sockaddr_in server_addr{AF_INET, htons(port), {INADDR_ANY}};
        bind(server_socket, (sockaddr*)&server_addr, sizeof(server_addr));
        listen(server_socket, SOMAXCONN);
        
#ifdef __APPLE__
        kqueue_fd = kqueue();
        struct kevent ev;
        EV_SET(&ev, server_socket, EVFILT_READ, EV_ADD, 0, 0, nullptr);
        kevent(kqueue_fd, &ev, 1, nullptr, 0, nullptr);
#else
        epoll_fd = epoll_create1(0);
        struct epoll_event ev{EPOLLIN, {.fd = server_socket}};
        epoll_ctl(epoll_fd, EPOLL_CTL_ADD, server_socket, &ev);
#endif
    }
    
    void run_async() {
        while (running.load()) {
#ifdef __APPLE__
            struct kevent events[64];
            int nev = kevent(kqueue_fd, nullptr, 0, events, 64, nullptr);
            
            for (int i = 0; i < nev; ++i) {
                if ((int)events[i].ident == server_socket) {
                    handle_new_connection();
                } else {
                    handle_client_data((int)events[i].ident);
                }
            }
#else
            struct epoll_event events[64];
            int nfds = epoll_wait(epoll_fd, events, 64, -1);
            
            for (int i = 0; i < nfds; ++i) {
                if (events[i].data.fd == server_socket) {
                    handle_new_connection();
                } else {
                    handle_client_data(events[i].data.fd);
                }
            }
#endif
        }
    }
    
private:
    void handle_new_connection() {
        int client = accept(server_socket, nullptr, nullptr);
        fcntl(client, F_SETFL, O_NONBLOCK);
        active_clients.push_back(client);
        
        // Add client to event system
#ifdef __APPLE__
        struct kevent ev;
        EV_SET(&ev, client, EVFILT_READ, EV_ADD, 0, 0, nullptr);
        kevent(kqueue_fd, &ev, 1, nullptr, 0, nullptr);
#else
        struct epoll_event ev{EPOLLIN, {.fd = client}};
        epoll_ctl(epoll_fd, EPOLL_CTL_ADD, client, &ev);
#endif
    }
    
    void handle_client_data(int client_socket) {
        char buffer[1024];
        int n = recv(client_socket, buffer, sizeof(buffer), 0);
        if (n > 0) {
            send(client_socket, buffer, n, 0); // Echo back
        } else {
            // Client disconnected - remove from event system
#ifdef __APPLE__
            struct kevent ev;
            EV_SET(&ev, client_socket, EVFILT_READ, EV_DELETE, 0, 0, nullptr);
            kevent(kqueue_fd, &ev, 1, nullptr, 0, nullptr);
#else
            epoll_ctl(epoll_fd, EPOLL_CTL_DEL, client_socket, nullptr);
#endif
            close(client_socket);
            active_clients.erase(
                std::remove(active_clients.begin(), active_clients.end(), client_socket),
                active_clients.end()
            );
        }
    }
};
\end{lstlisting}

\subsection{Wydajność i bezpieczeństwo}

\begin{table}[H]
    \centering
    \caption{Porównanie wydajności i bezpieczeństwa implementacji współbieżnych}
    \begin{tabularx}{\textwidth}{lXX}
        \toprule
        \textbf{Aspekt} &
        \textbf{Rust} &
        \textbf{C++} \\
        \midrule
        Bezpieczeństwo kompilacji &
        Wysokie - system pożyczania eliminuje wyścigi danych &
        Średnie - wymaga zewnętrznych narzędzi (np. ThreadSanitizer) \\
        \hline
        Narzut w czasie wykonania &
        Niski - abstrakcje bezkosztowe \eng{zero-cost} &
        Średni - narzut tworzenia wątków \\
        \hline
        Efektywność pamięci &
        Wysoka - współdzielona własność z \texttt{Arc<T>} &
        Średnia - ryzyko wycieków pamięci \\
        \hline
        Skalowanie &
        Wysokie - model M:N, kradzież zadań &
        Ograniczone - model 1:1, ograniczenia systemowe \\
        \hline
        Odzyskiwanie po błędach &
        Strukturalne - propagacja błędów przez \texttt{Result<T, E>} &
        Oparte na wyjątkach - ryzyko wycieków zasobów \\
        \hline
        Trudność debugowania &
        Średnia - błędy kompilacji z \eng{borrow checker} &
        Wysoka - wyścigi, błędy pamięci \\
        \hline
        Dojrzałość ekosystemu &
        Rozwijający się - Tokio, async-std &
        Dojrzały - OpenMP, TBB, Boost.Asio \\
        \bottomrule
    \end{tabularx}
\end{table}

\subsection{Modele wykonania i zarządzanie zadaniami}

Implementacja Rust wykorzystuje bibliotekę Tokio, która dostarcza model wątków M:N z planistą kradzieży zadań. W praktyce oznacza to, że liczba wątków systemu operacyjnego jest niezależna od liczby zadań aplikacyjnych. Funkcja \texttt{tokio::spawn()} tworzy lekkie zadania zarządzane przez środowisko wykonawcze, co umożliwia efektywne wykorzystanie zasobów przy dużej liczbie współbieżnych operacji.

Implementacja C++ opiera się na modelu 1:1, gdzie każde połączenie obsługiwane jest przez dedykowany wątek systemu operacyjnego. Alternatywna wersja asynchroniczna wykorzystuje wywołania systemowe \texttt{epoll()} na Linux lub \texttt{kevent()} na macOS do implementacji pętli zdarzeń, jednak wymaga to jawnego zarządzania stanem połączeń.

\subsubsection{Komunikacja i synchronizacja}

W implementacji Rust komunikacja realizowana jest przez \texttt{mpsc::channel()} w konfiguracji \texttt{Arc<Mutex<Receiver<T> > >} dla wielu odbiorców. Kanał zapewnia bezpieczeństwo typów i automatyczne sprzątanie przy zamknięciu wszystkich nadawców.

Implementacja C++ definiuje własną klasę \texttt{Channel<T>} opartą na \texttt{std::queue} z synchronizacją przez \texttt{std::mutex} i \texttt{std::condition\_variable}. Klasa implementuje metody \texttt{send()}, \texttt{try\_recv()} oraz \texttt{recv()} z semantyką blokującą, wymagając ręcznego zarządzania stanem zamknięcia kanału.

\subsubsection{Obsługa I/O}

Rust wykorzystuje \texttt{TcpListener::bind().await} z nieblokującymi operacjami \mbox{wejścia/wyjścia}. Makro \texttt{tokio::select!} umożliwia równoległe oczekiwanie na wiele zdarzeń (nowe połączenia, sygnał zamknięcia). Ograniczenia połączeń realizowane są przez \texttt{Semaphore::try\_acquire\_owned()}.

C++ implementuje blokujące wejście/wyjście przez bezpośrednie wywołania systemowe: \texttt{socket()}, \texttt{bind()}, \texttt{listen()}, \texttt{accept()}. Każde połączenie obsługiwane jest w dedykowanym wątku z synchronicznymi operacjami \texttt{recv()}/\texttt{send()}. Wersja asynchroniczna wymaga ręcznego zarządzania stanem dla każdego deskryptora pliku.

\subsection{Zarządzanie błędami i zasobami}

Rust wymusza jawną obsługę błędów przez typy \texttt{Result<T, E>} i operator \texttt{?}. Automatyczne sprzątanie zasobów zapewniany jest przez cechę \texttt{Drop} - połączenia gniazd, alokacje pamięci oraz uchwyty wątków są automatycznie zwalniane.

C++ opiera się na wzorcu RAII z ręcznymi wywołaniami destruktorów. Sprzątanie gniazd wymaga explicite wywołań \texttt{close()}, a obsługa błędów realizowana jest przez rzucanie wyjątków lub sprawdzanie \texttt{errno}. Wycieki pamięci mogą wystąpić przy nieprawidłowym zarządzaniu cyklem życia objektów.


\subsubsection{Bezpieczeństwo współbieżności}

Najbardziej zasadnicza różnica między Rust a C++ dotyczy podejścia do bezpieczeństwa współbieżnego. Rust stosuje rygorystyczny system typów oraz cechy \texttt{Send} i \texttt{Sync}, które gwarantują bezpieczeństwo dostępu do danych współdzielonych już na etapie kompilacji. Kompilator odrzuca programy mogące prowadzić do wyścigów danych, co eliminuje całą klasę trudnych do wykrycia błędów.

W C++ bezpieczeństwo współbieżności opiera się na odpowiedzialności programisty. Wszelkie błędy synchronizacji (np. brak blokady, nadmierne współdzielenie danych) mogą prowadzić do wyścigów, które są trudne do wykrycia i debugowania. Choć dostępne są narzędzia analityczne, takie jak ThreadSanitizer, nie eliminują one problemów przed uruchomieniem programu i nie oferują gwarancji podobnych do Rust.



\subsection{Wnioski}

Analiza rzeczywistych implementacji ujawnia konkretne kompromisy między językami w kontekście programowania współbieżnego.
Rust z Tokio charakteryzuje się:
\begin{itemize}
    \item Bezpieczeństwem w czasie kompilacji przez system własności i cechy \texttt{Send/Sync}
    \item Architekturą sterowaną zdarzeniami z efektywnym wykorzystaniem zasobów
    \item Wymuszonym obsługą błędów eliminującym ciche awarie \eng{silent failures}
    \item Ograniczonymi możliwościami introspekcji środowiska wykonawczego
    \item Zależnością od zewnętrznych pakietów \eng{external crates} dla podstawowej funkcjonalności
\end{itemize}

C++ ze standardową biblioteką oferuje:
\begin{itemize}
    \item Bezpośrednią kontrolę nad zasobami systemowymi i układem pamięci
    \item Kompleksowe monitorowanie i profilowanie w czasie wykonania  
    \item Kompatybilność międzyplatformową z ręcznymi abstrakcjami platformowymi
    \item Model wątek na połączenie z przewidywalnym użyciem zasobów
    \item Ryzyko błędów w czasie wykonania wymagające ostrożnej synchronizacji
\end{itemize}

W praktyce wybór między językami zależy od priorytetów projektu: bezpieczeństwo a~kontrola, produktywność a elastyczność, nowoczesność a kompatybilność. Należy również uwzględnić kompetencje zespołu oraz wymagania środowiska docelowego.

%Porównanie Rust i C++
\chapter{Wnioski i rekomendacje}

\section{Omówienie wyników badań}
Przeprowadzone w ramach niniejszej pracy badania porównawcze mechanizmów programowania współbieżnego i równoległego w językach Rust i C++ na architekturach ARM64 i x86\_64 pozwoliły na sformułowanie obserwacji dotyczących wydajności, skalowania oraz charakterystyki wykorzystania zasobów systemowych. Analiza obejmowała benchmarki {NAS} {Parallel} {Benchmarks} (CG, EP, IS) dla programowania równoległego oraz implementacje wzorców producent-konsument i echo-serwer dla programowania współbieżnego.

\subsection{Wydajność implementacji równoległych}

\subsubsection{Benchmark EP (trywialnie równoległy)}

Analiza benchmarku EP, reprezentującego klasę problemów o wysokim stopniu równoległości bez zależności międzyzadaniowych, ujawniła systematyczną przewagę architektury ARM64 nad x86\_64. Implementacja Rust z biblioteką Rayon osiągnęła najwyższe wartości wydajności na platformie ARM (164 MFLOPS średnio), przewyższając zarówno Intel TBB (148 MFLOPS), jak i implementacje OpenMP. Na architekturze x86\_64 wyniki dla Rust wynosiły 109 MFLOPS, co oznacza współczynnik przyspieszenia ARM/x86 na poziomie 1,55x - najwyższy spośród wszystkich badanych implementacji.

Analiza skalowania wykazała niemal liniowy wzrost wydajności do poziomu 5-6 wątków na obu architekturach, po czym następowało spłaszczenie charakterystyki ze względu na rosnące narzuty synchronizacji. Efektywność równoległości dla 8 wątków wynosiła około 68\% dla ARM i 78\% dla x86\_64, co wskazuje na większą wrażliwość architektury ARM na koszty wielowątkowości przy wyższej liczbie wątków roboczych.

\subsubsection{Benchmark CG (gradient sprzężony)}

W przypadku benchmarku CG, charakteryzującego się intensywną komunikacją między wątkami i nieregularnymi wzorcami dostępu do pamięci, implementacja Intel TBB konsekwentnie osiągała najwyższą wydajność na obu platformach (5761 MFLOPS na ARM, 4757 MFLOPS na x86\_64). Implementacja Rust również wykazywała lepszą wydajność na platformie ARM z współczynnikiem przyspieszenia ARM/x86 wynoszącym 1,62x.

Szczególnie istotną obserwacją był wpływ architektury na wzorce zarządzania pamięcią. Analiza operacji zwolnień stron pamięci wykazała, że implementacja \texttt{new\_omp} generowała najwyższe liczby takich operacji (ponad 64 tys. dla klasy B), podczas gdy Rust charakteryzował się stosunkowo niskim poziomem zwolnień we wszystkich klasach problemowych, co wskazuje na bardziej efektywne wykorzystanie pamięci przez alokator środowiska Rust.

\subsubsection{Benchmark IS (sortowanie liczb całkowitych)}

Benchmark IS ujawnił najbardziej zróżnicowane charakterystyki wydajnościowe między implementacjami. Implementacja Rust wykazywała problemy ze skalowalnością, osiągając współczynnik przyspieszenia oscylujący wokół 1,0 niezależnie od liczby wątków. W przeciwieństwie do tego, Intel TBB demonstrował doskonałe skalowanie z~współczynnikiem przyspieszenia osiągającym 4,8x w~klasie W przy 6-8 wątkach.

Architektura ARM ponownie wykazywała przewagę wydajnościową, osiągając średnio 2,26x wyższą wydajność niż x86\_64 dla implementacji TBB. Największe różnice były widoczne w~klasach A i B (4506 vs 3458 MFLOPS), co potwierdza efektywność architektury ARM w~zadaniach intensywnie wykorzystujących przepustowość pamięci.

\subsection{Analiza mechanizmów programowania współbieżnego}

\subsubsection{Wzorzec producent-konsument}

Badanie implementacji wzorca producent-konsument ujawniło znaczące różnice w przepustowości komunikacji międzywątkowej między architekturami. Implementacje Rust z biblioteką Tokio osiągały przepustowość rzędu $10^4$--$10^5$ wiadomości/s na architekturze ARM, podczas gdy implementacje C++ z wykorzystaniem tradycyjnych wątków osiągały jedynie $10^1$--$10^2$ wiadomości/s na x86\_64. Różnica ta wynosi 1-2 rzędy wielkości na korzyść architektury ARM.

Pod względem wykorzystania zasobów, implementacje Rust charakteryzowały się znacznie niższym zużyciem pamięci (około 3 MB) w porównaniu do implementacji C++ (około 10 MB). Wykorzystanie CPU było bardziej zróżnicowane na ARM (0-100\%) niż na x86\_64 (0-30\%), co~wskazuje na bardziej dynamiczne zarządzanie zasobami na platformie macOS.

\subsubsection{Implementacja echo-serwera}

W testach echo-serwera architektura ARM osiągnęła średnią przepustowość sieciową na poziomie 100 Mbps z wartościami szczytowymi do 600 Mbps, podczas gdy x86\_64 utrzymywał średnią przepustowość na poziomie 20-30 Mbps. Implementacje C++ na ARM osiągały medianę przepustowości 170 Mbps w porównaniu do 100 Mbps dla Rust, co wskazuje na potencjał optymalizacji wydajności w C++ przy odpowiedniej konfiguracji.

Efektywność pamięciowa, wyrażona jako stosunek liczby obsłużonych wiadomości do zużytej pamięci, była znacząco wyższa na ARM (3-5 wiadomości/MB) niż na x86\_64 \mbox{(0,3-1 wiadomości/MB)}. Model asynchroniczny Rust wykazał najlepszą efektywność w tej metryce.

\section{Wpływ architektury i modeli pamięci na wyniki}

\subsection{Modele zarządzania pamięcią}

System własności i pożyczania w Rust eliminuje wyścigi danych na etapie kompilacji poprzez statyczną analizę modułu kontroli własności i pożyczania, co przekłada się na przewidywalne wzorce zarządzania pamięcią i często lepszą wydajność w scenariuszach z intensywnym wykorzystaniem pamięci. Koszt tego bezpieczeństwa przejawia się narzutem związanym z koniecznością użycia konstrukcji \texttt{Arc<Mutex<T>>} w scenariuszach współdzielenia stanu, co obserwowano jako spadek przepustowości o 8-12\% względem bezpośredniego dostępu do pamięci współdzielonej w C++.

Model C++ oferuje większą elastyczność kosztem ryzyka błędów związanych z zarządzaniem pamięcią. Intel TBB wykorzystuje zaawansowane optymalizacje lokalności pamięci podręcznej, co jest szczególnie widoczne w wynikach benchmarku CG, gdzie różnice w wydajności między implementacjami sięgały 12-18\% na korzyść TBB ze względu na lepsze zarządzanie fragmentacją pamięci przy dynamicznych kontenerach.

\subsection{Architektura sprzętowa}

Przewaga architektury ARM64 (Apple M1) może być przypisana kilku czynnikom technologicznym. Heterogeniczna architektura \emph{big.LITTLE} z 4 rdzeniami wydajnościowymi (P-cores) i~4 energooszczędnymi (E-cores) umożliwia efektywniejsze zarządzanie obciążeniem wielowątkowym. Ujednolicona architektura pamięci \eng{unified memory architecture}, eliminuje koszty kopiowania danych między CPU a pamięcią, co jest szczególnie korzystne w aplikacjach intensywnie wykorzystujących przepustowość pamięci.

Hierarchiczna pamięć podręczna (12MB L2 dla P-cores, 4MB dla E-cores) przyczynia się do poprawy wydajności w warunkach zmiennego obciążenia. Niemniej jednak, jak wskazują autorzy \cite{arml2c}, główne ograniczenia wynikają nie tyle z przepustowości przy intensywnej komunikacji między wątkami, ile z nieoptymalnego wykorzystania pamięci podręcznej, zwłaszcza w~kontekście lokalności i alokacji danych w środowiskach wielowątkowych, co obserwowano jako szybszy spadek efektywności skalowania ARM przy większej liczbie wątków.

\subsection{Wpływ środowiska kompilacyjnego i bibliotek}

Zidentyfikowane różnice między kompilatorami Clang (macOS) a GCC (Linux) wprowadzały zmienność wyników na poziomie 10-15\% w testach współbieżności. Clang wykazywał lepsze wsparcie dla instrukcji wektorowych na ARM, podczas gdy GCC zapewniał skuteczniejsze mechanizmy optymalizacyjne dla architektury x86\_64. 

W trakcie prac nad implementacją aplikacji w języku C++ z wykorzystaniem biblioteki Threading Building Blocks (TBB) zaobserwowano istotne różnice w procesie budowania i konfiguracji projektu pomiędzy platformami opartymi na architekturze x86-64 (Linux) a systemem macOS z procesorem Apple M1 (ARM64), co jest również potwierdzone przez pracę \cite{ARMTBB}. Aplikacja, która kompilowała się bezproblemowo i działała optymalnie w środowisku x86, wymagała licznych modyfikacji przy próbie przeniesienia jej na platformę Apple Silicon. W szczególności konieczne było ręczne dostosowanie flag kompilatora, aktualizacja konfiguracji CMake z~uwzględnieniem architektury ARM oraz zastosowanie społecznościowych łatek w celu rozwiązania problemów z rozpoznawaniem architektury \texttt{(Unknown architecture flag: -arch armv4t)}. Te trudności potwierdzają, że proces przenoszenia aplikacji opartych na TBB na \mbox{macOS} z procesorem M1 nie jest trywialny i wymaga świadomego podejścia projektowego oraz głębszego zrozumienia różnic międzyplatformowych -- zarówno na poziomie sprzętowym, jak i systemowym. Dodatkowo jak podają autorzy \cite{TBBARMCONCLUSIONS}, testy na maszynie wirtualnej z emulacją Intel wykazały spadek wydajności w porównaniu z natywnym wykonaniem na M1.

Klasa \texttt{std::jthread} została wprowadzona dopiero w standardzie C++20, który nie jest w pełni wspierany przez kompilator Apple Clang w wersji 15.0.0. Ograniczenia te należy uwzględnić przy tworzeniu aplikacji na system macOS, co wymusza zastosowanie alternatywnych rozwiązań programistycznych zapewniających kompatybilność z daną wersją kompilatora. Analiza wyników działania CMake wskazuje, że nawet przy użyciu kompilatora Clang w wersji 20.1.6, klasa \texttt{std::jthread} pozostaje niedostępna. Wynika to z faktu, iż jej obsługa zależy nie tylko od samego kompilatora, lecz również od implementacji biblioteki standardowej C++. W~środowisku macOS, mimo wykorzystania najnowszej wersji Clanga, systemowa wersja biblioteki \texttt{libc++} może nie zawierać jeszcze implementacji \texttt{std::jthread}. W związku z tym pełna zgodność ze standardem C++20 w zakresie zarządzania wątkami wymaga nie tylko odpowiedniego kompilatora, ale także aktualnej wersji biblioteki standardowej.

\section{Rekomendacje praktyczne}

\subsection{Dobór technologii w zależności od charakterystyki problemu}

\subsubsection{Obliczenia typu \emph{embarrassingly parallel}}

Na podstawie wyników benchmarku EP, rekomenduje się Rust z biblioteką Rayon jako rozwiązanie główne ze względu na najwyższą osiągniętą wydajność (164 MFLOPS na ARM), bezpieczeństwo pamięci oraz abstrakcje bezkosztowe. Alternatywnie, dla istniejących baz kodu C++, zaleca się Intel TBB z uwagi na dobre skalowanie i dojrzałość rozwiązania.

\subsubsection{Algorytmy z intensywną komunikacją międzywątkową}

Dla algorytmów iteracyjnych i symulacji numerycznych zaleca się C++ z Intel TBB jako rozwiązanie główne, co potwierdza przewaga w benchmarkach CG i IS. Dojrzałe optymalizacje pamięci podręcznej oraz zaawansowane techniki przydzielania zadań w czasie wykonywania czynią TBB optymalnym wyborem dla tej klasy problemów. Rust wymaga dalszych optymalizacji w tym obszarze, szczególnie w kontekście skalowalności przy nieregularnych wzorcach dostępu do pamięci.

\subsubsection{Aplikacje sieciowe wysokiej przepustowości}

Rekomenduje się Rust z biblioteką Tokio ze względu na model asynchroniczny M:N, wysoką efektywność pamięciową (3-5 wiadomości/MB) oraz wbudowane mechanizmy \texttt{async/await}. C++ może być rozważany w przypadkach wymagających precyzyjnej kontroli latencji pojedynczych operacji, szczególnie przy wykorzystaniu zaawansowanych bibliotek asynchronicznych.

\subsection{Wybór platformy sprzętowej}

Wybór platformy sprzętowej powinien być dostosowany do charakteru zadań oraz wymagań aplikacji. Architektura ARM64 (Apple Silicon) jest szczególnie korzystna dla obciążeń o zmiennym charakterze oraz systemów, w których kluczowa jest efektywność energetyczna. Z kolei architektura x86\_64 pozostaje preferowanym rozwiązaniem w przypadku klasycznych obliczeń wysokiej wydajności (HPC), środowisk wymagających pełnej kontroli nad przypisaniem wątków do procesorów, a także tam, gdzie istotne jest korzystanie z dojrzałego i stabilnego ekosystemu narzędzi oraz bibliotek. Ponadto, architektura ta gwarantuje kompatybilność z istniejącym kodem \eng{legacy code} i umożliwia precyzyjną kontrolę nad szczegółami implementacji.

\subsection{Rekomendacje dotyczące środowiska kompilacyjnego}

Dla C++ zaleca się wykorzystanie kompilatora Clang zamiast GCC do kompilacji kodu równoległego, co w badaniach skutkowało skróceniem czasu wykonania o 10-15\% dzięki lepszemu wsparciu dla instrukcji wektorowych. W Rust zaleca się stosowanie Rayon do równoległego przetwarzania danych oraz Tokio do aplikacji asynchronicznych w celu minimalizacji narzutu kontekstu.

\section{Ograniczenia metodologiczne badania}

\subsection{Heterogeniczność środowisk testowych}

Największym ograniczeniem przeprowadzonych badań była niemożność zapewnienia jednolitego środowiska testowego. Wykorzystanie różnych kompilatorów (Clang 14.0.3 na macOS a~GCC 11.4.0 na Linux), różnych systemów operacyjnych (macOS z jądrem Darwin vs Linux) oraz różnych implementacji bibliotek standardowych (\texttt{libc++} vs \texttt{libstdc++}) wprowadzało zmienne systemowe mogące wpływać na wyniki porównań.

\subsection{Ograniczenia narzędzi diagnostycznych}

Przeprowadzone badanie ujawniło fundamentalne ograniczenie narzędzia \texttt{hwloc-ps} na platformie macOS, szczególnie w architekturze Apple Silicon, wynikające z celowych restrykcji systemowych w jądrze XNU. Jak wykazano w \cite{HWLOC555}, brak implementacji interfejsów \texttt{sched\_setaffinity()}/\texttt{sched\_getaffinity()} uniemożliwia odczyt i kontrolę przypisań procesów do rdzeni \eng{process-to-core binding}, co stanowi barierę metodologiczną w porównawczych badaniach wydajnościowych między architekturami ARM (M1) i x86\_64. W~odróżnieniu od pełnej funkcjonalności \texttt{hwloc-ps} w systemie Linux \cite{hwlocHardwareLocality}, gdzie narzędzie precyzyjnie raportuje przypisania wątków, na macOS możliwe jest jedynie wykrywanie topologii sprzętowej przez \texttt{lstopo} - przy użyciu mechanizmów \texttt{sysctl}. Ograniczenie to uniemożliwiło bezpośrednią weryfikację wpływu przypisań wątków \eng{thread pinning} na wydajność implementacji współbieżnych w językach Rust/C++.

Różnice w narzędziach profilowania (\texttt{Instruments} na macOS a \texttt{perf} na Linux) mogły wprowadzać dodatkową zmienność w pomiarach wykorzystania zasobów.

\subsection{Ograniczenia doboru benchmarków}

Wybrane benchmarki NAS (CG, EP, IS) reprezentują głównie problemy o regularnej strukturze obliczeniowej charakterystyczne dla klasycznych zastosowań HPC. Współczesne aplikacje często charakteryzują się nieregularnymi wzorcami dostępu do pamięci, dynamiczną alokacją zadań oraz hybrydowymi modelami obliczeniowymi, co ogranicza generalizację wniosków do innych domen aplikacyjnych.

\section{Kierunki dalszych badań}

\subsection{Ujednolicone środowisko badawcze}

Priorytetem dla przyszłych badań powinno być przeprowadzenie eksperymentów na maszynach Apple wyposażonych zarówno w procesory Intel, jak i Apple Silicon, co pozwoliłoby na:
\begin{itemize}
    \item Eliminację zmienności wynikającej z różnic systemowych między macOS a Linux
    \item Użycie identycznego kompilatora (Clang) i bibliotek systemowych
    \item Precyzyjną izolację wpływu architektury procesora od wpływu środowiska systemowego
    \item Kontrolę nad identycznością mechanizmów szeregowania zadań
\end{itemize}

\subsection{Rozszerzenie zakresu architektur i platform}

\subsubsection{Architektury heterogeniczne}

Dalsze badania powinny uwzględnić systemy hybrydowe wykorzystujące:
\begin{itemize}
    \item GPU poprzez CUDA (C++) i rust-gpu/wgpu (Rust)
    \item Systemy SoC z dedykowanymi akceleratorami (NPU, DSP)
    \item Platformy FPGA w kontekście programowania wysokiej wydajności
    \item Procesory RISC-V jako alternatywę dla ARM i x86
\end{itemize}

\subsubsection{Nowe paradygmaty programowania}

Eksploracja powinna obejmować:
\begin{itemize}
    \item Model aktorowy (Actix dla Rust a CAF dla C++)
    \item programowanie przepływu danych oraz reaktywne strumienie danych
    \item Mechanizmy C++20 \texttt{coroutines} a Rust \texttt{async/await}
    \item Hybrydowe modele łączące synchroniczne i asynchroniczne wzorce
\end{itemize}


\section{Wkład pracy}

\subsubsection{Pierwsze porównanie wydajności benchmarków NAS Parallel Benchmarks na architekturach ARM64 oraz x86\_64 w językach Rust i C++}

Niniejsza praca stanowi pierwsze w literaturze kompleksowe porównanie wydajności mechanizmów programowania równoległego między językami Rust i C++ z wykorzystaniem standardowych benchmarków NAS Parallel Benchmarks na architekturach ARM64 \mbox{(Apple Silicon)} i x86\_64. Dotychczasowe badania koncentrowały się głównie na platformach x86 lub nie uwzględniały nowoczesnych architektur ARM w kontekście zastosowań HPC. Wykazanie systematycznej przewagi architektury ARM64 w większości testowanych scenariuszy (współczynniki przyspieszenia 1,55x--2,26x) stanowi istotne odkrycie dla społeczności zajmującej się obliczeniami wysokiej wydajności.

\subsubsection{Empiryczna walidacja abstrakcji wysokopoziomowych w programowaniu równoległym}

Badania empirycznie potwierdzają tezę, że nowoczesne abstrakcje wysokopoziomowe (Rust Rayon, Intel TBB) nie tylko zwiększają bezpieczeństwo programowania, ale również oferują wydajność konkurencyjną lub wyższą od tradycyjnych rozwiązań niskopoziomowych (klasyczne OpenMP). Szczególnie znaczące jest wykazanie, że biblioteka Rayon osiąga najwyższą wydajność w benchmarku EP (164 MFLOPS na ARM), przewyższając dojrzałe implementacje OpenMP i TBB.

\subsubsection{Analiza wpływu modeli zarządzania pamięcią na wydajność współbieżną w kontekście architektur ARM64 i x86\_64}

Niniejsza praca wnosi nowe spojrzenie na oddziaływanie różnych modeli zarządzania pamięcią, w~szczególności systemu własności w języku Rust oraz manualnego zarządzania pamięcią w~C++, na wydajność aplikacji wielowątkowych. Przeprowadzone systematyczne porównania na architekturach ARM64 oraz x86\_64 umożliwiły empiryczne wykazanie, że eliminacja wyścigów danych na etapie kompilacji w Rust przekłada się na przewidywalne wzorce wykorzystania pamięci. Obserwowany narzut wydajnościowy różni się w zależności od platformy - wynosi około 8-12\% na architekturze x86\_64, podczas gdy na ARM64 ujawnia inne charakterystyki. Wyniki te dostarczają istotnych informacji umożliwiających lepsze zrozumienie kompromisów pomiędzy bezpieczeństwem a efektywnością w programowaniu systemowym, zwłaszcza w kontekście środowisk wieloplatformowych.

\subsubsection{Identyfikacja ograniczeń metodologicznych w badaniach wieloplatformowych}

Badania ujawniły ograniczenia w dostępnych narzędziach diagnostycznych dla architektur ARM na platformie macOS, szczególnie w kontekście \texttt{hwloc-ps} i mechanizmów przypisania procesu do konkretnego rdzenia procesora. Zidentyfikowanie tych ograniczeń oraz ich wpływu na wiarygodność pomiarów między architekturami może pomóc przyszłym badaczom w wyborze narzędzi pomiarowych.

\subsubsection{Walidacja efektywności architektur heterogenicznych w programowaniu równoległym}

Wyniki potwierdzają teoretyczne założenia dotyczące korzyści płynących z heterogenicznych architektur \emph{big.LITTLE} (Apple M1) w kontekście aplikacji wielowątkowych. Empiryczne wykazano, że architektura ta zapewnia lepszą wydajność przy zmiennym obciążeniu, przy jednoczesnej identyfikacji jej ograniczeń przy wysokiej liczbie wątków (spadek efektywności do 68\% a 78\% na x86\_64).

\subsubsection{Praktyczne wytyczne dla wyboru technologii w programowaniu współbieżnym}

Na podstawie systematycznych pomiarów wydajności sformułowano empirycznie uzasadnione wytyczne dla wyboru języka programowania i bibliotek w zależności od charakterystyki problemu obliczeniowego i docelowej architektury. Rekomendacje te, oparte na konkretnych danych wydajnościowych, stanowią praktyczny wkład dla inżynierów oprogramowania projektujących systemy współbieżne i równoległe.

\section{Podsumowanie końcowe}

Przeprowadzone badania wykazały, że wybór między Rust a C++ oraz między różnymi bibliotekami równoległości powinien być podyktowany specyfiką problemu obliczeniowego, wymaganiami wydajnościowymi, docelową architekturą sprzętową oraz kontekstem organizacyjnym. Nie istnieje uniwersalnie optymalne rozwiązanie - każde z badanych podejść wykazuje przewagi w określonych scenariuszach.

Rust z biblioteką Rayon wyłania się jako silna alternatywa dla C++ w dziedzinie programowania systemowego i współbieżnego, oferując unikalne gwarancje bezpieczeństwa przy zachowaniu wysokiej wydajności, szczególnie na architekturach ARM. C++, z bogatym ekosystemem i dojrzałymi narzędziami takimi jak Intel TBB, pozostaje kluczowym językiem dla aplikacji wymagających najwyższej wydajności, zwłaszcza na architekturach x86\_64 w zastosowaniach HPC.

Architektura ARM64, reprezentowana przez Apple Silicon, udowodniła swoją konkurencyjność względem x86\_64, oferując często wyższą wydajność przy niższym zużyciu energii. Wymaga to jednak adaptacji istniejących narzędzi i bibliotek, co stanowi wyzwanie dla szerokiego przyjęcia.

Dalszy rozwój obu języków, postępująca dywersyfikacja platform sprzętowych oraz rosnące wymagania aplikacji będą stymulować ewolucję mechanizmów programowania współbieżnego jak i równoległego. Kontynuacja badań porównawczych, uwzględniająca nowe architektury i paradygmaty programowania, pozostaje niezbędna dla świadomego wyboru technologii w szybko zmieniającym się środowisku obliczeniowym.


% The option of developing new computer languages may be the clean- est and most efficient way to provide support for parallel processing. However, practical issues make the wide acceptance of a new computer language close to impossible. Nobody likes to rewrite old code to new lan- guages. It is difficult to justify such effort in most cases. Also, educating and convincing a large enough group of developers to make a new lan- guage gain critical mass is an extremely difficult task.
\chapter{Podsumowanie}
W ramach niniejszej pracy magisterskiej zrealizowano porównawczą analizę mechanizmów programowania współbieżnego i równoległego w językach Rust i C++ na architekturach ARM64 (Apple Silicon M1) oraz x86\_64 (Intel). Zastosowana metodologia eksperymentalna, wykorzystująca zestawy testowe NAS Parallel Benchmarks oraz implementacje aplikacji współbieżnych, umożliwiła systematyczne porównanie wydajności bibliotek Rust Rayon, Intel TBB oraz OpenMP.

Badania wykazały znaczącą przewagę architektury ARM64 nad x86\_64 we wszystkich scenariuszach testowych, ze współczynnikami przyspieszenia 1,55x-2,26x. Implementacja z~biblioteką Rayon osiągnęła najwyższą wydajność w obliczeniach trywialnie równoległych \mbox{(164 vs 109 MFLOPS)}, podczas gdy Intel TBB dominowała w zadaniach z intensywną komunikacją międzywątkową. Kluczowym odkryciem było wykazanie, że model bezpieczeństwa pamięci w Rust nie generuje istotnych narzutów wydajnościowych, a w niektórych przypadkach przyczynia się do lepszej efektywności. W aplikacjach sieciowych architektura ARM uzyskała przepustowość o rząd wielkości wyższą przy zmniejszonym zużyciu pamięci.

Główne ograniczenia metodologiczne obejmowały heterogeniczność środowisk testowych (macOS vs Linux) oraz ograniczenia narzędzi diagnostycznych na platformie macOS, zwłaszcza w kontekście kontroli przypisań wątków do rdzeni. Dodatkowo, wybrane benchmarki NAS reprezentują problemy o regularnej strukturze obliczeniowej, co ogranicza generalizację wniosków do aplikacji o nieregularnych wzorcach dostępu do pamięci.

Praca wypełnia istotną lukę badawczą, dostarczając kompleksowej analizy porównawczej mechanizmów programowania równoległego w kontekście nowoczesnych architektur sprzętowych. Wyniki potwierdzają porównywalną wydajność Rust względem C++, jednocześnie uwydatniając wpływ architektury na efektywność mechanizmów współbieżności.

W kontekście dalszych badań sugeruje się ujednolicenie środowiska testowego poprzez wykorzystanie maszyn Apple z procesorami Intel i Apple Silicon, rozszerzenie zakresu na architektury heterogeniczne oraz eksplorację nowych paradygmatów programowania, w tym modelu aktorowego i mechanizmów C++20 \texttt{coroutines} w porównaniu z Rust \texttt{async/await}.

Podsumowując, przeprowadzone badania pogłębiają wiedzę na temat charakterystyk wydajnościowych nowoczesnych języków programowania w obliczeniach równoległych oraz wskazują na rosnący potencjał architektury ARM64 w zastosowaniach wysokowydajnościowych. Praca stanowi solidną podstawę do dalszych badań nad optymalizacją mechanizmów współbieżności w systemach wymagających wysokiej wydajności i niezawodności.

%\show\chapter
%\show\section
%\show\subsection

%\showthe\secindent
%\showthe\beforesecskip
%\showthe\aftersecskip
%\showthe\secheadstyle
%\showthe\subsecindent
%\showthe\beforesubsecskip
%\showthe\aftersubsecskip
%\showthe\subseccheadstyle
%\showthe\parskip




% SPIS RYSUNKÓW (zostanie wygenerowany automatycznie)
\pdfbookmark[0]{Spis rysunków}{spisRysunkow.1} % jeśli chcemy mieć w spisie treści, to zamarkować tę linię, a odmarkować linie poniższe
\phantomsection
\addcontentsline{toc}{chapter}{Spis rysunków}
\listoffigures*
\clearpage

% SPIS TABEL (zostanie wygenerowany automatycznie)
\pdfbookmark[0]{Spis tabel}{spisTabel.1} %
\phantomsection
\addcontentsline{toc}{chapter}{Spis tabel}
\listoftables*
\clearpage

% SPIS LISTINGÓW (zostanie wygenerowany automatycznie)
\pdfbookmark[0]{Spis listingów}{spisListingow.1} %
\phantomsection
\addcontentsline{toc}{chapter}{Spis listingów}
\lstlistoflistings*

% LITERATURA (zostanie wygenerowana automatycznie)
%UWAGA: bibliotekę referencji należy przygotować samemu. Dobrym do tego narzędziem jest JabRef.
%       JabRef oferuje jednak większą liczbę typów rekordów niż obsługuje BibTeX.
%       Proszę nie deklarować rekordów o typach nieobsługiwanych przez BibTeX.
%       Formatowania wykazu literatury i cytowań odbywać się ma zgodnie z zadeklarowanym stylem.
%       Zalecane są style produkujące numeryczne cytowania (w postaci [1], [2,3]).
%       Takim stylem jest np. plabbrv
\bibliographystyle{plabbrv}
%       Aby zapanować nad odstępami w wykazie literatury można posłużyć się poniższą komendą
\setlength{\bibitemsep}{2pt} % - zacieśnia wykaz
%       Pozycja Literatura pojawia się w spisie treści nieco inaczej niż spisy rysunków, tabel itp.
%       Aby zachować właściwe odstępy należy użyć poniższej komendy
\addtocontents{toc}{\addvspace{2pt}} % ustawiamy odstęp w spisie treści przed pozycją Literatura 
%       Nazwę pliku przygotowanej biblioteki wpisuje się bez rozszerzenia .bib
%       (linia poniżej załaduje rekordy z pliku "dokumentacja.bib")
\bibliography{dokumentacja}
\nocite{*}

\appendix
%\chapter{Instrukcja wdrożeniowa}
Jeśli praca kończy się tworzeniem oprogramowania, to w dodatku powinna znaleźć się instrukcja wdrożeniowa, która opisuje, jak skompilować lub zainstalować to oprogramowanie.

Dodatkowo, przyda się krótka instrukcja ,,how to'' (jak uruchomić system i wykonać podstawową czynność). Można to zaprezentować na prostym przypadku użycia. Warto rozważyć umieszczenie tych informacji w osobnym dodatku.

%\chapter{Opis załączonej płyty CD/DVD}
\label{chap:opis-plyty}
Ten rozdział jest miejscem, w którym zamieszczamy opis zawartości dołączonej płyty. Należy pamiętać, że opis ten jest przygotowywany przed załadowaniem pracy do systemu APD USOS, dlatego nie znamy jeszcze nazwy, jaką ten system wygeneruje dla załadowanego pliku. Dlatego warto stosować ogólniki typu: "Na płycie zamieszczono dokument PDF z tekstem pracy" bez wskazywania nazwy konkretnego pliku.

Wcześniej obowiązywała reguła, aby nadawać dokumentom nazwy zgodnie z wzorcem "W04\_[numer albumu]\_[rok kalendarzowy]\_[rodzaj pracy]". Rok kalendarzowy odnosił się do roku realizacji kursu "Praca dyplomowa", a nie roku obrony. Na przykład, wzorzec nazwy dla pracy dyplomowej inżynierskiej wyglądałby tak: "W04\_123456\_2015\_praca inżynierska.pdf". Takie nazwy były utrwalane w systemie składania prac dyplomowych. Jednak obecnie procedura jest inna.


% Jeśli w pracy pojawiać się ma indeks, należy odkomentować poniższe linie
%%\chapterstyle{noNumbered}
%%\phantomsection % sets an anchor
%%\addcontentsline{toc}{chapter}{Indeks rzeczowy}
%%\printindex

\end{document}
